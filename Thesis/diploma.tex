\documentclass[12pt,a4paper]{amsart}
% ukazi za delo s slovenscino -- izberi kodiranje, ki ti ustreza
\usepackage[slovene]{babel}
\usepackage[T1]{fontenc}
\usepackage[utf8]{inputenc}
\usepackage{amsmath,amssymb,amsfonts,amsthm}
\usepackage{url}
%\usepackage[normalem]{ulem}
\usepackage[dvipsnames,usenames]{color}
\usepackage{graphicx}
\usepackage{tikz}
\usepackage{dsfont}
\usepackage{caption}
\usepackage{subcaption}
\usepackage{bm}
\usepackage{float}

% ne spreminjaj podatkov, ki vplivajo na obliko strani
\textwidth 15cm
\textheight 24cm
\oddsidemargin.5cm
\evensidemargin.5cm
\topmargin-5mm
\addtolength{\footskip}{10pt}
\pagestyle{plain}
\overfullrule=15pt % oznaci predlogo vrstico


% ukazi za matematicna okolja
\theoremstyle{definition} % tekst napisan pokoncno
\newtheorem{definicija}{Definicija}[section]
\newtheorem{primer}[definicija]{Primer}
\newtheorem{opomba}[definicija]{Opomba}

\renewcommand\endprimer{\hfill$\diamondsuit$}


\theoremstyle{plain} % tekst napisan posevno
\newtheorem{lema}[definicija]{Lema}
\newtheorem{izrek}[definicija]{Izrek}
\newtheorem{trditev}[definicija]{Trditev}
\newtheorem{posledica}[definicija]{Posledica}



% ukaz za slovarsko geslo
\newlength{\odstavek}
\setlength{\odstavek}{\parindent}
\newcommand{\geslo}[2]{\noindent\textbf{#1}\hspace*{3mm}\hangindent=\parindent\hangafter=1 #2}


% naslednje ukaze ustrezno popravi
\newcommand{\program}{Finančna matematika} % ime studijskega programa: Matematika/Finan"cna matematika
\newcommand{\imeavtorja}{Anej Rozman} % ime avtorja
\newcommand{\imementorja}{~doc.~dr. Martin Raič} % akademski naziv in ime mentorja
\newcommand{\naslovdela}{Sestavljen Poissonov proces in njegova uporaba v financah} % naslov dela
\newcommand{\letnica}{2024} % letnica diplome

% Moji ukazi
\newcommand{\R}{\mathbb{R}}
\newcommand{\N}{\mathbb{N}}
\newcommand{\E}{\mathbb{E}}
\newcommand{\F}{\mathcal{F}}
\newcommand{\Prob}{\mathbb{P}}
\newcommand{\1}{\mathds{1}}
\newcommand{\Pois}[1]{\text{Pois}(#1)}





\begin{document}

% od tod do povzetka ne spreminjaj nicesar
\thispagestyle{empty}
\noindent{\large
UNIVERZA V LJUBLJANI\\[1mm]
FAKULTETA ZA MATEMATIKO IN FIZIKO\\[5mm]
\program\ -- 1.~stopnja}
\vfill

\begin{center}{\large
\imeavtorja\\[2mm]
{\bf \naslovdela}\\[10mm]
Delo diplomskega seminarja\\[1cm]
Mentor: \imementorja}
\end{center}
\vfill

\noindent{\large
Ljubljana, \letnica}
\pagebreak

\thispagestyle{empty}
\tableofcontents
\pagebreak

\thispagestyle{empty}
\begin{center}
{\bf \naslovdela}\\[3mm]
{\sc Povzetek}
\end{center}
% tekst povzetka v slovenscini

\vfill
\begin{center}
{\bf Compound Poisson process and its application in finance}\\[3mm] % angleski naslov
{\sc Abstract}
\end{center}
% tekst povzetka v anglescini
Prevod zgornjega povzetka v angle"s"cino.

\vfill\noindent
{\bf Math. Subj. Class. (2010):} 91G10 60G00 60G01  \\[1mm]
{\bf Klju"cne besede:} Slu"cajni procesi, Sestavljen Poissonov proces  \\[1mm]
{\bf Keywords:} Stochastic processes, Lévy processes
\pagebreak



% tu se zacne besedilo seminarja
\section{Uvod}

    Poissonov proces "steje "stevilo prihodov v danem "casovnem intervalu, kjer narava prihodov 
    sledi dolo"cenim omejitvam. Sestavljen Poissonov proces, je podoben 
    Poissonovemu procesu, razen da je vsak prihod ute"zen glede na porazdelitev. Na primer, stranke, 
    ki gredo v trgovino, sledijo Poissonovemu procesu, znesek denarja, ki ga porabijo, pa lahko 
    sledi sestavljenemu Poissonovemu procesu. Hitro vidimo, da je to zelo zanimiv a ideja slu"cajnega
    procesa, ki ima veliko potencialnih uporab. V delu se bomo osredoto"cili na njegovo uporabo v 
    financah. Za za"cetek definirajmo osnovne pojme ter Sestavljen Poissonov proces.

    \begin{definicija}
        Naj bo $(\Omega, \mathcal{F}, \mathbb{P})$ verjetnostni prostor in naj bo $T\neq\emptyset$
        neprazna indeksna množica ter $(S, \Sigma)$ merljiv prostor. \textit{Slučajni proces}, 
        parametriziran s $T$, je družina slučajnih spremenljivk $X_t : \Omega \to S$,
         ki so $(\mathcal{F}, \Sigma)$-merljive za vsak $t \in T$.
    \end{definicija}

    \begin{opomba}
        Dr"zali se bomo konvencije, da $T$ predstavlja "cas, torej $T = [0, \infty)$.
        V tem primeru govorimo o zveznem slu"cnem procesu.
    \end{opomba}

    \begin{definicija}
        Za fiksen $\omega \in \Omega$ je preslikava 
        $[0, \infty) \rightarrow \mathbb{R}; \ t \mapsto X_t(\omega)$ 
        \textit{trajektorija} oziroma \textit{realizacija} slučajnega procesa $(X_t)_{t\geq0}$.
    \end{definicija}

    \begin{opomba}
        Na slu"cajni proces lahko gledamo tudi kot na predpis, ki nam iz vor"cnega prostora 
        $\Omega$ priredi slu"cajno funkcijo
        $(X_t(\omega))_{t\geq0}: [0, \infty) \rightarrow \mathbb{R}$.
    \end{opomba}

    \begin{definicija}
        Naj bo $(X_t)_{t\geq0}$ slu"cajni proces. Potem za $s < t$ definiramo
        \textit{prirastek procesa} $X_t - X_s$ na intervalu $[s, t]$. Proces $(X_t)_{t\geq0}$ ima 
        \textit{neodvisne prirastke}, če so za vsak nabor realnih "stevil
        $0 \leq t_1 < t_2 < \ldots < t_n < \infty$ prirastki (slučajne spremenljivke)
        $$
            X_{t_2} - X_{t_1}, \ X_{t_3} - X_{t_2}, \ \ldots, \ X_{t_n} - X_{t_{n-1}}
        $$
        med seboj neodvisni.
    \end{definicija}

    \begin{definicija}
        Naj bo $(X_t)_{t\geq0}$ slu"cajni proces. Potem pravimo, da ima proces
        \textit{stacionarne prirastke}, "ce za vsak $s < t$ in vsak $h > 0$ velja, 
        da ima $X_{t+h} - X_{s+h}$ enako porazdelitev kot $X_t - X_s$.
    \end{definicija}

    \begin{definicija}
        Naj bo $\lambda > 0$. Slučajnemu procesu $(N_t)_{t\geq 0}$ definiranem na verjetnostnem 
        prostoru $(\Omega, \mathcal{F}, \mathbb{P})$ z vrednostmi v $\mathbb{R}$ pravimo 
        \textit{Poissonov proces} z intenzivnostjo $\lambda$, če zadošča naslednjim pogojem:
        \begin{enumerate}
            \item $B_0 = 0$ skoraj gotovo,
            \item proces ima neodvisne prirastke,
            \item prirastki so porazdeljeni $\Pois{\lambda t}$,
            \item proces ima stacionarne prirastke.
        \end{enumerate}
    \end{definicija}

    \begin{definicija}
        Naj bo $(N_t)_{t\geq0}$ Poissonov proces z intenzivnostjo $\lambda$. 
        Naj bo $(X_i)_{i\geq1}$ zaporedje neodvisnih in enako porazdeljenih slučajnih spremenljivk 
        z vrednostmi v $\mathbb{R}$. Potem je \textit{sestavljen Poissonov proces} 
        $(S_t)_{t\geq0}$ definiran kot
        $$
            S_t = \sum_{i=1}^{N_t} X_i.
        $$
    \end{definicija}


\section*{Slovar strokovnih izrazov}

%\geslo{}{}
%
%\geslo{}{}
%


% seznam uporabljene literature
\begin{thebibliography}{99}

%\bibitem{}

\end{thebibliography}

\end{document}
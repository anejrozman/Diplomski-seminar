\documentclass[12pt, a4paper, reqno]{amsart}
\usepackage[slovene]{babel}
\usepackage[T1]{fontenc}
\usepackage[utf8]{inputenc}
\usepackage{amsmath,amssymb,amsfonts,amsthm}
\usepackage{url}
\usepackage[dvipsnames,usenames]{color}
\usepackage{graphicx}
\usepackage{tikz}
\usepackage{dsfont}
\usepackage{caption}
\usepackage{subcaption}
\usepackage{bm}
\usepackage{float}
\usepackage{xcolor}
\usepackage{eurosym}
\usepackage{bbm}
\usepackage{calligra}
\usepackage{calc}
\usepackage{xcolor}
\definecolor{propercyan}{HTML}{14bbff}
\definecolor{properpurple}{HTML}{b409e7}
\usepackage[
    colorlinks=true,
    linkcolor=propercyan,      
    citecolor=propercyan,    
    filecolor=magenta,      
    urlcolor=propercyan,   
]{hyperref}
\usepackage{footnotebackref}
%\usepackage{titlesec}
\allowdisplaybreaks

%--------------------------------------OBLIKA DOKUMENTA--------------------------------------------%

% Oblika strani
\textwidth 15cm
\textheight 24cm
\oddsidemargin.5cm
\evensidemargin.5cm
\topmargin-5mm
\addtolength{\footskip}{10pt}
\pagestyle{plain}
\overfullrule=15pt % oznaci predlogo vrstico

% Naslovi
%\titleformat{\section}
%{\normalfont\Large\bfseries\centering\MakeUppercase}{\thesection}{1em}{}
%
%\titleformat{\subsection}
%{\normalfont\large\bfseries}{\thesubsection}{1em}{}
%
%\titleformat{\subsubsection}
%{\normalfont\normalsize\bfseries}{\thesubsubsection}{1em}{}


% Matematicna okolja (tekst napisan pokoncno)
\theoremstyle{definition}
\newtheorem{definicija}{Definicija}[section]
\newtheorem{zgled}[definicija]{Zgled}
\newtheorem{opomba}[definicija]{Opomba}

\renewcommand\endzgled{\hfill$\diamondsuit$}

% Matematicna okolja (tekst napisan posevno)
\theoremstyle{plain} 
\newtheorem{lema}[definicija]{Lema}
\newtheorem{izrek}[definicija]{Izrek}
\newtheorem{trditev}[definicija]{Trditev}
\newtheorem{posledica}[definicija]{Posledica}



% Ukaz za slovarsko geslo
\newlength{\odstavek}
\setlength{\odstavek}{\parindent}

% Podatki o delu
\newcommand{\program}{Finančna matematika} 
\newcommand{\imeavtorja}{Anej Rozman} 
\newcommand{\imementorja}{~doc.~dr. Martin Raič} 
\newcommand{\naslovdela}{Sestavljeni Poissonov proces in njegova uporaba v financah} 
\newcommand{\letnica}{2024} 

% Dodatni ukazi
\newcommand{\R}{\mathbb{R}}
\newcommand{\N}{\mathbb{N}}
\newcommand{\E}{\mathbb{E}}
\newcommand{\F}{\mathcal{F}}
\newcommand{\B}{\mathcal{B}}
\newcommand{\Prob}{\mathbb{P}}
\newcommand{\1}{\mathds{1}}
\newcommand{\Pois}[1]{\text{Pois}(#1)}
\newcommand{\Var}[1]{\text{$\mathbb{V}\!\mathrm{ar}$}\left[#1\right]}
\DeclareMathOperator{\HPP}{HPP}
\DeclareMathOperator{\CPP}{CPP}

% Barva linka tritev iz poglavja Priloga
\colorlet{linkequation}{blue}
\newcommand*{\refPriloga}[1]{%
  \begingroup
    \hypersetup{
      linkcolor=red,
      linkbordercolor=red,
    }%
    \ref{#1}%
  \endgroup
}

%------------------------------------------NASLOVNE STRANI-----------------------------------------%

\begin{document}

\thispagestyle{empty}
\noindent{\large
UNIVERZA V LJUBLJANI\\[1mm]
FAKULTETA ZA MATEMATIKO IN FIZIKO\\[5mm]
\program\ -- 1.~stopnja}
\vfill

\begin{center}{\large
\imeavtorja\\[2mm]
{\bf \naslovdela}\\[10mm]
Delo diplomskega seminarja\\[1cm]
Mentor: \imementorja}
\end{center}
\vfill

\noindent{\large 
Ljubljana, \letnica}
\pagebreak   

\thispagestyle{empty}
\tableofcontents
\pagebreak

\thispagestyle{empty}
\begin{center}
{\bf \naslovdela}\\[3mm]
{\sc Povzetek}
\end{center}

%--------------------------------------POVZETEK V SLOVENSCINI--------------------------------------%

V prvem delu diplome najprej definiramo sestavljeno Poissonovo porazdelitev in izpeljemo obliko njenih rodovnih 
funkcij, obravnavamo njeno povezavo s splo"snimi porazdelitvami in izpeljemo Panjerjevo rekurzivno shemo. 
Nato definiramo sestavljeni Poissonov proces in poka"zemo nekaj osnovnih lastnosti kot je neodvisnost 
in stacionarnost prirastkov. Izpeljemo nekaj rezultatov, ki jih dobimo, ko 
sestavljeni Poissonov proces markiramo.
V drugem delu diplome obravnavamo aplikacijo sestavljenega Poissonovega procesa v Cramér--Lundbergovem modelu. Definiramo 
verjetnost propada in pre"zivetja ter slednjo izrazimo z defektno prenovitveno ena"cbo. Doka"zemo 
Lundbergovo neenakost in obravnavamo asimptoti"cno obna"sanje verjetnosti propada, ko zahtevke modeliramo z 
lahkorepimi in te"zkorepimi porazdelitvami. Obna"sanje verjetnosti propada na koncu prakti"cno prika"zemo 
z ve"ckratnim simuliranjem procesa tveganja.
%--------------------------------------------------------------------------------------------------%

\vfill
\begin{center}
{\bf Compound Poisson process and its application in finance}\\[3mm] 
{\sc Abstract}
\end{center}

%--------------------------------------POVZETEK V ANGLESCINI---------------------------------------%
In the first half of the diploma, we define the compound Poisson distribution and derive the form of its
generating functions. We discuss its connection with general distributions and derive the Panjer recursion scheme.
We then define the compound Poisson process and show some basic properties such as the independence and
stationarity of increments. We derive some results that follow from a space-time decomposition of the compound Poisson process.
In the second half of the diploma, we discuss the application of the compound Poisson process in the Cramér--Lundberg model. 
We define the probability of ruin and survival and express the latter as a defective renewal equation. 
We prove the Lundberg inequality and discuss the asymptotic behavior of the probability of ruin when 
claims are modeled with light-tailed and heavy-tailed distributions. We practically demonstrate the behavior 
of the probability of ruin by repeatedly simulating the risk process.

%-----------------------------------------------------------------------------------%

\vfill\noindent
{\bf Math. Subj. Class. (2020):} 60G07 60G20 60G51 \\[1mm]
{\bf Klju"cne besede:} slu"cajni proces, sestavljena Poissonova porazdelitev, Panjerjeva rekurzivna shema, 
sestavljeni Poissonov proces, markiranje,
 Cramér--Lundbergov model, Verjetnost propada, lahkorepa porazdelitev, te"zkorepa porazdelitev\\[1mm]
{\bf Keywords:} stochastic process, compound Poisson distribution, Panjer recursion scheme, 
compound Poisson process, space-time decomposition,
Cramér--Lundberg model, probability of ruin, light-tailed distribution, heavy-tailed distribution
\pagebreak


%--------------------------------------------ZAHVALA-----------------------------------------------%
\addtocontents{toc}{\protect\setcounter{tocdepth}{0}}
\section*{Zahvala}
\noindent
Zahvaljujem se doc.\ dr.\ Martinu Raiču za 
njegovo mentorstvo. S svojim znanjem in izkušnjami mi 
je pomagal ne le pri razumevanju snovi, predstavljene v delu, 
ampak tudi pri razumevanju naprednejših tem v teoriji slučajnih procesov, ki se 
v diplomi ne pojavijo. Hvaležen sem mu za njegovo izjemno pripravljenost, 
večkratne (dolge) diskusije, svetovanje pri dokazovanju izrekov, oblikovanju 
in natančnost pri pregledovanju besedila.

%\noindent
%Zahvaljujem se tudi svoji dru"zini za vso podporo med "studijem. 
%
\addtocontents{toc}{\protect\setcounter{tocdepth}{2}}
\pagebreak
%--------------------------------------------VSEBINA-----------------------------------------------%

\section{Uvod}
    Na razli"cnih podro"cjih financ so se v zadnjem stoletju razvile raznovrstne verjetnostne tehnike, kako 
    modelirati spreminjanje vrednosti finan"cnih instrumentov s "casom. V delu se osredoto"cimo na zavarovalni"stvo, 
    kjer "zelimo modelirati razli"cne produkte, ki jih zavarovalnice ponujajo, da lahko ustrezno dolo"cimo vi"sino premije, ki jo
    bodo zavarovanci pla"cevali. Zadostiti "zelimo dolo"cenim zastavljenim merilom tveganja in dobiti verjetnostna zagotovila, 
    da lahko skozi "cas pri"cakujemo dobi"cek. 
    Standarden pristop k problemu je uporaba sestavljenih porazdelitev oziroma procesov, kjer skozi "cas se"stevamo naklju"cno "stevilo 
    neodvisnih enako porazdeljenih slu"cajnih spremenljivk, ki predstavljajo vi"sino posameznih zahtevkov zavarovancev.\ Prvi tak model za opis
    poslovanja zavarovalnice 
    je v za"cetku 20.\ stoletja razvil Filip Lundberg. Le-ta je danes eden izmed najbolj raziskanih in 
    uporabljenih modelov v teoriji tveganja. Model temelji na sestavljenem Poissonovem procesu, kjer predpostavimo, da ima "stevilo zahtevkov,
    ki jih zavarovalnica prejme v nekem "casovnem intervalu, Poissonovo porazdelitev.

    \begin{figure}[H]
        \centering
        \includegraphics[width=\textwidth]{
            C:/Users/38651/OneDrive - Univerza v Ljubljani/Desktop/Diploma/Diplomski-seminar/GraphsAndPhotos/slika1.pdf
            }
        \caption{Primer trajektorije sestavljenega Poissonovega procesa z intenzivnostjo $\lambda = 0{,}1$ in
        eksponentno porazdeljenimi zahtevki; $X_i\sim\text{Exp}(20)$.}
        \label{fig:slika1}
    \end{figure}
    \noindent
    Sliko \ref{fig:slika1} v kontekstu zavarovalni"stva interpretiramo kot kumulativno obveznost zavarovalnice 
    svojim zavarovancem v nekem "casovnem intervalu. 

    Vsebina se v grobem deli na dva dela. Drugo in tretje poglavje sta namenjeni tehni"cni obravnavi klju"cnih lastnosti 
    sestavljene Poissonove porazdelitve in procesa. V "cetrtem poglavju pa se posvetimo prakti"cni obravnavi uporabe sestavljenega
    Poissonovega procesa v zavarovalni"stvu. 

    V drugem poglavju definiramo sestavljeno Poissonovo porazdelitev in izpeljemo obliko njenih rodovnih funkcij.
    Te"zava, ki si jo delijo sestavljene porazdelitve, je, da so redko analiti"cno izra"cunljive. 
    Zato izpeljemo Panjerjevo rekurzivno shemo, ki 
    je ena izmed najbolj popularnih metod za numeri"cno aproksimacijo sestavljene Poissonove porazdelitve.

    V tretjem poglavju definiramo sestavljeni Poissonov proces in izpeljemo nekaj osnovnih lastnosti, kot sta 
    neodvisnost in stacionarnost prirastkov ter pri"cakovana vrednost. Poglavje zaklju"cimo z izpeljavo rezultatov,
    ki jih dobimo, ko sestavljeni Poissonov proces markiramo.

    V "cetrtem poglavju se posvetimo Cramér--Lundbergovemu modelu, ki je prvi model, ki je bil uporabljen za
    modeliranje tveganja v zavarovalni"stvu. Osredoto"cimo se na verjetnost propada in zanjo izpeljemo prenovitveno
    ena"cbo. Obravnavamo asimptotiko verjetnosti propada, ko zahtevke modeliramo z lahkorepimi in te"zkorepimi
    porazdelitvami.

    V delu privzamemo, da je bralec seznanjen z osnovami verjetnosti, homogenega Poissonovega procesa in prenovitvenimi procesi.
    Na slike, vire in rezultate, ki jih izpeljemo v delu, se sklicujemo z modro barvo \raisebox{0.6ex}{\fcolorbox{black}{propercyan}{\rule{0pt}{1pt}\rule{1pt}{0pt}}}.
    Na koncu dela je priloga, kjer je zbrana velika ve"cina definicij in izrekov, s katerimi naj bi bil bralec seznanjen pred branjem dela. Na njih 
    se skli"cemo, ko jih prvi"c omenimo v delu in pri dokazih trditev, ter jih ozna"cimo z rde"co barvo \raisebox{0.6ex}{\fcolorbox{black}{red}{\rule{0pt}{1pt}\rule{1pt}{0pt}}}.

    \newpage
\section{Sestavljena Poissonova porazdelitev}

    \noindent
    Razdelek je prirejen po \cite{1}, \cite{2} in  \cite{4}.

    Sestavljena Poissonova porazdelitev je osnovni gradnik za sestavljeni Poissonov proces, ki ga obravnavamo 
    v naslednjem razdelku. Lastnosti, ki jih doka"zemo, so direktno prenosljive na sam proces. Izpeljemo 
    obliko rodovnih funkcij, porazdelitveno funkcijo, zanimive rezultate v povezavi s 
    splo"snimi slu"cajnimi spremenljivkami in Panjerjevo rekurzivno shemo, ki jo prika"zemo na 
    prakti"cnem zgledu.

    \begin{definicija}
        Naj bo $N\sim \Pois{\lambda}$  za $\lambda >0$ in $X_1, X_2, \dots$ zaporedje neodvisnih (med seboj in od $N$)
        enako porazdeljenih slučajnih spremenljivk. Potem pravimo, da ima slu"cajna spremenljivka
        \begin{equation*}
            S = \sum_{i=1}^NX_i
        \end{equation*}
        \textit{sestavljeno Poissonovo porazdelitev}. 
        \label{def:sestavljenaPoissonovaPorazdelitev}
    \end{definicija}

    \begin{opomba}
        V primeru, ko je $X_i = 1$ za vsak $i\in\N$, je $S\sim\Pois{\lambda}$.
    \end{opomba}

    \begin{opomba}
        Na enak na"cin definiramo splo"sne sestavljene porazdelitve, kjer je $N$ poljubna slu"cajna spremenljivka,
        ki zavzema vrednosti v $\N_0$\footnote{V delu se dr"zimo dogovora, da je $S$ na dogodku $\{N=0\}$ prazna vsota in zato 
        enaka $0$.}. Konkreten primer nas zanima zaradi njegove povezave s sestavljenim
        Poissonovim procesom. V nadaljevanju bomo uporabljali oznako
        \begin{equation*}
            S_0 = 0 \quad \text{in} \quad S_k = \sum_{i=1}^kX_i \quad \text{za} \ k\in\N\>; 
        \end{equation*}
        opazimo, da se brezpogojna porazdelitev slu"cajne spremenljivke $S_k$ ujema s pogojno porazdelitvijo 
        slu"cajne spremenljivke $S\mid\{N = k\}$.
        \label{op:gneralCaseCOmpound}
    \end{opomba}

    \subsection{Rodovne, momentno-rodovne in karakteristi"cne funkcije} Klju"cno \newline orodje pri dokazovanju
    lastnosti slu"cajnih spremenljivk so rodovne funkcije \refPriloga{def:rodovneFunkcije}, saj so zelo uporabne
    pri obravnavi vsot neodvisnih slu"cajnih spremenljivk in ("ce obstajajo) popolnoma dolo"cajo njihovo porazdelitev.  

    \begin{trditev}
        Naj bo $N\sim \Pois{\lambda}$  za $\lambda >0$ in $X_1, X_2, \dots$ zaporedje neodvisnih (med seboj in od $N$)
        enako porazdeljenih slučajnih spremenljivk z momentno-rodovno funkcijo $M_{X_1}$. Potem ima 
        momentno-rodovna funkcija vsote $S = \sum_{i=1}^NX_i$ obliko
        \begin{equation*}
            M_{S}(u) = e^{\lambda \left(M_{X_1}(u) - 1\right)}.
        \end{equation*}
        \label{trd:MomentGener}
    \end{trditev}
    
    \begin{proof}
        Velja
        \begin{align}
            M_{S}(u) 
                    &= \E\left[\exp\left[uS\right]\right] \nonumber\\
                    &= \sum_{k=0}^{\infty}
                        \E\left[\exp\left[uS\right] \ \big| \ N=k\right]\Prob\left(N = k\right) \nonumber \\ 
                    &= \sum_{k=0}^{\infty}
                        \E\left[\exp\left[uS_k\right]\right]\Prob\left(N = k\right) \nonumber \\
                    &= \sum_{k=0}^{\infty}
                        \underbrace{\E\left[e^{uX_1}\right]^k}_{M_{X_1}(u)^k}\frac{\lambda^k}{k!}e^{-\lambda } \label{eq:MomentS}\\ 
                    &= e^{-\lambda }\sum_{k=0}^\infty\frac{\left(M_{X_1}(u)\lambda \right)^k}{k!} \nonumber \\
                    &= e^{\lambda \left(M_{X_1}(u) - 1\right)}. \nonumber
        \end{align}
    \end{proof}
    
    \begin{posledica}
        Rodovna in karakteristi"cna funkcija slu"cajne vsote $S=\sum_{i = 1}^{N}X_i$ imata obliko 
        \begin{equation*}
            G_{S}(u) = e^{\lambda \left(G_{X_1}(u) - 1\right)} \qquad \text{in} \qquad
            \varphi_{S}(u) = e^{\lambda \left(\varphi_{X_1}(u) - 1\right)}.
        \end{equation*}
        Za obstoj rodovne funkcije $G_{S}$ potrebujemo dodatno predpostavko, 
        da so slu"cajne spremenljivke $X_i$ diskretne.
        \label{pos:RodovnaKarakteristicna}
    \end{posledica}

    \begin{proof}
    V splo"snem velja, da je karakteristi"cna funkcija neke slu"cajne spremenljivke $X$ v to"cki $u\in\R$ enaka
    njeni momentno-rodovni funkciji, izvrednoteni v $iu$, torej $\varphi_X(u) = M_X(iu)$. Rodovna funkcija
    v to"cki $u>0$ pa je enaka momentno-rodovni funkciji, izvrednoteni v $\ln(u)$, torej 
    $G_X(u) = M_X\left(\ln(u)\right)$, "ce slednja vrednost obstaja. 
    \end{proof}



    V nadaljevanju bomo uporabljali predvsem karakteristi"cno funkcijo $\varphi_S$, saj je ta vedno definirana 
    za vsak $u\in\R$. Prav nam bo pri"sla tudi naslednja povezava med $\varphi_S$ in $G_N$, ki je 
    poseben primer splo"snej"sega rezultata za sestavljene porazdelitve.

    \begin{trditev}
        Karakteristi"cno funkcijo $\varphi_S$ lahko izrazimo kot kompozitum rodovne funkcije $G_N$ in 
        karateristi"cne funkcije $\varphi_{X_1}$.

        \begin{align*}
            \varphi_{S}(u) = G_{N}\left(\varphi_{X_1}(u)\right).
        \end{align*}

        \label{lema:povezavaRodovneKarkateristicne}
    \end{trditev}

    \begin{proof}
        Po ena"cbi (\ref{eq:MomentS}) iz trditve \ref{trd:MomentGener} za $u\in\R$ velja
        \begin{align*}
            \varphi_{S}(u) &= \sum_{k=0}^{\infty}
            \varphi_{X_1}(u)^n\frac{\lambda^k}{k!}e^{-\lambda} \\
            &= G_{N}\left(\varphi_{X_1}(u)\right).
        \end{align*}
    \end{proof}

    \subsection{Porazdelitvena funkcija}
    Z uporabo izreka o popolni verjetnosti s pogojevanjem na $N$ pridemo do formule za 
    porazdelitveno funkcijo slu"cajne spremenljivke $S$. Za $x \in \R$ velja 
    \begin{align*}
        F_{S}(x) = \Prob\bigl(S \leq x\bigr) 
        &= \sum_{k=0}^\infty \Prob\bigl(S \leq x \mid N = k\bigr)\Prob\bigl(N = k\bigr) \\
        &= \sum_{k=0}^\infty \Prob\bigl(S_k \leq x\bigr)\Prob\bigl(N = k\bigr) \\
        &= \sum_{k=0}^\infty F_{X_1}^{*k}(x) \frac{\lambda^k}{k!} e^{-\lambda},
    \end{align*}

    \noindent
    kjer je $F_{X_1}^{*k}(x)$  $k$-ta konvolucija \refPriloga{def:konvolucija} funkcije $F_{X_1}$.
    
    \begin{zgled}
        Poglejmo enega enostavnej"sih primerov, ko so $X_1, X_2, \dots$ porazdeljene kot
        \begin{equation*}
            X_1\sim\text{Exp}(a),\quad \text{torej z gostoto} \quad  f_{X_1}(x) = a e^{-a x}\mathbbm{1}_{(0, \infty)}(x),
        \end{equation*}
        kjer je $a>0$. Vemo, da je $k$-ta 
        konvolucija porazdelitve slu"cajne spremenljivke $X_1$ porazdelitev
         $\text{Gamma}(k, a)$ in ima gostoto
        \begin{equation*}
            f_{X_1 + \cdots + X_k}(x) = \frac{1}{\Gamma(k)}a^kx^{k-1}e^{-a x}\mathbbm{1}_{(0, \infty)}(x).
        \end{equation*}
        Za $s>0$ velja
        \begin{align*}
            F_{S}(s) 
            &= \sum_{k=0}^\infty \int_0^s\frac{1}{\Gamma(k)}a^kx^{k-1}e^{-ax}dx \ \frac{(\lambda )^k}{k!}e^{-\lambda }
            \qquad \qquad \text{Tonelli} \ \refPriloga{izr:TonellijevIzrek} \\
            &= \int_0^s\underbrace{\sum_{k=0}^\infty \frac{1}{(k-1)!k!}(a\lambda)^kx^{k-1}e^{-(ax + \lambda)}}_{f_S(x)}dx.
        \end{align*}
        Vidimo, da lahko "ze celo v primeru, ko poznamo eksplicitno formulo za $F^{*k}_{X_1}$, te"zko pridemo do 
        porazdelitve slu"cajne spremenljivke $S$ v zaklju"ceni obliki. V praksi se zato poslu"zujemo numeri"cnega 
        ocenjevanja.
        \label{zgd:sestavljenaPoissonovaPorazdelitevGamma}
    \end{zgled}

    Vemo, da za neodvisne slu"cajne spremenljivke $X_1,  \dots, X_n$, ki so porazdeljene 
    kot $X_1\sim\Pois{\lambda_1}, \ \dots, \ X_n\sim\Pois{\lambda_n}$, 
    velja, da je njihova vsota $S = \sum_{i=1}^nX_i$ porazdeljena kot $S\sim\Pois{\lambda}$, kjer je
    $\lambda = \sum_{i=1}^n\lambda_i$.
    Izka"ze se, da ima sestavljena Poissonova porazdelitev 
    podobno lastnost.

    \begin{definicija}
        Naj bo $(\lambda_k)_{k=1}^n$ zaporedje pozitivnih realnih "stevil, za katerega velja 
        $\sum_{k=1}^n\lambda_k = 1$. Naj bodo $F_1, \dots, F_n$ porazdelitvene funkcije
        slu"cajnih spremenljivk $X_1, \dots, X_n$. Potem porazdelitvi s porazdelitveno 
        funkcijo 
        \begin{equation*}
            F = \sum_{k=1}^n\lambda_kF_k
        \end{equation*}
        pravimo \textit{me"sanica porazdelitev} teh slu"cajnih spremenljivk z ute"zmi $(\lambda_k)_{k=1}^n$.
    \end{definicija}
    \noindent
    Poka"zimo, da je $F$ res porazdelitvena funkcija. "Ce definiramo slu"cajno spremenljivko 
    $$
    I \sim 
    \begin{pmatrix}
        1 & 2  & \dots & n\\
        \lambda_1 & \lambda_2  & \dots & \lambda_n
    \end{pmatrix}
    $$
    in je le-ta neodvisna od vsake od slu"cajnih spremenljivk $X_1, \dots, X_n$, vidimo, da je $F$ 
    porazdelitev slu"cajne spremenljivke $X_I = \mathbbm{1}_{\{I = 1\}}X_1 + \cdots + \mathbbm{1}_{\{I = n\}}X_n$, 
    kar enostavno poka"zemo z uporabo zakona o popolni verjetnosti. Za poljuben $n\in\N$ in $x\in\R$ velja 
    \begin{align*}
        \Prob\left(X_I \leq x\right) 
        &= \sum_{k=1}^n\Prob\left(X_I \leq x \ \big| \ I = k\right)\Prob\left(I = k\right) \\
        &= \sum_{k=1}^n\Prob\left(X_k \leq x\right)\lambda_k \\
        &= \sum_{k=1}^nF_k(x)\lambda_k.
    \end{align*}
    Z enakim argumentom lahko poka"zemo, da je $\varphi_{X_I}(u) = \sum_{k=1}^n\lambda_k\varphi_{X_k}(u)$.


    \begin{trditev}
        Naj imajo neodvisne slu"cajne spremenljivke $S^{(1)}, \dots, S^{(n)}$ sestavljeno Poissonovo porazdelitev, torej 
        \begin{equation*}
            S^{(k)} = \sum_{i=1}^{N_k}X_i^{(k)} \quad \text{za} \ k=1, \dots, n,
        \end{equation*}
        kjer je $N_k\sim \Pois{\lambda_k}$ za $\lambda_k > 0$ in kjer je $(X_i^{(k)})_{i\in\N}$ za vsak 
        $k=1, \dots, n$
        zaporedje enako porazdeljenih slu"cajnih spremenljivk, ki so neodvisne tako med seboj
        kot tudi od $N_k$. Potem velja   
        \begin{equation*}
            S = \sum_{k=1}^nS^{(k)} \sim \sum_{i=1}^{N}Y_i,
        \end{equation*}
        kjer je $N\sim\Pois{\lambda}$ s parametrom $\lambda = \sum_{k=1}^n\lambda_k$ in $(Y_i)_{i\in\N}$ zaporedje
        enako porazdeljenih slu"cajnih spremenljivk z me"sano porazdelitveno funkcijo
        \begin{equation*}
        F_{Y_1} = \sum_{k=1}^n\frac{\lambda_k}{\lambda}F_{X_1^{(k)}},
        \end{equation*}
        ki so neodvisne tako med seboj kot tudi od $N$.
        \label{trd:vsotaCPDjeCPD}
    \end{trditev}

    \begin{proof}
        Karakteristi"cna funkcija slu"cajne spremenljivke $S^{(k)}$ ima obliko
        \begin{equation*}
            \varphi_{S^{(k)}}(u) = \exp\left[\lambda_k\left(\varphi_{X_1^{(k)}}(u) - 1\right)\right], \quad k=1, \dots, n.
        \end{equation*}
        Ker so $S^{(1)}, \dots, S^{(n)}$ neodvisne, velja
        \begin{align*}
            \varphi_{S}(u) 
                &= \prod_{k=1}^n\varphi_{S^{(k)}}(u) \\
                &= \prod_{k=1}^n\exp\left[\lambda_k\left(\varphi_{X_1^{(k)}}(u) - 1\right)\right] \\
                &= \exp\Biggl[\lambda\Biggl(\underbrace{\sum_{k=1}^n \frac{\lambda_k}{\lambda} \varphi_{X_1^{(k)}}(u)}_{\varphi_{Y_1}(u)} - 1\Biggr)\Biggr].
        \end{align*}
        Po izreku o 
        enoli"cnosti \refPriloga{izr:enolicnost} sledi $S\sim\sum_{i=1}^{N}Y_i$.
    \end{proof}

    Na podoben na"cin poka"zemo, kako se sestavljena Poissonova porazdelitev izra"za v primeru, ko so
    slu"cajne spremenljivke $X_i$ diskretno porazdeljene.

    \begin{trditev}
        Naj bo $N\sim \Pois{\lambda}$  za $\lambda >0$ in $X_1, X_2, \dots X_n$ neodvisne s.s. (neodvisne 
        med sabo in od $N$) enako porazdeljene po shemi
        $$ 
        \begin{pmatrix}
            a_1 & a_2 & a_3 &  \dots  \\
            \tfrac{\lambda_1}{\lambda} & \tfrac{\lambda_2}{\lambda} & \tfrac{\lambda_3}{\lambda} & \dots 
        \end{pmatrix},
        $$
        kjer je $(a_n)_{n\in\N}$ poljubno zaporedje realnih "stevil in 
        $(\lambda_n)_{n\in\N}$ zaporedje pozitivnih realnih "stevil, za katerega velja 
        ${\sum_{i=1}^\infty\lambda_i = \lambda}$.
        Potem velja 
        \begin{equation*}
            \sum_{j=1}^\infty a_jY_j \sim \sum_{j=1}^NX_j,
        \end{equation*}
        kjer so $Y_1,Y_2,  \dots$ neodvisne slu"cajne spremenljivke s porazdelitvami \\
        $\Pois{\lambda_1},\Pois{\lambda_2}, \dots$
        \label{trd:NXjeEnakoaY}
    \end{trditev}

    \begin{proof}
        S $\varphi_{Z_n}$ ozna"cimo karakteristi"cno funkcijo slu"cajne spremenljivke 
        $Z_n := a_1Y_1 + a_2Y_2 + \dots + a_nY_n$ in s $\varphi_{Z}$ karakteristi"cno funkcijo
        slu"cajne vsote $Z:= \sum_{j=1}^{N}X_j$. Po neodvisnosti velja
        \begin{align*}
            \varphi_{Z_n}(u) 
                    &= \prod_{j=1}^{n}\varphi_{Y_j}(a_ju)\\
                    &= \prod_{j=1}^{n}\exp\left[\lambda_j\left(e^{a_j i u} - 1\right)\right] \\
                    &= \exp\left[\sum_{j=1}^{n}\lambda_j\left(e^{a_j i u} - 1\right)\right].
        \end{align*}

        \noindent
        Po posledici  \ref{pos:RodovnaKarakteristicna} velja
        \begin{align*}
            \varphi_{Z}(u) 
                    &= \exp\left[\lambda\left(\varphi_{X_1}(u) - 1\right)\right] \\
                    & = \exp\left[\lambda\left(\sum_{j=1}^\infty\frac{\lambda_j}{\lambda}e^{a_jiu} - 1\right)\right]\\
                    &= \exp\left[\sum_{j=1}^{\infty}\lambda_j\left(e^{a_j i u} - 1\right)\right].
        \end{align*}

        \noindent 
        Po izreku o dominirani konvergenci \refPriloga{izr:dominiranaKonvergenca} zaporedje $\varphi_{Z_n}(u)$ za vsak $u\in\R$ po to"ckah konvergira k $\varphi_{Z}(u)$, torej
        \begin{equation*}
            \varphi_{Z_n} \xrightarrow{n\to\infty}\varphi_Z
        \end{equation*}
        in po Lévijevem izreku o kontinuiteti \refPriloga{izr:LevijevIzrek} velja $Z_n \xrightarrow[n\to\infty]{d}Z$.
    \end{proof}

    Rezultat je zanimiv predvsem zato, ker nam v nasprotju s trditvijo \ref{trd:vsotaCPDjeCPD} pove, da 
    lahko slu"cajno vsoto izrazimo kot linearno kombinacijo oziroma vrsto Poissonovih slu"cajnih spremenljivk.

    %\begin{opomba}
    %Kaj pa v primeru, ko $X_i$ niso diskretno porazdeljene? Ali lahko
    %trditev \ref{trd:NXjeEnakoaY} posplo"simo?
    %Izka"ze se, da tudi v splo"snem dobimo konvergenco v porazdelitvi (nisem prepri"can "ce je to res). 
    %Naj bo $F(x)$ porazdelitvena funkcija realno"stevilske slu"cajne spremenljivke $X_1$.
    %Ideja je, da definiramo funkcijo $F_n(x) := F(\tfrac{k + 1}{n})$ na intervalu 
    %$\bigl[\frac{k}{n}, \frac{k + 1}{n}\bigr)$ za $k\in\mathbb{Z}$.
    %    \begin{figure}[H]
    %        \begin{center}
    %        
    %            \begin{tikzpicture}
    %                % coordinate system
    %                \draw[->] (-0.75,0) -- (9,0) node[right] {$x$};
    %                \draw[->] (2,0) -- (2,4.5) node[above] {$F, F_n$};
    %                \draw (2, 3.4) -- (2, 3.4) node[left] {$1$};
    %                \draw[dashed] (-0.75,3.2) -- (9,3.2);
    %            
    %                % CDF of continuous random variable
    %                \draw[cyan] (-0.75, 0.1) .. controls (0,0.2) and (2.2, 0.3) .. (2.8, 1.2);
    %                \draw[->, cyan] (2.8, 1.2) .. controls (3.1, 1.7) and (3.6, 2) .. (5, 2.1);
    %                \filldraw[cyan] (5, 2.5) circle (0.7pt);
    %                \draw[cyan] (5, 2.5) .. controls (6, 2.8) and (7.5, 3) .. (9, 3.05) node[below] {$F(x)$};    
    %            
    %                % CDF of F_n,   
    %                \draw[->, red] (-0.75, 0.22) -- (0.25, 0.22);
    %                \filldraw[red] (-0.75, 0.22) circle (0.7pt);
    %                \draw[->, red] (0.25, 0.39) -- (1.25, 0.39);
    %                \filldraw[red] (0.25, 0.39) circle (0.7pt);
    %                \draw[->, red] (1.25, 0.74) -- (2.25, 0.74);
    %                \filldraw[red] (1.25, 0.74) circle (0.7pt);
    %                \draw[->, red] (2.25, 1.68) -- (3.25, 1.68);
    %                \filldraw[red] (2.25, 1.68) circle (0.7pt);
    %                \draw[->, red] (3.25, 2.01) -- (4.25, 2.01) node[above left]{$F_n(x)$};
    %                \filldraw[red] (3.25, 2.01) circle (0.7pt);
    %                \draw[->, red] (4.25, 2.58) -- (5.25, 2.58);
    %                \filldraw[red] (4.25, 2.58) circle (0.7pt);
    %                \draw[->, red] (5.25, 2.79) -- (6.25, 2.79);
    %                \filldraw[red] (5.25, 2.79) circle (0.7pt);
    %                \draw[->, red] (6.25, 2.94) -- (7.25, 2.94);
    %                \filldraw[red] (6.25, 2.94) circle (0.7pt);
    %                \draw[->, red] (7.25, 3.03) -- (8.25, 3.03);
    %                \filldraw[red] (7.25, 3.03) circle (0.7pt);
    %            
    %            
    %                %intervals of F_n
    %            
    %            \end{tikzpicture}
    %            \caption{Aproksimacija $F$ s $F_n$}
    %            \label{fig:slika2}
    %        \end{center}
    %    \end{figure}
%
    %\noindent
    %Kot je razvidno iz slike \ref{fig:slika2}, je $F_n(x)$ stopni"casta funkcija, ki aproksimira 
    %porazdelitveno funkcijo $F(x)$. Vemo, da $F_n$ ustreza diskretni porazdelitvi
%
    %$$ 
    %\begin{pmatrix}
    %    & \dots & \frac{k}{n} & \frac{k + 1}{n} & \frac{k+2}{n} & \dots & \\
    %    & \dots & F(\frac{k}{n}) - F(\frac{k-1}{n}) & F(\frac{k+1}{n}) - F(\frac{k}{n}) & F(\frac{k+2}{n}) - F(\frac{k+1}{n}) & \dots & 
    %\end{pmatrix}.
    %$$
    %Izka"ze se, da $F_n \xrightarrow{n\to\infty}F$ povsod kjer je $F$ zvezna, ampak
    %dokaz presega obseg tega dela. Interesirani bralec ga lahko najde v \cite{4} (moram dobiti dejansko referenco).
    %\label{op:aproksimacijaZDiskretno}
    %\end{opomba}





    \subsection{Panjerjeva rekurzivna shema}
    Poglejmo si popularno metodo za numeri"cno aproksimacijo sestavljene Poissonove porazdelitve v praksi. Kot
    smo videli v zgledu \ref{zgd:sestavljenaPoissonovaPorazdelitevGamma}, je izra"zava eksplicitne 
    porazdelitvene funkcije $S$ v zaklju"ceni obliki v splo"snem nemogo"ca. Izka"ze pa se, da jo je v posebnih primerih vselej mogo"ce
    rekurzivno izraziti in ustrezno posplo"siti na "sir"si razred porazdelitev. 

    \begin{trditev}(Panjer)
        Naj bo $N$ diskretna slu"cajna spremenljivka z vrednostmi v $\N_0$, za katero velja 
        \begin{equation*}
            \Prob(N = n) = \left(a + \frac{b}{n}\right)\Prob\left(N = n-1\right) \quad \text{za} \ n\in\N \ \text{in} \ a, b \in \R.
        \end{equation*}
        Naj bo $X_1, X_2, \dots$ zaporedje neodvisnih in enako porazdeljenih slu"cajnih spremenljivk, ki 
        zavzemajo vrednosti v $\N_0$. Potem za $S = \sum_{i=1}^NX_i$ ob dogovoru, da je $0^0:=1$, velja
        \begin{equation*}
        \Prob\left(S = 0\right) =  \E\left[\Prob\left(X_1 = 0\right)^N\right],
        \end{equation*}
        za $n\in\N$ pa velja
        \begin{equation}
            \Prob(S = n) = \frac{1}{1 - a\Prob(X_1 = 0)}\sum_{k = 1}^n\left(a + \frac{bk}{n}\right)\Prob(X_1 = k)\Prob(S = n - k).
            \label{eq:PanjerRecursionScheme}
        \end{equation}
        \label{tdr:PanjerRecursionScheme}
    \end{trditev}

    \begin{proof}
        Po zakonu za popolno pri"cakovano vrednost velja
        \begin{align*}
            \Prob(S = 0) 
                &= \sum_{j = 0}^\infty\Prob(S = 0\mid N = j)\Prob(N = j) \\
                &= \sum_{j = 0}^\infty\Prob(S_j = 0)\Prob(N = j) \\
                &= \sum_{j = 0}^\infty\Prob(X_1 = 0)^j\,\Prob(N = j) \\
                &= \E\left[\Prob(X_1 = 0)^N\right].
        \end{align*}
        Za $n\in\N$ pa velja
        \begin{align}
            \Prob(S = n) 
                &= \sum_{j = 1}^\infty\Prob(S_j = n)\Prob(N = j) \nonumber \\
                &= \sum_{j = 1}^\infty\Prob(S_j = n)\left(a + \frac{b}{j}\right)\Prob(N = j - 1). \label{eq:PanjerRecursionSchemeProof}
        \end{align}
        Ker so $X_1, X_2, \dots$ neodvisne in enako porazdeljene, so izmenljive (definicija \refPriloga{def:Izmenljivost}) in tako po 
        trditvi \refPriloga{trd:izmenljivostSimetricnaFunkcija} velja
        \begin{equation*}
            1 = \E\left[\frac{S_j}{S_j} \biggm|  S_j\right] = \sum_{k = 1}^j\E\left[\frac{X_k}{S_j} \biggm|  S_j\right] = j\,\E\left[\frac{X_1}{S_j} \biggm|  S_j\right],
        \end{equation*}
        torej je 
        \begin{equation*}
            \E\left[\frac{X_1}{S_j}  \biggm|  S_j\right] = \frac{1}{j}
        \end{equation*}
        in posledi"cno 
        \begin{equation}
            \E\left[a + \frac{bX_1}{n} \biggm|   S_j = n\right] = a + \frac{b}{j}.
            \label{eq:PanjerRecursionSchemeProof2}
        \end{equation}         
        Nadalje velja  
        \begin{align}   
            & \E\left[a + \frac{bX_1}{n} \biggm|   S_j = n\right] \nonumber \\
            &= \sum_{k = 0}^n\left(a + \frac{bk}{n}\right)\Prob(X_1 = k\mid S_j = n) \nonumber \\
            &= \sum_{k = 0}^n\left(a + \frac{bk}{n}\right)\frac{\Prob(X_1 = k,\, S_j - X_1 = n - k)}{\Prob(S_j = n)} \nonumber \\
            &= \sum_{k = 0}^n\left(a + \frac{bk}{n}\right)\frac{\Prob(X_1 = k)\,\Prob(S_{j- 1} = n - k)}{\Prob(S_j = n)} \label{eq:PanjerRecursionSchemeProof3} 
        \end{align}

        \pagebreak
        \noindent
        "Ce sedaj vstavimo enakost (\ref{eq:PanjerRecursionSchemeProof2}) v (\ref{eq:PanjerRecursionSchemeProof}) 
        in upo"stevamo (\ref{eq:PanjerRecursionSchemeProof3}), dobimo
        \begin{align*}
            \Prob(S = n)
                = \sum_{j = 1}^\infty\sum_{k = 0}^n \left(a + \frac{bk}{n}\right)\Prob(X_1 = k)\Prob(S_{j - 1} = n - k)\Prob(N = j - 1).
        \end{align*}
        Po Tonellijevem izreku \refPriloga{izr:TonellijevIzrek} lahko zamenjamo vrstni red vsot, kar nam da
        \begin{align*}
            \Prob(S = n) 
                &= \sum_{k = 0}^n\left(a + \frac{bk}{n}\right)\Prob(X_1 = k)\sum_{j = 1}^\infty\Prob(S_{j - 1} = n - k)\Prob(N = j - 1)\\
                &= \sum_{k = 0}^n\left(a + \frac{bk}{n}\right)\Prob(X_1 = k)\Prob(S = n - k).
        \end{align*}
        Izpostavimo prvi "clen vsote in izraz preoblikujemo.
        \begin{align*}
            \Prob(S = n) 
                &= a\Prob(X_1 = 0)\Prob(S = n) + \sum_{k = 1}^n\left(a + \frac{bk}{n}\right)\Prob(X_1 = k)\,\Prob(S = n - k), \\
            \Prob(S = n)
                &= \frac{1}{1 - a\Prob(X_1 = 0)}\sum_{k = 1}^n\left(a + \frac{bk}{n}\right)\Prob(X_1 = k)\,\Prob(S = n - k).
        \end{align*}
        S tem je trditev dokazana.
    \end{proof}

    \begin{opomba}
        Izka"ze se, da le tri porazdelitve 
        ustrezajo rekurzivni izra"zavi iz trditve \ref{tdr:PanjerRecursionScheme}. Te so $\Pois{\lambda}$, $\text{Bin}(p)$ in
        $\text{NegBin}(r, p)$ ob dogovoru, da ima slednja zalogo vrednosti $\N_0$ in $r\geq0$. Pravimo jim \textit{porazdelitve Panjerjevega razreda}. 
        V primeru $N\sim\Pois{\lambda}$ za $n\in\N_0$, $a = 0$ in $b = \lambda$ velja $\Prob\left(N = n\right) = \frac{\lambda^n}{n!}e^{-\lambda} = 
        \left(0 + \frac{\lambda}{n}\right)\Prob\left(N = n - 1\right)$. 
        Tudi v ostalih primerih (argument je podan v \cite{4} na strani 122) se izka"ze, da je $a < 1$, tako da je 
        ena"cba (\ref{eq:PanjerRecursionScheme}) res dobro definirana.
        \label{op:PanjerRazsiritev}
    \end{opomba}

    \begin{opomba}
        Zahtevo, da $X_i$ zavzemajo vrednosti v $\N_0$, se spla"ca posplo"siti, tako da 
        zahtevamo le, da $X_i$ zavzemajo vrednosti v $h\N_0$ za neki $h>0$. V tem primeru 
        zapi"semo $S = h\sum_{i=1}^N\frac{X_i}{h}$ in tako rekurzivna zveza velja za $\frac{S}{h}$. 
        Tako lahko aproksimiramo splo"sne slu"cajne 
        spremenljivke, ki zavzemajo vrednosti v $[0, \infty)$, poljubno natan"cno.
    \end{opomba}  

    \begin{zgled}(Nadaljevanje zgleda \ref{zgd:sestavljenaPoissonovaPorazdelitevGamma}) Recimo, da imamo
        konkretni porazdeltivi $N\sim\Pois{9{,}17}$ in $X_i\sim\text{Exp}(\frac{1}{\pi})$. S stopni"castima funkcijama
        $F^u_h$ in $F_h^l$ aproksimiramo porazdelitveno funkcijo $F_{X_1}$ za vrednosti $h \in \{1, \ 0{,}1\}$. 
        (Aproksimacije so na sliki \ref{fig:slika7} in so obarvane;  $F^u_{1}$ modra, $F^u_{0{,}1}$ 
        rde"ca, $F^l_{1}$ vijoli"cna in $F^l_{0{,}1}$ oran"zna).
        Za vsak $n\in\N$ velja $F^u_h(x) = F_{X_1}((n+1)h)$ za $x\in\bigl[nh, (n+1)h\bigr)$ in 
        $F^l_h(x) = F_{X_1}(nh)$ za $x\in\bigl[nh, (n+1)h\bigr)$.
        S Panjerjevo rekurzivno shemo izra"cunamo pribli"zke porazdelitvene funkcije slu"cajne 
        spremenljivke $S$ na intervalu $[0, 60]$.  
 
        \begin{figure}[h!]
            \begin{center}
                \includegraphics[width=\textwidth]{
                    C:/Users/38651/OneDrive - Univerza v Ljubljani/Desktop/Diploma/Diplomski-seminar/GraphsAndPhotos/slika7.pdf
                }
                \caption{Aproksimacija porazdelitve $S$ s Panjerjevo rekurzivno shemo.}
                \label{fig:slika7}
            \end{center}
        \end{figure}
        
    \pagebreak
    Vidimo, da "ze za $h = 0{,}1$ dobimo zelo natan"cno aproksimacijo porazdelitve. Danes 
    Panjerjeva metoda predstavlja alternativo Monte Carlo metodam. Njena glavna prednost 
    je, da z manj"sanjem koraka $h$ dose"zemo poljubno natan"cno to"cno aproksimacijo neke porazdelitve.\ 
    Monte Carlo metode so bolj splo"sne, saj temeljijo zgolj na ponavljanju simulacij in se lahko uporabljajo za modeliranje bolj zapletenih 
    porazdelitev, ki ne zado"s"cajo pogojem trditve \ref{tdr:PanjerRecursionScheme}, ali njene posplo"sitve
    v opombi \ref{op:PanjerRazsiritev}. 

    \label{zgd:PanjerExp}  
    \end{zgled}

    \newpage


\section{Sestavljeni Poissonov proces}

    \noindent
    Razdelek je prirejen po \cite{1}, \cite{2}, in \cite{3}.

    Sedaj se posvetimo "studiju sestavljenega Poissonovega procesa. Osnovne definicije in 
    lastnosti splo"snih slu"cajnih procesov, ki nas zanimajo, lahko bralec najde 
    med definicijama \refPriloga{def:slucProc} in \refPriloga{def:neodvisnostProcesov} v prilogi.
    Najprej doka"zemo in izpeljemo nekaj osnovnih lastnosti procesa kot so neodvisnot in stacionarnost
    prirastov, pri"cakovana vrednost in varianca. Na koncu obravnavamo markiranje procesa glede "cas in 
    njegovo vrednost ter doka"zemo nekaj zanimivih in uporabnih posledic.

    Sestavljeni Poissonov proces temelji na homogenenem Poissonovem procesu, zato najprej podamo definicijo 
    s katero bomo delali, saj ima homogeni Poissonov proces ve"c ekvivalentnih karakterizacij. 

    \begin{definicija}
        Naj bo $\lambda > 0$. Dru"zini slu"cajnih spremenljivk $(N_t)_{t\geq 0}$, definiranimi na verjetnostnem 
        prostoru $(\Omega, \mathcal{F}, \mathbb{P})$ z vrednostmi v $\N_0$, pravimo 
        \textit{homogeni Poissonov proces} z intenzivnostjo $\lambda$, če zadošča naslednjim pogojem:
        \begin{enumerate}
            \item $N_0 = 0$ \ $\Prob$-skoraj gotovo.
            \item $(N_t)_{t\geq 0}$ ima neodvisne in stacionarne prirastke,
            \item Za $0 \leq s < t$ velja $ N_t - N_s \sim\Pois{\lambda(t - s)}$,
        \end{enumerate}
        \label{def:HPP}
    \end{definicija}

    \begin{opomba}
    V delu bomo z $V_n$ ozna"cevali "cas $n$-tega prihoda v homogenem Poissonovem procesu in s $T_n$ $n$-ti 
    medprihodni "cas. Za medprihodne "case velja, da so neodivsni in enako porazdeljeni, kot 
    $T_1 \sim \text{Exp}(\lambda)$. Ta lastnost je tudi alternativna definicija procesa, kot poseben 
    primer prenovitvenega procesa \refPriloga{def:PrenovitveniProces}. Bralec 
    lahko najde dokaz ekvivalence v \cite{10} na strani 15.
    \end{opomba}

    \begin{definicija}
        Naj bo $(N_t)_{t\geq0}$ homogeni Poissonov proces z intenzivnostjo $\lambda$. 
        Naj bo $(X_i)_{i\geq1}$ zaporedje neodvisnih (med sabo in od procesa $(N_t)_{t\geq0}$) in enako 
        porazdeljenih slučajnih spremenljivk. Potem je 
        \textit{sestavljeni Poissonov proces} $(S_t)_{t\geq0}$ definiran kot dru"zina
        slu"cajnih spremenljivk
        $$
            S_t = \sum_{i=1}^{N_t} X_i, \quad t\geq0.
        $$
        \label{def:CPP}
    \end{definicija}

    \begin{opomba}
        Vidimo, da je sestavljeni Poissonov proces naravna posplo"sitev homogenega Poissonovega procesa, saj "ce za
        $X_i$ vzamemo konstantno funkcijo $X_i = 1$ za vsak $i$, dobimo le tega. Bolj v splo"snem, "ce je 
        $X_i = \alpha$ deterministi"cna funkcija, velja $S_t = \alpha N_t$.
        \label{op:CPPHPPPovezava}
    \end{opomba}

    V nadaljevanju bomo homogen Poissonov proces z intenzivnostjo $\lambda >0$ ozna"cevali s $\HPP(\lambda)$ 
    ali naborom slu"cajnih spremenljivk $(N_t)_{t\geq0}$ (angl. Homogeneous Poisson Process), 
    sestavljeni Poissonov proces pa s $\CPP$ ali naborom slu"cajnih spremenljivk $(S_t)_{t\geq0}$ 
    (angl. Compound Poisson Process), kjer prihodi sledijo implicitno podanemu $\HPP(\lambda)$. 

    \subsection{Osnovne lastnosti}
    
        Pri "studiranju slu"cajnih procesov nas najprej zanimajo neke osnovne lastnosti, s katerimi 
        je la"zje delati kot z ne"stevnim naborom slu"cajnih spremenljivk in s pomo"cjo katerih 
        lahko poka"zemo globlje rezultate o procesu. 

        %\begin{trditev}
        %    Naj bo $(X_t)_{t\geq0}$ slu"cajni proces na $(\Omega, \F, \mathbb{P})$ in naj velja $X_0 = 0$ s.g.
        %    Potem ima $(X_t)_{t\geq0}$
        %    neodvisne prirastke natanko tedaj, ko je za vsak nabor "stevil 
        %    $0 \leq t_1 < \ldots < t_n < t_{n+1} <\infty$ prirastek $X_{t_{n+1}} - X_{t_n}$ neodvisen od
        %    slu"cajnega vektorja $(X_{t_1}, \dots, X_{t_n})$.
        %    \label{trd:ekvivKarakterizacija}
        %\end{trditev}
%
        %\begin{proof}
        %    $(\Leftarrow):$ Recimo, da je za vsak $n\in\N$ in poljuben nabor "stevil $0 \leq t_1 < \ldots < t_n < t_{n+1} <\infty$ 
        %    prirastek $X_{t_{n+1}} - X_{t_n}$ neodvisen od slu"cajnega vektorja $(X_{t_1}, \dots, X_{t_n})$.
        %    Potem je $X_{t_{n+1}} - X_{t_n}$ neodvisen tudi od $h(X_{t_1}, \dots, X_{t_n})$ za poljubno merljivo funkcijo $h:\R^n\to \R^n$.
        %    O"citno je $(X_{t_1}, \dots, X_{t_n}) \mapsto (X_{t_2} - X_{t_1}, \dots, X_{t_{n}} - X_{t_{n-1}})$ taka funkcija,
        %    torej je $X_{t_{n+1}} - X_{t_n}$ neodvisen od $(X_{t_2} - X_{t_1}, \dots, X_{t_{n}} - X_{t_{n-1}})$. Ker 
        %    to velja za poljuben $n$ in poljubne $t_i$, sledi, da ima $(X_t)$ neodvisne prirastke. \newline
        %    $(\Rightarrow):$  Recimo, da ima $(X_t)_{t\geq0}$ neodvisne prirastke. Ker velja $X_0 = 0$ s.g., vemo, da 
        %    je za poljuben $n\in\N$ in poljuben nabor "stevil $0 \leq t_1 < \ldots < t_n < t_{n+1} <\infty$
        %    prirastek $X_{t_{n+1}} - X_{t_n}$ neodvisen od $(X_0, X_{t_1} - X_0, \dots, X_{t_n} - X_{t_{n-1}})$ in posledi"cno neodvisen od
        %    $h(X_0, X_{t_1} - X_0, \dots, X_{t_n} - X_{t_{n-1}})$ za poljubno merljivo funkcijo $h:\R^{n+1}\to \R^{n+1}$, ki 
        %    jo definiramo na slede"c na"cin:
        %    \begin{align*}
        %        h(x_0, x_1, \dots, x_n) &= (h_0, h_1, \dots, h_n),\\
        %                            h_0 &= x_0, \\
        %                            h_1 &= x_0 + x_1, \\
        %                            &\mathrel{\makebox[\widthof{=}]{\vdots}} \\
        %                            h_n &= x_0 + x_1 + \cdots + x_n.
        %    \end{align*}
        %    Tako velja $h(X_0, X_{t_1} - X_0, \dots, X_{t_n} - X_{t_{n-1}}) = (X_0, X_{t_1}, \dots, X_{t_n})$ in s tem je trditev dokazana.
        %\end{proof}

        \pagebreak
        \begin{trditev}
            Slu"cajni proces $(X_t)_{t\geq0}$ na $(\Omega, \F, \mathbb{P})$, za katerega je $X_0 = 0$, ima
            neodvisne in stacionarne prirastke natanko tedaj, ko za poljubna realna "stevila
            $0\leq t_1 < t_2 < \ldots < t_n < t_{n+1} < \infty$ velja 
            \begin{equation}
                X_{t_{n+1}} - X_{t_n}\mid X_{t_1} \dots, X_{t_n} \sim X_{t_{n+1} - t_n}.
                \label{eq:neodvisnostStacionarnostPrirastov}
            \end{equation}
            \label{trd:neodvisnostStacionarnostPrirastov}
        \end{trditev}

        \begin{proof}
            $(\Rightarrow):$ Recimo, da ima $(X_t)_{t\geq0}$ neodvisne in stacionarne prirastke. Ker je 
            $X_0$ konstanta, lahko pi"semo $X_0 = x_0$. Po predpostavki in definicijah \refPriloga{def:prirastek} in 
            \refPriloga{def:stacPrir} je 
            \begin{equation*}
                X_{t_{n+1}} - X_{t_n}\mid X_{t_1} - x_0 \dots, X_{t_n} - X_{t_{n-1}}\sim X_{t_n} - X_{t_{n-1}} \sim X_{t_{n+1} - t_n}.
            \end{equation*}
            Po trditvi \refPriloga{trd:pogojneLastnosti} je potem za poljubno Borelovo funkcijo $g:\R^{n}\to \R^{n}$
            tudi 
            \begin{equation*}
                X_{t_{n+1}} - X_{t_n}\mid g(X_{t_1} - x_0 \dots, X_{t_n} - X_{t_{n-1}}) \sim X_{t_{n+1} - t_n}.
            \end{equation*}
            Definiramo 
            \begin{align*}
                g(y_1, y_2, \dots, y_n) &= (g_1, g_2, \dots, g_n),\\
                                    g_1 &= y_1 + x_0, \\
                                    g_2 &= y_2 + y_1 + x_0, \\
                                    &\mathrel{\makebox[\widthof{=}]{\vdots}} \\
                                    g_n &= y_n + y_{n-1} + \cdots + x_0.
            \end{align*}
            Funkcija $g$ je o"citno merljiva, saj jo definiramo le s se"stevanjem. Ker velja
            $g(X_{t_1} - x_0, \dots, X_{t_n} - X_{t_{n-1}}) = (X_{t_1}, \dots, X_{t_n})$, 
            smo dokazali prvo smer implikacije. \newline
            \noindent
            $(\Leftarrow):$ Recimo, da za poljubna realna "stevila $0\leq t_1 < t_2 < \ldots < t_n < t_{n+1} < \infty$ velja
            (\ref{eq:neodvisnostStacionarnostPrirastov}). Zdaj definiramo 
            \begin{align*}
                h(y_1, y_2, \dots, y_n) &= (h_1, h_2, \dots, h_n),\\
                                    h_1 &= y_1 - x_0, \\
                                    h_2 &= y_2 - y_1, \\
                                    &\mathrel{\makebox[\widthof{=}]{\vdots}} \\
                                    h_n &= y_n - y_{n-1}. 
            \end{align*}  
            Po trditvi \refPriloga{trd:pogojneLastnosti} je potem $X_{t_{n+1}} - X_{t_n}$ neodvisna od
            $(X_{t_1} - X_0, \dots, X_{t_n} - X_{t_{n-1}})$ in enako porazdeljena kot $X_{t_{n+1} - t_n}$.
            Z indukcijo dobimo, da so potem tudi prirastki med seboj neodvisni in stacionarni.     
            
        \end{proof}

        \begin{trditev}
            $\CPP$ ima neodvisne in stacionarne prirastke.
            \label{trd:neodvPrirCPP}
        \end{trditev}

        \begin{proof}
            Po trditvi \ref{trd:neodvisnostStacionarnostPrirastov} je dovolj pokazati, da za poljubna 
            realna "stevila
            $0\leq t_1 < t_2 < \ldots< t_n < t_{n+1}$ velja 
            \begin{equation*}
                S_{t_{n+1}} - S_{t_n}\mid S_{t_1}, \dots, S_{t_n} \sim S_{t_{n+1} - t_n}.
            \end{equation*}
            Naj bodo $k_1, \dots, k_n \in \N_0$ in $k_1\leq \cdots \leq k_n$. Na dogodku 
            $\{N_{t_n} = k_n\}$ velja 
            \begin{equation*}
                S_{t_{n+1}} - S_{t_n} = \sum_{i = k_n + 1}^{k_n + N_{t_{n+1}} - N_{t_n}}X_i,
            \end{equation*}
            zato je 
            \begin{align*}
                &S_{t_{n+1}} - S_{t_n}\mid N_{t_1}= k_1, \dots, N_{t_n} = k_n, X_1, \dots, X_{k_n} \\
                &\sim \sum_{i = k_n + 1}^{k_n + N_{t_{n+1}} - N_{t_n}}X_i\Bigm| N_{t_1}= k_1, \dots, N_{t_n} = k_n, X_1, \dots, X_{k_n}.
            \end{align*}
            Ker pa so $X_1, \dots, X_{k_n}, X_{k_{n+1}}, X_{k_{n+2}}, \dots,
             N_{t_2} - N_{t_1}, \dots, N_{t_n} - N_{t_{n-1}}, N_{t_{n+1}} - N_{t_n}$, neodvisne, sledi, 
            da sta neodvisna tudi vektorja 
            $$(N_{t_1}, N_{t_2} - N_{t_1}, \dots, N_{t_n} - N_{t_{n-1}}, X_1, \dots, X_{k_n}) \quad \text{in} \quad
            (N_{t_{n+1}} - N_{t_n} X_{k_n + 1}, \dots),$$ z njima pa tudi vektor
            $(N_{t_1}, \dots, N_{t_n}, X_1, \dots, X_{k_n})$ in slu"cajna spremenljivka \newline
            $\sum_{i = k_n + 1}^{k_n + N_{t_{n+1}} - N_{t_n}}X_i$. Torej je
            \begin{equation*}
                S_{t_{n+1}} - S_{t_n}\mid N_{t_1} = k_1, \dots, N_{t_n} = k_n, X_1, \dots, X_{k_n} \sim
                \sum_{i = k_n + 1}^{k_n + N_{t_{n+1}} - N_{t_n}}X_i.
            \end{equation*}
            Vemo, da po stacionarnosti prirastkov $\HPP$ velja $N_{t_{n+1}} - N_{t_n} \sim N_{t_{n+1} - t_n}$. 
            Ker pa je zaporedje $X_{k_n + 1}, X_{k_n + 2}, \dots$ neodvisno od $N_{t_{n+1}} - N_{t_n}$,
            zaporedje $X_1, X_2, \dots$ neodvisno od $N_{t_{n+1} - t_n}$ ter ker sta zaporedji 
            $X_1, X_2, \dots$ in $X_{k_n + 1}, X_{k_n + 2}, \dots$ enako porazdeljeni, je tudi 
            \begin{equation*}
                N_{t_{n+1}} - N_{t_n}; X_{k_n + 1}, X_{k_n + 2}, \dots \sim N_{t_{n+1} - t_n}; X_1, X_2, \dots
            \end{equation*}
            in zato tudi 
            \begin{align*}
                &S_{t_{n+1}} - S_{t_n} \mid N_{t_1} = k_1, \dots, N_{t_n} = k_n, X_1, \dots, X_{k_n} \\
                &\sim \sum_{i = k_n + 1}^{k_n + N_{t_{n+1}} - N_{t_n}}X_i 
                \sim \sum_{i = 1}^{N_{t_{n+1} - t_n}}X_i = S_{t_{n+1} - t_n}.
            \end{align*}
            Potem pa je po trditvi \refPriloga{trd:pogojneLastnosti} tudi
            \begin{equation*}
                S_{t_{n+1}} - S_{t_n}\mid N_{t_1} = k_1, \dots, N_{t_n} = k_n, \sum_{i=1}^{k_1}X_i, \dots, \sum_{i=k_{n-1} + 1}^{k_n}X_i \sim S_{t_{n+1} - t_n}.
            \end{equation*}
            Ker pa na dogodku $\{N_{t_1} = k_1, \dots, N_{t_n} = k_n\}$ velja 
            \begin{equation*}
                \sum_{i=1}^{k_1}X_i = S_{t_1}, \dots, \sum_{i=k_{n-1} + 1}^{k_n}X_i = S_{t_n} - S_{t_{n-1}},
            \end{equation*}
            je kon"cno 
            \begin{equation*}
                S_{t_{n+1}} - S_{t_n}\mid N_{t_1} = k_1, \dots, N_{t_n} = k_n, S_{t_1}, \dots, S_{t_n} - S_{t_{n-1}} \sim S_{t_{n+1} - t_n}
            \end{equation*}
            oziroma 
            \begin{equation*}
                S_{t_{n+1}} - S_{t_n}\mid N_{t_1}, \dots, N_{t_n}, S_{t_1}, \dots, S_{t_n} - S_{t_{n-1}} \sim S_{t_{n+1} - t_n}
            \end{equation*}
            in spet po trditvi \refPriloga{trd:pogojneLastnosti} velja
            \begin{equation*}
                S_{t_{n+1}} - S_{t_n}\mid S_{t_1}, \dots, S_{t_n} - S_{t_{n-1}} \sim S_{t_{n+1} - t_n}.
            \end{equation*}
        \end{proof}

        %\begin{opomba}
        %    Podobno kot v opombi \ref{op:gneralCaseCOmpound} bomo od zdaj naprej za $t\geq0$  
        %    pogojno porazdelitev $S_t \mid \{N_t = k\}$ ozna"cevali s
        %    \begin{equation*}
        %        S_{t, 0} = 0 \quad \text{in} \quad S_{t, k} = \sum_{i=1}^kX_i \quad \text{za} \ k\in\N.
        %    \end{equation*}
        %\end{opomba}

        \begin{trditev}
            Naj bo $(S_t)_{t\geq 0}$ $\CPP$ in naj bosta $\mu = \E\left[X_i\right] < \infty$ 
            pri"cakovana vrednost in $\sigma^2= \Var{X_i} <\infty$ varianca
            slu"cajnih spremenljivk $X_i$. Potem sta za $t\geq0$ pri"cakovana vrednost in 
            varianca $S_t$ enaki 
            \begin{equation*}
                \E\left[S_t\right] = \mu\lambda t \qquad \text{in} \qquad \Var{S_t} = \lambda t\left(\sigma^2 + \mu^2\right).
            \end{equation*}
            \label{trd:PricVarCPP}
        \end{trditev}

        \begin{proof}

            Za $t\geq0$ in $k\in \N_0$ velja 
            \begin{align*}
            \E\left[S_t\mid N_t = k\right]  = k\mu \qquad \text{in} \qquad
            \Var{S_t\mid N_t = k} = k\sigma^2.
            \end{align*} 
            Po formuli za popolno pri"cakovano vrednost velja 
            $\E\left[S_t\right] = \E\left[\E\left[S_t\mid N_t\right]\right]$. Torej je
            \begin{align*}
                \E\left[S_t\right] = \E\left[\E\left[S_t\mid N_t\right]\right] = \E\left[\mu N_t\right] = \mu\lambda t.
            \end{align*}
            Prek formule $\Var{S_t} = \E\left[\Var{S_t\mid N_t}\right] + \Var{\E\left[S_t\mid N_t\right]}$ ra"cunamo 
            \begin{equation*}
                \E\left[\Var{S_t\mid N_t}\right] = \E\left[\Var{X_i}N_t\right] = \sigma^2\lambda t
            \end{equation*}
            in 
            \begin{equation*}
                \Var{\E\left[S_t\mid N_t\right]} = \Var{\E\left[X_i\right]N_t} = \mu^2\lambda t.
            \end{equation*}
            "Ce ena"cbi se"stejemo, dobimo $\Var{S_t} = \lambda t\left(\sigma^2 + \mu^2\right)$.
        \end{proof}
    
    \begin{zgled}
        Poglejmo si primer ko je zaporedje $(X_i)_{i\in\N}$ porazdeljeno kot $X_1\sim N(2, 42)$. 
        Tedaj za $t\geq 0 $ velja $\E\left[S_t\right] = 2t$ in 
        $\Var{S_t} = t(2^2 + 42^2) = 1768t$ ter $\sigma_{S_t} = \sqrt{1768t}$. Simuliramo 10 realizacij CPP do "casa $T=1000$, 
        ki jih prika"zemo na sliki \ref{fig:slika6} skupaj s funkcijami $t \mapsto \E\left[S_t\right]$ in $t \mapsto \E\left[S_t\right] \pm 3\sigma_{S_t}$. 
        \begin{figure}[H]
            \centering
            \includegraphics[width=\textwidth]{
                C:/Users/38651/OneDrive - Univerza v Ljubljani/Desktop/Diploma/Diplomski-seminar/GraphsAndPhotos/slika6.pdf
                }
            \caption{Trajektorije CPP s funkcijami $t \mapsto \E\left[S_t\right]$ in $t \mapsto \E\left[S_t\right] \pm 3\sigma_{S_t}$}
            \label{fig:slika6}
        \end{figure}

    \end{zgled}
    %Poka"zimo, da je $CPP$ v resnici porazdeljen, kot limita linearne 
    %kombinacije neodvisnih Poissonovih slu"cajnih spremenljivk. 
    
    %\noindent
    %"Ce sedaj po"sljemo $n \to \infty$, dobimo
    %\begin{align}
    %    \varphi_{Z}(u) := \lim_{n\to\infty}\varphi_{Z_n}(u) = e^{\sum_{j=1}^{\infty}\lambda_j\left(e^{a_j i u} - 1\right)}.
    %    \label{eq:karFunkcVrste}
    %\end{align}
%
    %\noindent
    %Kot smo izpeljali zgoraj je karakteristi"cna funkcija $CPP$ podana s predpisom
%
    %\begin{align*}
    %    \varphi_{S_t}(u) = e^{\lambda t\left(\varphi_X(u) - 1\right)}, 
    %\end{align*}
%
    %\noindent
    %kar lahko zapi"semo kot 
%
    %\begin{align*}
    %    \varphi_{S_t}(u) = e^{\lambda t\int_{\R}\left(e^{i u z} - 1\right) \mu(dz)},
    %\end{align*}
%
    %\noindent
    %kjer je $\mu := X*\Prob$ potisk mere naprej po s.s.\ $X$. Prav tako lahko (\ref{eq:karFunkcVrste}) zapi"semo
    %kot 
%
    %\begin{align*}
    %    \varphi_{Z}(u) = e^{\int_{\R}\left(e^{i u x} - 1\right)\nu(dx)},
    %\end{align*}
%
    %\noindent
    %Za neko ustrezno mero $\nu$. Vidimo, da ko po"sljemo $n\to \infty$, za ustrezen izbor 
    %$a_1, a_2, \dots$ in $\lambda_1, \lambda_2, \dots$ karakteristi"cna funkcija 
    %vrste $Z_n$ konvergira h karakteristi"cni funkciji $S_t$. Torej po Lévijevem izreku o zveznosti 
    %sledi, da je $S_t$ enako porazdeljena kot $Z = \sum_{i=1}^{\infty}\alpha_iY_i$.
%

%    \subsection{Martingali}
%
%        \begin{definicija}
%            Slu"cajni proces $X_t$ prilagojen glede na filtracijo (\ref{def:filtracija}) $(\F_t)_{t\geq0}$
%            martingal, "ce velja 
%            $$
%                \E\left[X_t\mid\F_s\right] = X_s
%            $$
%            za vsak $0\leq s \leq t$.
%            \label{def:martingal}
%        \end{definicija}
%
%        \begin{trditev}
%            Naj bo $(S_t)_{t\geq0}$ $CPP$ z intenzivnostjo $\lambda>0$ in naj bodo $X_i$ neodvisne
%            in enako porazdeljene slu"cajne spremenljivke z $\E\left[X_i\right] = \mu$ za vsak $i$.
%            Potem je $S_t$ martingal natanko tedaj, ko je $\mu = 0$.
%            \label{trd:CPPnimartingal}
%        \end{trditev}
%
%        \begin{proof}
%            Naj bo $0\leq s\leq t$. Potem velja
%            \begin{align*}
%                \E\left[S_t\mid\F_s\right] 
%                        &= \E\left[S_t - S_s + S_s\mid \F_s\right] \\
%                        &= \E\left[S_t - S_s\right] + \E\left[S_s\mid \F_s\right] \\
%                        &= \mu\lambda(t-s) + S_s
%            \end{align*}
%           Enakost $\mu\lambda(t-s) + S_s = S_s$ velja $\iff$ $\mu\lambda(t-s) = 0 \iff \mu = 0$.
%        \end{proof}
%
%        \begin{opomba}
%            Seveda, "ce velja $\mu \geq 0$, potem je $S_t$ submartingal, "ce pa $\mu \leq 0$, je
%            $S_t$ supermartingal.
%        \end{opomba}
%
%        \begin{posledica}
%            Za poljuben $\mu \in \R$ je proces 
%            $$
%                S_t - \mu\lambda t
%            $$
%            martingal. 
%            \label{trd:CPPpostanemartingal}
%        \end{posledica}
%
%        \begin{proof}
%            Naj bosta $0 \leq s < t$. Velja
%            \begin{align*}
%                \E\left[S_t - \mu\lambda t\mid\F_s\right] 
%                        &= \E\left[S_t - S_s\right] + S_s - \mu\lambda t\\
%                        &= \mu\lambda(t-s) + S_s - \mu\lambda t\\
%                        &= S_s - \mu\lambda s.
%            \end{align*}
%        \end{proof}
%
%    \subsection{"Casi ustavljanja in lastnosti Markova}
%        \begin{definicija}
%            Naj bo $(\Omega, \mathcal{F}, \mathbb{P}, (\mathcal{F}_t)_{t\geq0})$ filtriran verjetnostni 
%            prostor. Slu"cajna spremenljivka 
%            $$
%            \tau : \Omega \to [0, \infty]
%            $$ 
%            je \textit{"cas ustavljanja} glede na filtracijo $(\mathcal{F}_t)_{t\geq0}$, "ce je za vsak $t\geq0$
%            dogodek $\{\tau \leq t\}$ element $\mathcal{F}_t$.
%            \label{def:casUstavljanja}
%        \end{definicija}
%    
%        \begin{definicija}
%            Naj bo $(\Omega, \mathcal{F}, \mathbb{P}, (\mathcal{F}_t)_{t\geq0})$ filtriran verjetnostni
%            in $\tau$ "cas ustavljanja glede na filtracijo $(\mathcal{F}_t)_{t\geq0}$. Potem je
%            \begin{equation*}
%                \mathcal{F}_\tau = \bigl\{A \in \mathcal{F} \mid A \cap \{\tau \leq t\} \in \mathcal{F} \ \text{za vsak} \ t\geq 0\bigr\}
%            \end{equation*}
%            \textit{$\sigma$-algebra zgodovine} "casa ustavljanja $\tau$.
%        \end{definicija}
%
%        \begin{definicija}
%            Naj bo $(\Omega, \mathcal{F}, \mathbb{P}, (\mathcal{F}_t)_{t\geq0})$ filtriran verjetnostni
%            prostor in $\tau$ "cas ustavljanja glede na filtracijo $(\mathcal{F}_t)_{t\geq0}$. Naj bo 
%            $(X_t)_{t\geq0}$ prilagojen slu"cajni proces. Pravimo, da $(X_t)_{t\geq0}$ zado"s"ca
%            \textit{"sibki lastnosti Markova}, "ce za poljuben $t\geq 0$ proces 
%            \begin{equation*}
%                X_s - X_t \sim X_{s - t} \quad \text{za} \quad s\geq t.
%            \end{equation*}
%            Pravimo, da $(X_t)_{t\geq0}$ zado"s"ca \textit{krepki lastnosti Markova}, "ce za
%            $s \geq 0$ velja 
%            \begin{equation*}
%                X_{s + \tau} - T_\tau \sim X_{s - \tau}.
%            \end{equation*}
%
%        \begin{izrek}
%            CPP zado"sca krepki in "sibki lastnosti Markova.
%        \end{izrek}
%
%        \begin{proof}
%            "Sibka lastnost sledi neposredno iz stactionarnosti prirastkov CPP \ref{trd:neodvPrirCPP}.
%            Dokaz krepke lastnosti Markova zahteva nekoliko ve"c truda.
%        \end{proof}
%
%            
%            \label{def:sibkaKrepkaLastnostMarkova}
%        \end{definicija}

    \subsection{Markiranje}
        V trditvi \ref{trd:vsotaCPDjeCPD} smo pokazali, da ima vsota neodvisnih sestavljenih Poissonovih porazdelitev 
        spet sestavljeno Poissonovo porazdelitev. Podobno lahko $\CPP$ razdelimo na ve"c neodvisnih
        sestavljenih Poissonovih procesov, tako da ga markiramo glede na "cas in vrednost posameznega 
        prihoda. 
 
        \begin{izrek}(o markiranju)
            Naj bo $(S_t)_{t\geq0}$ $\CPP$. Naj bodo $A_1, \dots, A_n$ disjunktne podmno"zice mno"zice 
            $[0, \infty) \times \R$. Potem so za fiksen $t\geq0$ slu"cajne spremenljivke
            \begin{equation}
                S_t^{(j)} = \sum_{i=1}^{N_t}X_i\mathbbm{1}_{A_j}\left(V_i, X_i\right), \quad j = 1, \dots, n,
                \label{eq:markedCPP}
            \end{equation}
            med seboj neodvisne ($V_i$ ozna"cuje "cas $i$-tega prihoda). "Se ve"c, za vsak $j = 1, \dots, n$ je
            \begin{equation*}
                S_t^{(j)} \sim \sum_{i=1}^{N_t}X_i\mathbbm{1}_{A_j}\left(U_i, X_i\right),
            \end{equation*}
            kjer je $(U_i)_{i\in\N}$ zaporedje neodvisnih (med sabo, ter "se od $N_t$ in $(X_i)_{i\in\N}$) 
            slu"cajnih spremenljivk, porazdeljenih enakomerno $U\left([0, t]\right)$. Slu"cajna spremenljivka 
            $S_t^{(j)}$ ima torej sestavljeno Poissonovo porazdelitev.
            \label{izr:MarkiranjeCPP}
        \end{izrek}

        \begin{proof}
            Za splo"sni $\text{HPP}(\lambda)$ velja lastnost vrstilnih statistik \refPriloga{trd:VrstilneStatistikeHPP},
            torej
            \begin{equation*}
                \left(V_1, \dots, \ V_k \mid N_t = k\right)\sim \left(U_{(1)}, \dots, \ U_{(k)}\right), \quad k\in\N.
            \end{equation*}
            Tako lahko za $j\in\{1, \dots, n\}$ pogojno porazdelitev vsote (\ref{eq:markedCPP}) na dogodek $\{N_t = k\}$ zapi"semo kot
            \begin{equation*}
                S_t^{(j)}\mid \{N_t = k\} \sim \sum_{i=1}^{k}X_i\mathbbm{1}_{A_j}\left(U_{(i)}, X_i\right).
            \end{equation*}
            Vrstni red sumandov je nepomemben. Z upo"stevanjem neodvisnosti in enake 
            porazdeljenosti slu"cajnih spremenljivk $X_i$ lahko dano pogojno porazdelitev zapi"semo kot 
            \begin{equation*}
                S_t^{(j)}\mid \{N_t = k\} \sim \sum_{i=1}^{k}X_i\mathbbm{1}_{A_j}\left(U_{i}, X_i\right).
            \end{equation*}
            (Bolj natan"cen argument bralec lahko najde v \cite{4} na strani 28.)
            Sedaj pa si poglejmo skupno karakteristi"cno funkcijo slu"cajnega vektorja \refPriloga{def:karakteristicnaSlucVektor}
            $(S_t^{(1)}, \dots, \ S_t^{(n)})$:
            \begin{align*}
                \varphi_{S_t^{(1)},\dots, S_t^{(n)}}(u_1, \dots, \ u_n) 
                    &= \E\left[\exp\left[iu_1S_t^{(1)} + \dots + iu_nS_t^{(n)}\right]\right] \\
                    &= \sum_{k =0}^\infty \E\left[\exp\left[iu_1S_t^{(1)} + \dots + iu_nS_t^{(n)}\right]\bigg| \ N_t = k\right]\Prob(N_t = k)\\
                    &= \sum_{k=0}^\infty \E\left[\exp\left[\sum_{j=1}^niu_j\sum_{i=1}^kX_i\mathbbm{1}_{A_j}\left(U_i, X_i\right)\right]\right]\Prob(N_t = k)\\
                    &= \sum_{k=0}^\infty \E\left[\exp\left[\sum_{i=1}^k\sum_{j=1}^niu_jX_i\mathbbm{1}_{A_j}\left(U_i, X_i\right)\right]\right]\Prob(N_t = k)\\
                    &= \E\left[\exp\left[\sum_{i=1}^{N_t}\sum_{j=1}^niu_jX_i\mathbbm{1}_{A_j}\left(U_i, X_i\right)\right]\right]. 
            \end{align*}
            Opazimo, da imamo v eksponentu sestavljeno Poissonovo vsoto, za katero poznamo obliko karakteristi"cne 
            funkcije iz posledice \ref{pos:RodovnaKarakteristicna}. Za namene berljivosti pi"simo

            \begin{align*}
                    \varphi_{S_t^{(1)},\dots, S_t^{(n)}}(u_1, \dots, \ u_n) &= e^{g(u_1, \dots, \ u_n)},
            \end{align*}
            kjer je
            \begin{align}
                    g(u_1, \dots, u_n) &= \lambda t \biggl(\E\left[\exp\left[\sum_{j = 1}^niu_jX_1\mathbbm{1}_{A_j}\left(U_1, X_1\right)\right]\right] - 1\biggr). \label{eq:logKarakteristicnaFunkcija1}
            \end{align}
            Ker so mno"zice $A_j$ disjunktne, velja identiteta
            \begin{equation*}
                \exp\left[\sum_{j = 1}^niu_jX_1\mathbbm{1}_{A_j}\left(U_1, X_1\right)\right] - 1 
                = \sum_{j=1}^n\left(\exp\left[iu_jX_1 - 1\right]\mathbbm{1}_{A_j}\left(U_1, X_1\right)\right).
            \end{equation*}
            Torej 
            \begin{align*}
                g(u_1, \dots, u_n) &=
                \lambda t\E\left[\exp\left[\sum_{j = 1}^niu_jX_1\mathbbm{1}_{A_j}\left(U_1, X_1\right)\right] - 1 \right] \\
                &= \lambda t\sum_{j=1}^n\E\left[\exp\left[iu_jX_1 - 1\right]\mathbbm{1}_{A_j}\left(U_1, X_1\right)\right] \\
                &= \lambda t\sum_{j=1}^n\E\left[\exp\left[iu_jX_1\mathbbm{1}_{A_j}\left(U_1, X_1\right) - 1\right]\right].
            \end{align*}
            Vidimo, da je desna stran ena"cbe (\ref{eq:logKarakteristicnaFunkcija1}) ravno vsota 
            eksponentov karakteristi"cnih funkcij $\varphi_{S_t^{(j)}}$. 
            Po trditvi \refPriloga{trd:karakteristicnaNeodivsnost} sledi, da so slu"cajne 
            spremenljivke $S_t^{(1)}, \dots, S_t^{(n)}$ neodvisne 
            in po obliki karakteristi"cne 
            funckcije vidimo, da so res porazdeljene sestavljeno Poissonovo.
        \end{proof}

        Izrek o markiranju CPP ima vrsto uporabnih posledic. Direktno nam poda enostaven
        alternativen dokaz za neodvisnost in stacionarnost prirastkov.

        \begin{posledica}
            $\CPP$ ima neodvisne in stacionarne prirastke.
        \end{posledica}

        \begin{proof}
            Naj bodo $0=t_0\leq t_1 < \cdots < t_n = t < \infty$
            poljubna realna "stevila. Naj bodo $A_1, \dots, \ A_n$ neodvisne podmno"zice
            mno"zice $[0, \infty) \times \R$ definirane kot
            \begin{equation*}
                A_1 = [0, t_1]\times \R, \quad A_j = (t_{j-1}, t_j]\times \R, \quad \text{za} \ j = 2, \dots, n.
            \end{equation*}
            Potem so po izreku o markiranju slu"cajne spremenljivke $S_{t}^{(1)}, \dots, \ S_{t}^{(n)}$ neodvisne in 
            po osnovni ekvivalenci $\{i\leq N_{t_j}\} \iff \{V_i \leq t_j\}$ velja 
            \begin{align*}
                S^{(j)}_{t} = 
                \sum_{i = N_{t_{j-1}}+ 1}^{N_{t_j}}X_i = S_{t_j} - S_{t_{j-1}} \quad \text{za} \ j = 1, \dots, n.
            \end{align*}
            Ker je $U_i \sim U([0, t])$, je $\mathbbm{1}({t_{j-1}} < U_i\leq t_j) \sim \mathbbm{1}(U_i \leq t_j - t_{j-1})$.
            Zaradi neodvisnosti $U_i$ in $X_i$ pa je tudi 
            \begin{align*}
                S_{t_j} - S_{t_{j-1}} = S_t^{(j)} &\sim \sum_{i=1}^{N_t}X_i\mathbbm{1}_{A_j}\left(U_i, X_i\right) \\
                &\sim \sum_{i=1}^{N_t}X_i\mathbbm{1}({t_{j-1}} < U_i\leq t_j) \\
                &\sim \sum_{i=1}^{N_t}X_i\mathbbm{1}(U_i \leq t_j - t_{j-1}).
            \end{align*}
            Torej je $S_{t_j} - S_{t_{j-1}} \sim S_{t_j - t_{j-1}}$.
        \end{proof}

        \begin{posledica}
            "Ce v predpostavkah izreka o markiranju $\CPP$ sprostimo $t\geq0$, so procesi $(S_t^{(j)})_{t\geq0}$ 
            med seboj neodvisni \refPriloga{def:neodvisnostProcesov} in imajo neodvisne prirastke. 
            \label{pos:neodvisnostCPPmedSabo}
        \end{posledica}

        \begin{proof}
            Po definiciji neodvisnosti slu"cajnih procesov \refPriloga{def:neodvisnostProcesov} in definiciji neodvisnoti 
            prirastkov \refPriloga{def:prirastek} je dovolj pokazati, da so za poljuna realna "stevila
            $$
                0 = t_0^{(j)} < t_1^{(j)} < \cdots < t_{m_j}^{(j)} < \infty; \quad j = 1, \dots, n
            $$
            vsi prirastki 
            $$
                S^{(j)}_{t_k^{(j)}} - S^{(j)}_{t_{k-1}^{(j)}}; \quad j = 1, \dots, n, \quad k = 1, \dots, m_j
            $$
            med seboj neodvisni. Velja 
            $$
                S^{(j)}_{t_k^{(j)}} - S^{(j)}_{t_{k-1}^{(j)}} = \sum_{i = N_{t_{k-1}^{(j)}}+1}^{N_{t_k^{(j)}}}X_i\mathbbm{1}_{A_{jk}}\left(V_i, X_i\right).
            $$
            Iz osnovne ekvivalence $\{i\leq N_{t_j}\} \iff \{V_i \leq t_j\}$ dobimo, da je tudi 
            $$
                S^{(j)}_{t_k^{(j)}} - S^{(j)}_{t_{k-1}^{(j)}} = \sum_{i = 1}^{N_t} X_i\mathbbm{1}_{A_{jk}}\left(V_i, X_i\right),
            $$
            kjer je $t = \max\{t_{m_1}^{(1)}, \dots, t_{m_n}^{(n)}\}$ in $A_{jk} = A_j \cap ((t_{k-1}^{(j)}, t_k^{(j)}]\times \R)$.
            Neodvisnost prirastkov sledi iz disjunktnosti mno"zic $A_{jk}$ in izreka o markiranju.
        \end{proof}
    
        \begin{zgled}
            Markiranje je v praksi posebej uporabno pri analizi produktov 
            proporcionalnega pozavarovanja, kjer si zavarovalnica in  
            pozavarovalnica delita obveznosti in dobi"cek glede na velikost zahtevkov. Naj bo $(S_t)_{t\geq0}$ $\CPP$,  
            v katerem se"stevamo nenegativne slu"cajne spremenljivke $X_i$, ki predstavljajo zahtevke, 
            ki jih zavarovalnica prejema, in $ 0 = d_0 < d_1 < \cdots < d_n < \infty$ pozitivna realna "stevila. 
            Za fiksen $t\geq0$ definiramo 
            $A_1, \dots, \ A_{n }$, ki so disjunktne podmno"zice  mno"zice $[0, \infty)^2$ oblike
            \begin{equation*}
                A_j = [0, t]\times [d_{j-1}, d_j), \ j = 1, \dots, n.
            \end{equation*}
            Mno"zice $A_1, \dots, A_n$ predstvaljajo razli"cne sloje zahtevkov, ki jih zavarovalnica prejema. 
            Za vsak sloj lahko predpi"semo kolik"sen dele"z "skode bo pozavarovalnica krila.
            Po izreku o markiranju vemo, da so slu"cajne spremenljivke $S_t^{(1)}, \dots, \ S_t^{(n)}$ 
            med seboj neodvisne in imajo sestavljeno Poissonovo porazdelitev. Tako lahko analiziramo 
            posamezen sloj kot produkt sam zase in ustrezno dolo"cimo dele"z premij, ki ga pozavarovalnica dobi 
            za kritje zahtevkov dolo"cenega sloja. "Ce v pogodbi ne dolo"cimo kon"cnega "casa $t$,
             nam posledica \ref{pos:neodvisnostCPPmedSabo} zagotavlja, da so procesi 
            $(S_t^{(1)})_{t\geq0}, \dots, \ (S_t^{(n)})_{t\geq0}$ vselej medsebojno neodvisni.
        \end{zgled}


    %\subsection{Neskon"cna deljivost}
    %    Koncept neskon"cne deljivosti je temeljni pri "studiranju splo"snejsih slu"cajnih procesov z 
    %    neodvisnimi in stacionarnimi prirastki. Teorijo je v 1950-ih razvil francoski matematik Paul Lévy
    %    in v 80. letih prej"snjega stoletja je postalo ime Lévijev proces standardno v literaturi. 
    %    Poglavje se razlikuje od ostalih, ker 
    %    je snov precej bolj napredna in tehni"cna "ce se spustimo v podrobnosti zato Predstavimo le nekaj razultatov
    %    iz prvih poglavji knjige \cite{10} 
    %    brez dokazov in komentiramo zakaj sta sestavljena Poissonova porazdelitev in proces temeljna pri obravnavi
    %    tovrstnih procesov. 
%
    %\begin{definicija}
    %    Naj bo $X$ slu"cajna spremenljivka. Pravimo, da je $X$ \textit{neskon"cno deljiva}, "ce za vsak $n\in\N$
    %    obstajajo neodvisne enako porazdeljene slu"cajne spremenljivke $X_1, \dots, X_n$, da velja
    %    \begin{equation*}
    %        X \sim X_1 + X_2 + \cdots + X_n.
    %    \end{equation*}
    %    \label{def:neskoncnaDeljivost}
    %\end{definicija}
%
    %\begin{zgled}
    %    Naj bo $N\sim \text{Pois}(\lambda)$. Potem je $X$ neskon"cno deljiva. To neposredno sledi 
    %    iz lastnsoti, da za $n\in\N$ velja $X\sim Y_1 + \cdots + Y_n$ kjer so $Y_i\sim\Pois{\frac{\lambda}{n}}$ neodvisne 
    %    enako porazdeljene. 
    %\end{zgled}
%
    %\begin{zgled}
    %    Naj bo $S = \sum_{i=1}^NX_i$ slu"cajna spremenljivka, porazdeljena sestavljeno Poissonovo.
    %    Potem je $S$ neskon"cno deljiva.
    %    Za $n\in\N$ definiramo neodivsne enako porazdeljene slu"cajne spremenljivke $S^{(1)}, \dots, \ S^{(n)}$ 
    %    kot
    %    \begin{equation*}
    %        S^{(j)} = \sum_{i=1}^{M}X_i, \quad j = 1, \dots, n,
    %    \end{equation*}
    %    kjer je $M\sim\Pois{\frac{\lambda}{n}}$. Po trditvi \ref{trd:vsotaCPDjeCPD} je 
    %    $S^{(1)} + \cdots + S^{(n)}\sim S$. 
    %\end{zgled}
%
    %Enostavno smo pokazali, da sta Poissonova in sestavljena Poissonova porazdelitev neskon"cno deljivi. 
    %Naslednji rezultat je precej bolj zpresenetljiv, ker se izka"ze, da ima vsaka neskon"cno deljiva 
    %slu"cajna spremenljivka v nekem smislu sestavljeno Poissonovo porazdelitev.
    %\begin{trditev}
    %    Naj bo $S$ slu"cajna spremenljivka, ki zavzame vrednosti v $\N_0$ in je neskon"cno deljiva.
    %    Potem ima $S$ sestavljeno Poissonovo porazdelitev.
    %    \label{trd:neskoncnoDeljivaYslediCPD}
    %\end{trditev}
%
    %%\begin{proof}
    %%    Ozna"cimo rodovno funkcijo $S$ z 
    %%    \begin{equation*}
    %%        G_S(u) = \sum_{k = 0}^\infty \underbrace{\Prob\left(S = k\right)}_{p_k}u^k.
    %%    \end{equation*} 
    %%    Pokazali bomo, da je $G_S(u)$ enaka rodovni funkciji neke slu"cajne spremenljivke, ki ima sestavljeno
    %%    Poissonovo porazdelitev. Ker za nenegativne celo"stevilske slu"cajne spremenljivke velja 
    %%    $\Prob\left(S = k\right) = \frac{G_S^{(k)}(0)}{k!}$, bo to pomenilo, da je $S$ sestavljeno Poissonova, 
    %%    saj v tem primeru rodovna funkcija dolo"ca porazdelitev $S$.
    %%    Ker je $S$ neskon"cno deljiva potem je za vsak $n\in\N$ 
%%
    %%    \begin{equation*}
    %%        G_{S_n}(u) := \left(G_S(u)\right)^{\frac{1}{n}} = 
    %%        \sum_{k = 0}^\infty\underbrace{\Prob\left(S^{(n)} = k\right)}_{p_{k_n}}u^k
    %%    \end{equation*}
    %%    rodovna funkcija neke slu"cajne spremenljivke $S^{(n)}$ in za vsak $u\in\R$ velja enakost
    %%    \begin{equation*}
    %%        G_{S^{(n)}}(u) = \left(G_{S^{(1)}}(u)\right)^n \ \text{oziroma} \ 
    %%        \sum_{k = 0}^\infty p_{k}u^k = \left(\sum_{k = 0}^\infty p_{k_n}u^k\right)^n.
    %%    \end{equation*}
    %%    "Ce raz"sirimo desno stran ena"cbe in predpostavimo $p_0 = 0$, dobimo, da mora biti 
    %%    $p_{0_n} = 0$ in posledi"cno tudi $p_1 = p_2 = \cdots = p_{n-1} = 0$. Ker to velja za poljuben 
    %%    $n\in\N$ dobimo, da je $G_S(u) = 0$, kar pa je protislovje. Torej $p_0 > 0$ in zagotovo 
    %%    $G_S(u) > 0$ za $u\in[0, 1]$. Za $u\in[0, 1]$ velja tudi $\lim_{n\to\infty}\left(\frac{G_S(u)}{p_0}\right)^{\frac{1}{n}} = 1$.
    %%    Velja $\lim_{x \to 0}\frac{\ln(1 + x)}{x} = 1.$
    %%    \begin{equation*}
    %%        \ln\left(\left(\frac{G_S(u)}{p_0}\right)^{\frac{1}{n}}\right) = \ln\left( 1 + \left(\left(\frac{G_S(u)}{p_0}\right)^{\frac{1}{n}} - 1\right)\right)
    %%        \approx \left(\frac{G_S(u)}{p_0}\right)^{\frac{1}{n}} - 1 \ \text{ko} \ n\to\infty.
    %%    \end{equation*}
    %%    Za $u = 1$ dobimo
    %%    \begin{equation*}
    %%        \ln\left(\left(\frac{1}{p_0}\right)^{\frac{1}{n}}\right) \approx \left(\frac{1}{p_0}\right)^{\frac{1}{p_0}} - 1 \ \text{ko} \ n\to\infty.
    %%    \end{equation*}
    %%\end{proof}
    %Trditev \ref{trd:neskoncnoDeljivaYslediCPD} posplo"simo na vse
    %neskon"cno deljive slu"cajne spremenljivke s tem, da zahtevamo le konvergenco v porazdelitvi 
    %(\refPriloga{def:KonvergencaVPorazdelitvi}).
    %
%
    %\begin{trditev}
    %    Naj ima slu"cajna spremenljivka $S$ neskon"cno deljivo porazdelitev. Potem obstaja zaporedje 
    %    slu"cajnih spremenljivk $(S_n)_{n\in\N}$, ki imajo sestavljeno Poissonovo porazdelitev in
    %    konvergirajo proti $S$ v porazdelitvi.
    %\end{trditev}
%
    %
    %
    %\begin{definicija}
    %    Slučajnemu procesu $(X_t)_{t\geq0}$ definiranem na verjetnostnemu
    %    prostoru $(\Omega, \mathcal{F}, \Prob)$ pravimo \textit{Lévijev proces}, če zadošča naslednjim pogojem:
    %    \begin{enumerate}
    %        \item $\Prob(X_0 = 0)=1$.
    %        \item Trajektorije $(X_t)_{t\geq0}$ so $\Prob$-skoraj gotovo zvezne z desne z levimi limitami (cádlág).
    %        \item $(X_t)_{t\geq0}$ ima neodvisne in stacionarne prirastke.
    %    \end{enumerate}
    %\end{definicija}
%
    %Takoj vidimo, da homogeni in sestavljeni Poissonov proces zadoščata prvemu in tretjemu pogoju. 
    %Izka"ze se, da imata tudi cádlág trajektorije in sta torej Lévijeva procesa.
    %Naslednji izrek nam pove, da Lévijeve procese karakteriziramo z neskon"cno deljivimi porazdelitvami.
%
    %\begin{izrek}(str 35)
    %    Naj bo $(X_t)_{t\geq0}$ Lévijev proces z vrednostmi v $\R$, potem je za fiksen $t\geq0$
    %    slu"cajna spremenljivka $X_t$ neskon"cno deljiva. Obratno, "ce je slu"cajna spremenljivka $X$ 
    %    neskon"cno deljiva, obstaja Lévijev proces $(X_t)_{t\geq0}$ na $\R$, da za $t=1$ velja
    %    $X_1\sim X$. "Ce sta $(X_t)_{t\geq0}$ in $(Y_t)_{t\geq0}$ dva Lévijeva procesa z vrednostmi V
    %    $\R$ in velja $X_1\sim Y_1$, potem sta procesa enako porazdeljena. 
    %\end{izrek}
%
    %Tako kot lahko poljubno neskon"cno deljivo porazdelitev dobimo z limito sestavljenih Poissonovih
    %se podobno izka"ze, da poljuben Lévijev proces izrazimo kot vsoto sestavljenega Poissonovega procesa
    %in Brownovega gibanja\footnote{Brownovo gibanje $(B_t)_{t\geq0}$ je Lévijev proces kjer poleg zahtev $(1)$ in $(3)$
    %dodatno zahtevamo, da za $t\geq0$ velja $B_t\sim N(0, t)$ in da ima proces zvezne trajektorije (posledi"cno so seveda cádlág).}.
    %To je posledica Lévy-Hin"cinove formule za karateristi"cno funkcijo Lévijevih procesov kateri je 
    %namenjeno drugo poglavje knjige \cite{10}. Izka"ze se, da za poljuben Lévijev proces $(X_t)_{t\geq0}$ obstajata enoli"cno 
    %dolo"ceni "stevili $\gamma\in\R$ in $\sigma>0$ ter kon"cna mera $\nu$ na $\R$ za katero velja $\nu(\{0\}) = 0$ in $\int_{\R}\min\{1, x^2\}\nu(dx)<\infty$, 
    %ki popolnoma dolo"cajo njegovo porazdelitev. Tedaj velja 
    %$$
    %    X_t = \gamma t + \sigma B_t + S_t, \qquad t\geq0,
    %$$
    %kjer je $(S_t)_{t\geq0}$ sestavljen Poissonov proces z intenzivnostjo $\lambda = \nu(\{\R\})$ in kjer 
    %imajo $X_i$ porazdelitveno funkcijo $F_{X_1}(x) = \frac{1}{\lambda}\int_{(-\infty, x]}\nu(dy)$.
%
    %
    %V tem razdleku smo si pogledali nekoliko bolj napredno in nadvse zanimivo teorijo slu"cajnih 
    %procesov ter njeno povezavo s sestavljenim Poissonovim procesom. 
    %Snov pa tehni"cno presega dodiplomsko raven in zato ne bomo nadaljevali v tej smeri. V drugem delu diplome se 
    %bomo raje posvetili prakti"cnemu primeru uporabe sestavljenega Poissonovega procesa v zavarovalni"stvu.

    \newpage

\section{Cramér--Lundbergov model}
    \noindent
    Razdelek je prirejen po \cite{3}, \cite{4},  \cite{5} in \cite{9}.

    V tem razdelku obravnavamo najbolj intenzivno raziskan model v teoriji propada, običajno imenovan 
    Cramér--Lundbergov model. V svoji najosnovnejši obliki 
    ga je na za"cetku 20.\ stoletja izpeljal "svedski aktuar Filip Lundberg, da bi ocenil ranljivost 
    zavarovalnice za propad. Čeprav je model v svoji ideji dokaj preprost, 
     zajema bistvo povezave ravni rezerv zavarovalnice in njene izpostavljenosti tveganju, 
    kar je razlog, zakaj je postal temeljni merilni model v teoriji propada.
    V preteklem stoletju je bilo razvitih veliko tehnik za analizo Cramér--Lundbergovega modela, 
    ki so se večinoma osredotočale na kvantifikacijo verjetnosti propada zavarovalnice. V razdelku definiramo 
    model in izpeljemo Lundbergovo neenakost ter asimptoti"cno obna"sanje verjetnosti propada v primeru, ko 
    zavarovalni"ske zahtevke modeliramo z lahkorepimi in te"zkorepimi porazdelitvami. V zgledih 
    poka"zemo, kako do rezultatov, ki nam jih zagotavlja teorija, pridemo v praksi z Monte Carlo 
    simulacijami procesa tveganja.

    \subsection{Proces tveganja in verjetnost propada}

        \begin{definicija}
            Naj bo $(S_t)_{t\geq0 }$ $\CPP$, kjer so slu"cajne spremenljivke $(X_i)_{i\in\N}$, 
            ki jih se"stevamo skoraj gotovo nenegativne.\ \textit{Proces tveganja} v Cramér--Lundbergovem 
            modelu definiramo kot dru"zino slu"cajnih spremenljivk 
            \begin{align*}
                U_t = u + p(t) - S_t, \quad t\geq0,
            \end{align*}
            kjer je $u \geq 0$ za"cetni kapital zavarovalnice in $p$ funkcija prihodkov iz premij. 
            \label{def:procesTveganja}
        \end{definicija}

        \begin{opomba}
            V resnici lahko veliko lastnosti procesa tveganja izpeljemo brez predpostavke, da prihodi 
            zahtevkov v $(S_t)_{t\geq0}$ sledijo homogenemu Poissonovemu procesu,
            ampak lahko privzamemo, da sledijo splo"snemu prenovitvenemu procesu. 
            Zato bomo pri dokazovanju nekaterih rezultatov medprihodne "case zahtevkov $T_i$ obravnavali v 
            splo"snem, ne da bi predpostavili, da so eksponentno porazdeljeni.
            \label{op:procesTveganja}
        \end{opomba}

        Vrednost $U_t$ predstavlja kapital zavarovalnice ob "casu $t\geq0$. Standardno je  
        vzeti deterministi"cno funkcijo $p(t) = ct$, kjer je $c>0$ stopnja prihodkov premij.
        Uporaba linearne funkcije za modeliranje premijskega dohodka v Cramér--Lundbergovem 
        modelu ponuja realističen približek zato, ker zavarovalnice pogosto doživljajo 
        stabilno povečevanje premijskega dohodka skozi čas. Poleg tega je izbira linearne 
        funkcije preprosta, zato bomo v nadaljevanju privzeli $p(t) = ct$. Poglejmo si 
        realizaciji procesa tveganja, ko so zahtevki $X_i$ porazdeljeni Weibullovo 
        \refPriloga{def:WeibullovaPorazdelitev} z razli"cnimi parametri. 

        \begin{zgled}
            Naj bo $(U_t)_{t\geq0}$ proces tveganja v Cramér--Lundbergovem modelu z za"cetnim kapitalom
            $u = 1000$ in $p(t) = 200t$ ter intenzivnostjo prihodov zahtevkov $\lambda=1$. % v $CPP$. 
            Naj bodo v prvem primeru (rde"ca) 
            zahtevki porazdeljeni kot $X_i \sim \text{Weibull}(2, 434)$ in v drugem primeru (modra) kot
            $Y_i \sim \text{Weibull}(\tfrac{1}{4}, 16)$.
            
            \begin{figure}[H]
                \centering
                \includegraphics[width=\textwidth]{
                    C:/Users/38651/OneDrive - Univerza v Ljubljani/Desktop/Diploma/Diplomski-seminar/GraphsAndPhotos/slika2.pdf
                    }
                \caption{Realizaciji procesa tveganja}
                \label{fig:slika3}
            \end{figure}

            \noindent
            Pri obeh realizacijah vidimo, da proces tveganja v nekem trenutku pade pod $0$ (tam ga 
            tudi ustavimo). "Ceprav je pri"cakova vrednost 
            $\E\left[Y_i\right] = 384 \approx \E\left[X_i\right] = 217\sqrt{\pi} \approx 384{,}62$ 
            opazimo bistveno razliko med realizacijama. V rde"cem primeru proces pade pod
            $0$ po ve"c zaporednih manj"sih izgubah, v modrem primeru pa po eni zelo veliki izgubi. 
            V nadaljevanju bomo primera lo"cili, ampak pred tem 
            definirajmo osnovne pojme, ki jih bomo obravnavali v razdelku.

            \label{zgd:weibullProcesTveganja}
        \end{zgled}

        \begin{definicija}
            \textit{Propad} definiramo kot dogodek, da proces tveganja $(U_t)_{t\geq0}$ kadarkoli pade pod $0$. 
            To je torej dogodek 
            \begin{align*}
                \bigl\{U_t<0 \ \text{za neki} \ t\geq 0\bigr\}.
            \end{align*}
            "Casu ustavljanja
            \begin{align*}
                T = \inf\{t\geq0 \mid U_t < 0\}
            \end{align*}
            pa pravimo \textit{"cas propada}. Seveda velja enakost
            \begin{align*}
                \{U_t<0 \ \text{za neki} \ t\geq0\} = \{T<\infty\}.
            \end{align*}
            \label{def:PropadCasPropada} 
        \end{definicija}

        \begin{definicija}
            \textit{Verjetnost propada} definiramo kot funkcijo $\psi: (0,\infty) \to [0,1]$ 
            s predpisom
            \begin{align*}
                \psi(u) = \Prob_u(T<\infty); \quad u\geq0.
            \end{align*}
            Ker proces tveganja vedno gledamo 
            v odvisnosti od  za"cetnega kapitala, za fiksen $u\geq0$ pripadajo"co verjetnostno mero
            ozna"cimo s $\Prob_u$.
            \label{def:VerjetnostPropada}
        \end{definicija}

        \begin{definicija}
            Po konstrukciji procesa tveganja $(U_t)_{t\geq0}$ je propad mogo"c le ob 
            prihodih zahtevkov. %, ki sledijo $HPP(\lambda)$
            Z $V_n$ ozna"cimo "cas, ob katerem prispe $n$-ti zahtevek, in definiramo 
            \textit{ogrodje procesa tveganja} kot $(U_{V_n})_{n\in\N}$.
            \label{def:ogrodjeProcesaTveganja}
        \end{definicija}

        \begin{trditev}
            Naj bo $(U_t)_{t\geq0}$ proces tveganja v Cramér--Lundbergovem modelu in $(U_{V_n})_{n\in\N}$ 
            njegovo ogrodje ter $T_n := V_n - V_{n-1}$ medprihodni "cas $n$-tega zahtevka 
            $(V_0 = T_0 = 0)$. Potem velja 
            \begin{equation*}
                \psi(u) = \Prob\left(\sup_{n\in\N}Z_n > u\right),
            \end{equation*}
            kjer je $Z_n = \sum_{i=1}^nY_i$  kumulativna izguba po $n$ zahtevkih in $Y_i = X_i - cT_i$
            izguba $i$-tega prihoda.
            \label{trd:verjetnostPropadaZOgrodjem}
        \end{trditev}

        \begin{proof}

            S pomo"cjo ogrodja procesa tveganja lahko dogodek propada zapi"semo kot
            \begin{align*}
                \bigl\{U_t<0 \ \text{za neki} \ t\geq 0\bigr\} &= 
                                \biggl\{\inf_{t\geq0}U_t<0\biggr\} \\
                              &= \biggl\{\inf_{n\in\N}U_{V_n}<0\biggr\} \\
                              &= \biggl\{\inf_{n\in\N}\bigl\{u + p(V_n) - S_{V_n}\bigr\} < 0\biggr\} \\
                              &= \biggl\{\inf_{n\in\N}\biggl\{u + 
                              \underbrace{cV_n - \sum_{i=1}^nX_i}_{-Z_n}\biggr\} < 0\biggr\} \\
                              &= \biggl\{\inf_{n\in\N}\{-Z_n\} < -u\biggr\} \\
                              &= \biggl\{\sup_{n\in\N}Z_n > u\biggr\},
            \end{align*}
            kar nam da "zeleno enakost.
        \end{proof}

        Tako verjetnost propada prevedemo na prehodno verjetnost diskretnega slu"cajnega 
        sprehoda $(Z_n)_{n\in\N}$.
        
        Cilj obravnavanja verjetnosti propada v 
        Cramér--Lundbergovem modelu je, da se izognemo propadu oziroma da je verjetnost, 
        da kumulativna izguba $(Z_n)_{n\in\N}$ prese"ze $u$,
        tako majhna, da lahko v praksi dogodek propada izklju"cimo. 

        \begin{trditev}
            Naj bo $(Z_n)_{n\in\N}$ zaporedje slu"cajnih spremenljivk, definirano kot 
            $Z_n = \sum_{i=1}^nY_i$ za neodvisne in enako porazdeljene slu"cajne spremenljivke 
            $Y_i$ z $\E\left[Y_i\right] < \infty$. 
            "Ce velja $\E\left[Y_i\right] \geq 0$, za vsak $u\geq0$ velja
            \begin{equation*}
                \Prob\left(\sup_{n\in\N}Z_n > u\right) = 1.
            \end{equation*}
            \label{trd:propadZVerjetnostjo1}
        \end{trditev}

        \begin{proof}
            Zaporedje slu"cajnih spremenljivk $(Y_i)_{i\in\N}$ zado"s"ca predpostavkam krepkega zakona
            velikih "stevil \refPriloga{izr:KrepkiZakonVelikihStevil}. Velja
            \begin{equation*}
                \frac{Y_1 + Y_2 + \cdots + Y_n}{n} = \frac{Z_n}{n} \xrightarrow[n\to\infty]{s.g.} \E\left[Y_n\right].
            \end{equation*}
            Torej bo $Z_n$ v primeru, ko je $\E\left[Y_n\right]>0$, 
             skoraj gotovo asimptoti"cno linearno nara"s"cal proti $\infty$ kot $n \E\left[Y_n\right]$, 
            zato bo veljalo celo
            \begin{equation*}
                \Prob\left(\sup_{n\in\N}Z_n = \infty\right) = \Prob\left(\sup_{n\in\N}Z_n > u\right) = 1,
            \end{equation*}
            za poljuben $u\geq 0$. Dokaz za primer, ko je $\E\left[Y_n\right] = 0$, pa
            je precej bolj tehni"cen in ne preve"c informativen, zato 
            ga bomo izpustili. Izka"ze se, da vedno obstajata neki podzaporedji $(n_k)_{k\in\N}$ in $(m_k)_{k\in\N}$, 
            za kateri gre
            $Z_{n_k} \xrightarrow[k\to\infty]{s.g.}\infty$ in 
            $Z_{m_k} \xrightarrow[k\to\infty]{s.g.}-\infty$.
            Dokaz lahko bralec najde v \cite{6} v poglavju 4.
        \end{proof}

        \begin{opomba}
            Iz trditve \ref{trd:propadZVerjetnostjo1} (ob predpostavkah $\E\left[X_i\right] < \infty$
            in $\E\left[T_i\right] < \infty$) sledi, da moramo premijo (in s tem $c$) izbrati tako, da bo 
            $\E\left[Y_i\right] < 0$, saj bo tako $Z_{n} \xrightarrow[n\to\infty]{s.g.}-\infty$
            in je to edini primer, ko lahko upamo, da verjetnost propada ne bo
            enaka 1.
            \label{op:izbiraPremije}
        \end{opomba}
    
        %V nadaljevanju bomo
        %predpostavili, da sta $\E\left[X_n\right]$ in $\E\left[W_n\right]$ kon"cni. To nam 
        %zagotovi, da je $\E\left[Z_n\right] = \sum_{i=1}^n\E\left[X_i\right] - c\E\left[W_i\right]$
        %kon"cna.
%
        %Ker pa velja ena od treh mo"znosti:
        %\begin{align*}
        %    \E\left[Y_n\right] = 
        %    \begin{cases}
        %        > 0, & \text{za} \ c\E\left[W_n\right] > \E\left[X_n\right], \\
        %        = 0, & \text{za} \ c\E\left[W_n\right] = \E\left[X_n\right], \\
        %        < 0, & \text{za} \ c\E\left[W_n\right] < \E\left[X_n\right],
        %    \end{cases}
        %\end{align*}
%
        %\noindent
        %bo verjetnost propada enaka 1, "ce velja $\E\left[Y_n\right] > 0$, saj $(Z_n)_{n\in\N}$ 
        %zado"s"ca predpostavkam krepkega zakona velikih "stevil.
        %\begin{equation*}
        %    \frac{Z_n}{n} \xrightarrow[n\to\infty]{s.g.} \E\left[Y_n\right] > 0.
        %\end{equation*}
        %Vidimo torej, da bo $Z_n$ skoraj gotovo linearno nara"scal proti $\infty$ in bo za poljuben 
        %$u>0$
        %\begin{equation*}
        %    \Prob\left(\sup_{n\in\N}Z_n > u\right) = 1.
        %\end{equation*}
%
        %Izka"ze se da celo v 
        %primeru ko bo $\E\left[Y_n\right] = 0$ bo verjetnost propada enaka 1, ker bosta obstajali 
        %pozdaporedji $(n_k)_{k\in\N}$ in $(m_k)_{k\in\N}$... Spitzer [138] je dokazano.

        \begin{definicija}
            Pravimo, da proces tveganja $(U_t)_{t\geq0}$ v Cramér--Lundbergovem modelu
             zado"s"ca \textit{pogoju neto zaslu"zka} (ang. \textit{net profit condition}), "ce velja 
            \begin{equation*}
                c > \frac{\E\left[X_1\right]}{\E\left[T_1\right]}, \quad \text{oziroma} \quad 
                c = (1 + \rho)\frac{\E\left[X_1\right]}{\E\left[T_1\right]} \quad \text{za $\rho > 0$}.
            \end{equation*}
            Pogoj bomo v nadaljevanju imenovali NPC.
            \label{def:NPC}
        \end{definicija}

        Zahteva NPC za analizo poslovanja zavarovalnice je dokaj intuitivna, saj pove, da mora  
        biti v nekem "casovnem intervalu pri"cakovani dohodek iz premij ve"cji od pri"cakovanega izpla"cila zahtevkov.

    \subsection{Verjetnost pre"zivetja kot integralska ena"cba} Od sedaj naprej predpostavimo, 
    da je $(S_t)_{t\geq0}$ v procesu tveganja $(U_t)_{t\geq0}$ sestavljeni Poissonov proces. V nadaljevanju nas bo predvsem zanimalo asimptoti"cno 
        vedenje $\psi(u)$, ko gre $u\rightarrow\infty$. Verjetnost propada zelo redko lahko eksplicitno 
        izra"cunamo, ampak veliko lahko povemo o redu konvergence s tem, da jo izrazimo kot integralsko
        ena"cbo\footnote{V delu se dr"zimo konvencije, da Riemannov integral funkcije $f$ ozna"cimo kot
        $\int_a^bf(x)dx$. Lebesgueov integral po Lebesgue-Stieltjesovi meri, porojeni s porazdelitveno
        funkcijo $F$, pa $\int_{(a, b]}f(x)dF(x)$.}.

    \begin{definicija}
        Za lep"so notacijo v nadaljevanju definiramo "se funkcijo \textit{verjetnosti pre"zivetja} kot
        $\theta:(0, \infty) \to [0, 1]$ s predpisom
        \begin{equation*}
            \theta(u) = \Prob_u\left(T=\infty\right) = 1 - \psi(u); \quad u\geq 0.
        \end{equation*}
        \label{def:verjetnostPrezivetja}
    \end{definicija}

    \begin{lema}(Integralska ena"cba za verjetnost pre"zivetja)
        Naj bo $(U_t)_{t\geq0}$ proces tveganja v Cramér--Lundbergovem modelu, ki zado"s"ca NPC, ter naj 
        velja $\E\left[X_1\right]<\infty$ in da imajo slu"cajne spremenljivke $(X_i)_{i\in\N}$ 
        gostoto. Potem $\theta$ zado"s"ca  enakosti
        \begin{equation}
            \theta(u) = \theta(0) + \frac{1}{(1+\rho)} \int_{(0, u]}\theta(u - x)d\overline{F}_{X_1}(x),
            \label{eq:verjetnostPrezivetja}
        \end{equation}
        kjer je $\overline{F}_{X_1}$ porazdelitev integriranega repa \refPriloga{def:porazdelitevintegriranegaRepa} 
        slu"cajne spremenljivke $X_1$.
        \label{lema:verjetnostPrezivetja}
    \end{lema}

    \begin{proof}
        Po trditvi \ref{trd:verjetnostPropadaZOgrodjem} velja
        \begin{equation*}
            \psi(u) = \Prob\left(\sup_{n\in\N}Z_n > u\right),
        \end{equation*}
        kjer je $Z_n = \sum_{i=1}^nY_i$ in $Y_i = X_i - cT_i$. Torej je
        \begin{align*}
            \theta(u) &= \Prob\left(\sup_{n\in\N}Z_n \leq u\right) \\
                      &= \Prob\biggl(\bigl\{Z_n \leq u \ \text{za vsak} \ n\in\N\bigr\}\biggr) \\
                      &= \Prob\biggl(\bigl\{Y_1 \leq u\bigr\}\cap \bigl\{Z_n - Y_1 \leq u - Y_1 \ \text{za vsak} \ n\geq2\bigr\}\biggr) \\
                      &= \E\biggl[\mathbbm{1}_{\{Y_1\leq u\}}\Prob\biggl(\bigl\{Z_n - Y_1 \leq u - Y_1 \ \text{za vsak} \ n\geq2\bigr\} \ \Big| \ Y_1\biggr)\biggr].
        \end{align*}
        Sedaj upo"stevamo, da je $Y_1 = X_1 - cT_1$ in je torej dogodek $\{Y_1 \leq u\}$ 
        enak dogodku $\{X_1 \leq u + cT_1\}$. Poleg tega velja, da je 
        $(Z_n - Y_1\mid Y_1)_{n\geq2} \sim (Z_n)_{n\in\N}$, saj so $Y_i$ neodvisne in enako porazdeljene
        slu"cajne spremenljivke.
        Upo"stevamo "se, da je  $T_1$ medprihodni "cas v $\HPP(\lambda)$ in dobimo

        \begin{align*}
                \theta(u)   &= \int_{(0, \infty)}\int_{(0, u + ct]}\Prob\biggl(\bigl\{Z_n \leq u - (x - ct) \ \text{za vse} \ n\in\N\bigr\}\biggr)dF_{X_1}(x)\,dF_{T_1}(t) \\
                            &= \int_0^\infty\int_{(0, u + ct]}\theta(u - x + ct)\,dF_{X_1}(x)\,\lambda e^{-\lambda t}dt.
        \end{align*}
        Uvedemo novo spremenljivko $z = u + ct$ $\bigl( \text{torej} \ t = \tfrac{z - u}{c}$ in $dt = \tfrac{dz}{c} \bigr)$ 
        in dobimo

        \begin{align*}
                    \theta(u) = \frac{\lambda}{c}\,e^{\frac{\lambda u}{c}}\int_u^\infty e^{-\frac{\lambda z}{c}}\underbrace{\int_{(0, z]}\theta(z - x)dF_{X_1}(x)}_{g(z)}dz.
        \end{align*}
        Ker ima porazdelitev $F_{X_1}$ gostoto in je $\theta$ zvezna omejena funkcija,
        sledi, da je funkcija $g$ zvezna in jo lahko (po osnovnem izreku analize)
        odvajamo. Tako dobimo

        \begin{equation*}
            \theta'(u) = \frac{\lambda}{c}\,\theta(u) - \frac{\lambda}{c}\int_{(0, u]}\theta(u - x)dF_{X_1}(x).
        \end{equation*}
        "Ce sedaj obe strani integriramo po $u$, dobimo
        
        \begin{equation}
            \int_0^t\theta'(u)du = \frac{\lambda}{c}\int_0^t\theta(u)du - \overbrace{\frac{\lambda}{c}\int_0^t\underbrace{\int_{(0, u]}\theta(u - x)dF_{X_1}(x)}_{(i)}du.}^{(ii)} 
            \label{eq:verjetnostPrezivetjaIntegral}
        \end{equation}
        Na integralu $(i)$ uporabimo per partes $\bigl(\alpha = \theta(u-x)$ in $d\beta = dF_{X_1}(x)\bigr)$:

        \begin{align*}
            (i)     &= \bigl(\theta(u - x)F_{X_1}(x)\bigr)\Big|_{0}^{u} + \int_0^u\theta'(u - x)F_{X_1}(x)dx \\
                    &= \theta(0)F_{X_1}(u) - \int_0^u\theta'(u - x)F_{X_1}(x)dx.
        \end{align*}
        Upo"stevamo, da je $F_{X_1}(0) = 0$, saj je $X_1 > 0$ skoraj gotovo. Vstavimo $(i)$ 
        v $(ii)$ in dobimo

        \begin{equation*}
            (ii) =  - \frac{\lambda}{c}\int_0^t\theta(0)F_{X_1}(u)du - \frac{\lambda}{c}\int_0^t\int_0^u\theta'(u - x)F_{X_1}(x)dxdu. 
        \end{equation*}
        Po Tonellijevem izreku \refPriloga{izr:TonellijevIzrek} lahko zamenjamo vrstni red integracije.

        \begin{align*}
            (ii)    &=  - \frac{\lambda}{c}\int_0^t\theta(0)F_{X_1}(u)du - \frac{\lambda}{c}\int_0^tF_{X_1}(x)\int_x^t\theta'(u - x)dudx \\
                    &= - \frac{\lambda}{c}\int_0^t\theta(0)F_{X_1}(u)du - \frac{\lambda}{c}\int_0^tF_{X_1}(x)\bigl(\theta(t-x) - \theta(0)\bigr)dx.\\
                    &= - \frac{\lambda}{c}\int_0^tF_{X_1}(x)\theta(t - x)dx.
        \end{align*}
        Vstavimo $(ii)$ v ena"cbo (\ref{eq:verjetnostPrezivetjaIntegral}) in dobimo
        \begin{align*}
            \theta(t) - \theta(0) &= \frac{\lambda}{c}\int_0^t\theta(u)du - \frac{\lambda}{c}\int_0^tF_{X_1}(x)\theta(t - x)dx,\\
            \theta(t) &= \theta(0) + \frac{\lambda}{c}\int_0^t\bigl(1 - F_{X_1}(x)\bigr)\theta(t - x)dx \\
                    &= \theta(0) + \frac{1}{1 + \rho}\int_{(0, u]}\theta(u - x)d\overline{F}_{X_1}(x).
        \end{align*}
        Pri enakosti v zadnji vrstici smo upo"stevali  
        \begin{equation*}
            \frac{\lambda}{c} = \frac{1}{1 + \rho}\frac{1}{\E\left[X_1\right]}
        \end{equation*}
        in zamenjali oznako spremenljivke $t\mapsto u$. S tem je lema dokazana. 

    \end{proof}

    \begin{opomba}
        Konstanto $\theta(0)$, ki se pojavi v (\ref{eq:verjetnostPrezivetja}), lahko izra"cunamo.\ Ker $c$ zado"s"ca NPC, po argumentu v dokazu trditve
        \ref{trd:propadZVerjetnostjo1} velja 
        \begin{equation*}
            Z_n \xrightarrow[n\to\infty]{s.g.} -\infty.
        \end{equation*}
        Po zveznosti $\Prob$ od spodaj sledi

        \begin{equation*}
            \lim_{u\to\infty}\Prob\left(\sup_{n\in\N}Z_n \leq u\right) = \Prob\left(\sup_{n\in\N}Z_n \leq \infty\right) = 1.
        \end{equation*}
        "Ce torej v ena"cbi (\ref{eq:verjetnostPrezivetja}) po"sljemo $u\to\infty$, dobimo

        \begin{equation*}
            \lim_{u\to\infty}\theta(u) = 1 = \theta(0) + \frac{1}{1 + \rho}\lim_{u\to\infty}\int_{(0, \infty)}\mathbbm{1}_{(0, u]}(x)\theta(u - x)d\overline{F}_{X_1}(x). \\
        \end{equation*}
        Po izreku o monotoni konvergenci \refPriloga{izr:monotonaKonvergenca} sledi

        \begin{align*}
            1 &= \theta(0) + \frac{1}{1 + \rho}\int_{(0, \infty)}1d\overline{F}_{X_1}(x) \\
             &= \theta(0) + \frac{1}{1 + \rho}.
        \end{align*}
        Torej je $\theta(0) = \frac{\rho}{1 + \rho}$.
        Enakost upo"stevamo v ena"cbi (\ref{eq:verjetnostPrezivetja}) in uvedemo oznako $\frac{1}{1 + \rho} = q$, kar nam da

        \begin{equation}
            \theta(u) = (1 - q) + q\int_{(0, u]}\theta(u - x)d\overline{F}_{X_1}(x).
            \label{eq:verjetnostPrezivetja3}
        \end{equation}
        \label{op:verjetnostPrezivetja2}
    \end{opomba}

    \begin{opomba}
    Integralska ena"cba (\ref{eq:verjetnostPrezivetja3}) skoraj ustreza definiciji prenovitvene ena"cbe
    \refPriloga{def:prenovitvenaEnacba}, s 
    to bistveno razliko, da $q\overline{F}_{X_1}$ ni verjetnostna mera, saj velja 
    $\lim_{x\to\infty}q\overline{F}_{X_1}(x) = q < 1$. Taki ena"cbi pravimo
    defektna prenovitvena ena"cba. V splo"snem lahko defektne prenovitvene ena"cbe re"sujemo s pomo"cjo 
    Banachovega izreka o negibni to"cki \refPriloga{izr:Banach}. Da funkcija $\theta$ re"si ena"cbo 
    (\ref{eq:verjetnostPrezivetja3}), lahko povemo tako, da je negibna to"cka operatorja 
    \begin{equation}
        Ag(u) = (1 - q) + q\int_{(0, u]}g(u - x)d\overline{F}_{X_1}(x),
        \label{eq:operator}
    \end{equation}
    ki je skr"citev na prostoru omejenih funkcij $B([0, \infty))$, opremljene s supremum metriko
    \begin{equation*}
        d_\infty(f, g) = \sup_{u\in[0, \infty)}\big|f(u) - g(u)\big|,  \quad f, g\in B([0, \infty)).
    \end{equation*}
    To enostavno poka"zemo z oceno
    \begin{align*}
        \sup_{u\in[0, \infty)}\big|Af(u) - Ag(v)\big| 
                    &= \sup_{u\in[0, \infty)}\bigg|q\int_{(0, u]}f(u- x) - g(u - x)d\overline{F}_{X_1}(x)\bigg| \\
                                &\leq \sup_{u\in[0, \infty)}q\int_{(0, u]}\big|f(u - x) - g(u - x)\big|d\overline{F}_{X_1}(x) \\
                                &\leq \sup_{u\in[0, \infty)}q\int_{(0, u]}d_\infty(f, g)d\overline{F}_{X_1}(x) \\
                                &\leq q\int_{(0, \infty)}d_\infty(f, g)d\overline{F}_{X_1}(x)\\
                                &= qd_\infty(f, g).
    \end{align*}
    Re"sitev $\theta = A\theta$ se tako izra"za kot limita $\theta = \lim_{n\to\infty}A^ng_0$ za 
    poljubno funkcijo $g_0\in B([0, \infty)).$ Naj bodo $\overline{X}_1, \overline{X}_2, \dots $ 
    neodvisne enako porazdeljene slu"cajne spremenljivke s porazdelitveno funkcijo $\overline{F}_{X_1}$, ki 
    jo raz"sirimo na celo realno os, tako da velja $\overline{F}_{X_1}(x) = 0$ za $x < 0$.
    Potem $A$ izrazimo kot
    \begin{equation*}
        Ag(u) = (1 - q)\mathbbm{1}(u\geq0) + q\E\left[g(u - \overline{X}_1)\right]
    \end{equation*}
    in z indukcijo poka"zemo, da za $g_0 = (1 - q)\mathbbm{1}(u\geq0)$ za vsak $n\in\N$ velja
    \begin{equation*}
        A^ng_0(u) = (1 - q)\sum_{k = 0}^nq^k\Prob\left(\overline{W}_k \leq u\right),
    \end{equation*}
    pri "cemer je $\overline{W}_k = \overline{X}_1 + \cdots + \overline{X}_k$ in $\overline{W}_0 = 0$.
    Enakost o"citno velja za $n=1$. Preostane "se indukcijski korak $n\mapsto n+1:$
    \begin{align*}
        A^{n+1}g_0(u) 
            &= 1 - q + q\int_{[0, u]}A^ng_0(u - x)d\overline{F}_{X_1}(x) \\
            &\stackrel{\text{\scalebox{0.8}{I.P.}}}{=} 1 - q + q(1 - q)\int_{[0, u]}\sum_{k = 0}^nq^k\int_{[0, u-x]}dF_{\overline{W}_k}(y)d\overline{F}_{X_1}(x) \\
            &=  (1 - q)\left[1 + \sum_{k=0}^nq^{k+1}\int_{[0, u]}\int_{[0, u-x]}dF_{\overline{W}_k}(y)d\overline{F}_{X_1}(x)\right] \\
            &= (1 - q)\left[1 + \sum_{k=0}^nq^{k+1}\Prob\left(\overline{W}_k \leq u - X_{k+1}, X_{k+1}\leq u\right)\right] \\
            &= (1 - q)\left[1 + \sum_{k=0}^nq^{k+1}\Prob\left(\overline{W}_{k+1}\leq u\right)\right] \\
            &= (1 - q)\left[1 + \sum_{k=1}^nq^k\Prob\left(\overline{W}_k\leq u\right)\right] \\
            &= (1 - q)\sum_{k=0}^nq^k\Prob\left(\overline{W}_k\leq u\right).
    \end{align*}
    Sledi 
    \begin{equation}
        \theta(u) = \lim_{n\to\infty} A^ng_0 = (1 - q)\sum_{k = 0}^\infty q^k\Prob\left(\overline{W}_k \leq u\right).
        \label{eq:bodocaGeometrijska}
    \end{equation}
    \label{op:OperatorPrenovitvena}
    \end{opomba}

    \begin{opomba}
    Verjetnost pre"zivetja zdaj spominja na porazdelitveno funkcijo neke slu"cajne spremenljivke s sestavljeno porazdelitvijo. 
    Definiramo slu"cajno spremenljivko $G$ za katero velja $G - 1\sim\text{Geom}(1 - q)$
    ($\Prob(G = k) = q^k(1 - q)$ za $k\in\N_0$) in je neodvisna od $\overline{X}_1, \overline{X}_2, \dots$
    Potem se izka"ze, da porazdelitvena funkcija slu"cajne spremenljivke $C := \overline{W}_G$ zado"s"ca ena"cbi (\ref{eq:bodocaGeometrijska}). 
    Za $u\geq0$ velja 
    \begin{align*}
        \Prob\left(C \leq u\right)
               &= \sum_{k = 0}^\infty\Prob\left(G = k\right)\Prob\left(\overline{W}_k \leq u\right) \nonumber\\
               &= (1 - q)\sum_{k = 0}^\infty q^k\Prob\left(\overline{W}_k \leq u\right) = \theta(u).
       \end{align*}
    Verjetnost pre"zivetja $\theta$ je torej porazdelitvena funkcija sestavljene geometrijske porazdelitve.
    \label{op:thetaKotGeometrijska}
    \end{opomba}
    
    \subsection{Lahkorepe porazdelitve}
        Pri analizi asimptotike verjetnosti propada se\newline bomo najprej omejili na primer, ko ima 
        porazdelitev slu"cajnih spremenljivk $X_i$, ki jih 
        se"stevamo v $\CPP$, lahek rep, saj je bila osnovna teorija, ki sta jo razvila H.\ Cramér in F.\ Lundberg,
        izpeljana pod to predpostavko.

        \begin{definicija}
            Pravimo, da ima slu"cajna spremenljivka $X$ \textit{lahkorepo porazdelitev}, "ce za 
            nek $\varepsilon > 0$ velja
        \begin{equation*}
            \E\left[e^{uX}\right] = M_X(u) < \infty \quad \text{za} \ u \in (-\varepsilon, \varepsilon).
        \end{equation*}
        Sicer pravimo, 
        da ima $X$ \textit{te"zkorepo porazdelitev}.
        \label{def:lahkorepnaPorazdelitev}
        \end{definicija}

        \begin{opomba}
            V razdelku ve"cinoma delamo z nenegativnimi slu"cajnimi spremenljivkami. Za te momentno-rodovna 
            funkcija vedno obstaja vsaj na intervalu $(-\infty, 0]$.
        \end{opomba}

        \begin{zgled}(Nadaljevanje zgleda \ref{zgd:weibullProcesTveganja})
            V zgledu smo obravnavali proces tveganja v Cramér--Lundbergovem 
            modelu, kjer so zahtevki (rde"ca) $X_i\sim\text{Weibull}(2, 434)$ in (modra) 
            $Y_i\sim\text{Weibull}(\tfrac{1}{4}, 16)$. Opazili smo, da je v prvem primeru propad
            posledica ve"c manj"sih izgub, v drugem pa ene velike izgube.\ To je zna"cilnost te"zkorepih
            porazdelitev. Za Weibullovo porazdelitev velja, da ima za parameter
            $a \geq 1$ lahek, za $a<1$ pa te"zak rep.
            \begin{proof}
                Momentno-rodovna funkcija $X\sim\text{Weibull}(a, b)$ je enaka
                \begin{align*}
                    M_X(u) &= \int_{0}^{\infty}e^{ux}\frac{a}{b}\left(\frac{x}{b}\right)^{a-1}e^{-\left(\frac{x}{b}\right)^a}dx \qquad \left(y = \tfrac{x}{b},\ dy = \tfrac{dx}{b}\right) \\
                           &= a\int_{0}^{\infty}e^{uby} y^{a-1}e^{-y^a}dy.
                \end{align*}
                Vidimo, da na spodnji meji $0$ ni te"zav za noben $a > 0$, medtem ko v neskon"cnosti 
                za $a\in(0, 1)$ integral divergira, saj se 
                eksponent poenostavi v $y^a(uby^{1 - a} - 1)\xrightarrow{y\to\infty}\infty$. "Ce v 
                nadaljevanju predpostavimo $a\geq 1$ in uvedemo $z = y^a$ 
                $\bigl(dz = ay^{a-1}dy\bigr)$ pa lahko pridemo do naslednje oblike 
                za momentno rodovno funkcijo $X$. \phantom{\qedhere}
                \begin{align*}
                    M_X(u) &= \int_{0}^{\infty}e^{ubz^{\frac{1}{a}}}e^{-z}dz \\
                           &= \int_{0}^{\infty}\sum_{k=0}^{\infty}\frac{(ubz^{\frac{1}{a}})^k}{k!}e^{-z}dz \qquad \qquad \text{Tonelli} \ \refPriloga{izr:TonellijevIzrek} \\
                           &= \sum_{k=0}^{\infty}\frac{(ub)^k}{k!}\int_{0}^{\infty}z^{\frac{k}{a}}e^{-z}dz \\
                           &= \sum_{k=0}^{\infty}\frac{(ub)^k}{k!} \, \Gamma \! \left(\frac{k}{a} + 1\right).
                \end{align*} 
            \end{proof}
            \label{zgd:weibullLahkorepnaPorazdelitev}
        \end{zgled}

            \begin{opomba}
                V praksi z lahkorepnimi porazdelitvami modeliramo zahtevke, kjer verjetnost ekstremnih 
                dogodkov (torej zelo velikih zahtevkov) eksponentno pada proti $0$. To neposredno sledi iz 
                definicije \ref{def:lahkorepnaPorazdelitev} in neenakosti Markova \refPriloga{trd:neenakostMarkova}, 
                saj za vsak 
                $x>0$ in $u\in(-\varepsilon, \varepsilon)$ velja
                \begin{equation*}
                    \Prob\left(X > x\right) = \Prob\left(e^{uX} > e^{ux}\right) \leq \frac{\E\left[e^{uX}\right]}{e^{ux}}.
                \end{equation*}
                \label{op:lahkorepnaPorazdelitev}
            \end{opomba}

            \begin{definicija}
                Naj velja, da ima slu"cajna spremenljivka $Y_1 = X_1 - cT_1$ iz trditve \ref{trd:verjetnostPropadaZOgrodjem} 
                lahek rep. "Ce obstaja enoli"cen $\ell > 0$, za katerega velja
                \begin{equation*}
                    M_{Y_1}(\ell)  = 1,
                \end{equation*}
                temu "stevilu pravimo \textit{Lundbergov koeficient}.
                \label{def:LundbergovKoeficient}
            \end{definicija}

            \begin{trditev}
                Br"z ko Lundbergov koeficient $\ell$ (pod predpostavkami definicije \ref{def:LundbergovKoeficient} in 
                pogoja NPC)
                obstaja, je enoli"cno dolo"cen.
                \label{trd:enolicnostLundbergovegaKoeficienta}
            \end{trditev}

            \begin{proof}
                Zaradi konveksnosti eksponentne funkcije je mno"zica $I = \{u\in\R \mid M_{Y_1}(u) < \infty\}$ konveksna, 
                torej je $I$ interval, poltrak ali kar cela realna os. Po predpostavki obstaja pozitiven $\ell \in I$ za 
                katerega vejla $M_{Y_1}(\ell) = 1$.
                Ker ima $Y_1$ lahek rep, obstaja $\varepsilon > 0$, da je $M_{Y_1}(u) < \infty$ za $u\in(-\varepsilon, \varepsilon)\subseteq I$.
                Ker velja $M_{Y_1}(0) = 1$ in $M_{Y_1}'(0) = \E\left[Y_1\right] < 0$ (zaradi pogoja NPC) ter
                $M_{Y_1}''(u) = \E\left[Y_1^2e^{Y_1u}\right] > 0$ $(Y_1 \neq 0 \ \text{skoraj gotovo})$ za 
                $u>0$, je $M_{Y_1}$ zvezna konveksna funkcija na $I$, kjer 
                v okolici ni"cle pada. Tako je $\ell$ zaradi konveksnosti $M_{Y_1}$ enoli"cno dolo"cen.
            \end{proof}

            \pagebreak

            \begin{izrek}(Lundbergova neenakost)
                Naj bo $(U_t)_{t\geq0}$ proces tveganja v Cramér--Lundbergovem modelu, ki zado"s"ca pogoju NPC in 
                 zanj obstaja Lundbergov koeficient $\ell$. Potem za vsak $u>0$ velja
                \begin{equation*}
                    \psi(u) \leq e^{-\ell u}.
                \end{equation*}
                \label{izr:LundbergovaNeenakost}
            \end{izrek}

            \begin{proof}
                Neenakost bomo dokazali z indukcijo. Za $u>0$ in $n\in\N$ definiramo
                \begin{equation*}
                    \psi_n(u) = \Prob\left(\max_{1\leq k\leq n}Z_k > u\right),
                \end{equation*}
                kjer je $Z_k = \sum_{i=1}^kY_i$ kumulativna izguba po $k$ zahtevkih enako definirana 
                kot v trditvi \ref{trd:verjetnostPropadaZOgrodjem}.
                Vidimo, da je (po zveznosti $\mathbb{P}$ od spodaj) $\psi(u) = \lim_{n\to\infty}\psi_n(u)$, 
                torej moramo pokazati, da za vsak $n\in\N$ velja $\psi_n(u) \leq e^{-\ell u}$. \\
                $(n = 1):$ Uporabimo neenakost Markova in dobimo
                \begin{equation*}
                    \psi_1(u) = \Prob\left(e^{\ell Z_1} > e^{\ell u}\right) \leq \frac{M_{Z_1}(\ell)}{e^{\ell u}} = e^{-\ell u}.
                \end{equation*}
                $(n \rightarrow n+1):$ 
                S $F_{Y_1}$ ozna"cimo porazdelitveno funkcijo slu"cajne spremenljivke
                 $Y_1$. Potem velja
                \begin{align*}
                    \psi_{n+1}(u) &= \Prob\left(\max_{1\leq k\leq n+1}Z_k > u\right) \\
                                  &= \underbrace{\Prob\left(Y_1 > u\right)}_{(i)} + 
                                  \underbrace{\Prob\left(\max_{2\leq k\leq n+1}\bigl\{Y_1 + (Z_k - Y_1)\bigr\} > u, Y_1 \leq u\right)}_{(ii)} \\
                \end{align*}
                Najprej se posvetimo $(ii)$. Po indukcijski predpostavki velja 
                \begin{align*}
                    (ii) &= \int_{(-\infty, u]}\Prob\left(\max_{1\leq k\leq n}\bigl\{x + Z_k\bigr\} > u\right)dF_{Y_1}(x) \\
                         &= \int_{(-\infty, u]}\Prob\left(\max_{1\leq k\leq n}Z_k > u - x\right)dF_{Y_1}(x) \\
                         &= \int_{(-\infty, u]}\psi_n(u - x)dF_{Y_1}(x) \\
                         &\stackrel{\text{\scalebox{0.8}{I.P.}}}{\leq} \int_{(-\infty, u]}e^{-\ell(u - x)}dF_{Y_1}(x). \\
                \end{align*}
                Za oceno $(i)$ kot v primeru $n=1$ uporabimo neenakost Markova in dobimo

                \begin{equation*}
                    (i) = \psi_1(u) \leq \frac{M_{Z_1}(\ell)}{e^{\ell u}} = \int_{(u, \infty)}e^{-\ell (u-x)}dF_{Y_1}(x).
                \end{equation*}

                \pagebreak 
                \noindent
                "Ce torej se"stejemo $(i)$ in $(ii)$ dobimo "zeleno oceno

                \begin{align*}
                    \psi_{n+1}(u) &\leq \int_{\R}e^{-\ell (u - x)}dF_{Y_1}(x) \\
                                  &= e^{-\ell u}M_{Y_1}(\ell) \\
                                  &= e^{-\ell u}.
                \end{align*}

            \end{proof}

            \begin{opomba}
                Iz izreka \ref{izr:LundbergovaNeenakost} je razvidno, da lahko v praksi z
                 dovolj visokim za"cetnim kapitalom
                $u$ verjetnost propada zadovoljivo omejimo blizu $0$. Seveda je meja 
                odvisna tudi od Lundbergovega koeficienta $\ell$.
                \label{op:LundbergovaNeenakost}
            \end{opomba}

            \begin{zgled}
                Naj bo $(U_t)_{t\geq0}$ proces tveganja v Cramér--Lundbergovem modelu, ki zado"s"ca NPC.\ Naj 
                nadalje velja, da so zahtevki neodvisno eksponentno porazdeljeni s parametrom $\mu > 0$. 
                Vemo, da ima momentno rodovna funkcija $X_1$ obliko 

                \begin{equation*}
                    M_{X_1}(u) = \frac{\mu}{\mu - u} \quad \text{za} \ u<\mu.
                \end{equation*}
                Tako dobimo, da ima momentno rodovna funkcija $Y_1 = X_1 - cT_1$ obliko 

                \begin{equation*}
                    M_{Y_1}(u) = M_{X_1}(u)M_{T_1}(-cu) = 
                    \frac{\mu}{\mu - u}\frac{\lambda}{\lambda + cu} \quad \text{za} \ u\in (-\tfrac{\lambda}{c}, \mu).
                \end{equation*}
                Sedaj lahko izra"cunamo Lundbergov koeficient $\ell$

                \begin{align*}
                    M_{Y_1}(\ell) &= 1, \\
                    \frac{\mu}{\mu - \ell}\frac{\lambda}{\lambda + c\ell} &= 1, \\
                    \mu\lambda &= (\mu - \ell)(\lambda + c\ell), \\
                    \mu\lambda &= \mu\lambda - \ell\lambda + \mu c - c\ell^2, \\
                    0 &= \mu c - c\ell - \lambda.
                \end{align*}
                Dobimo 
                \begin{equation}
                    \ell = \mu - \frac{\lambda}{c}.
                    \label{eq:LundbergovKoeficientExpPrva}
                \end{equation}
                Velja $\ell \in (0, \mu)$, saj v na"sem modelu velja pogoj NPC,
                \begin{equation*}
                    \frac{\E\left[X_1\right]}{\E\left[T_1\right]} = \frac{\lambda}{\mu} < c \iff \mu > \frac{\lambda}{c}.
                \end{equation*}
                "Ce uporabimo alternativno formulacijo pogoja NPC, dobimo
                \begin{equation}
                    c = (1 + \rho)\frac{\lambda}{\mu} \quad \Rightarrow \quad
                    \ell = \mu - \frac{\lambda}{(1 + \rho)\frac{\lambda}{\mu}} = \mu\left(\frac{\rho}{1 + \rho}\right).
                    \label{eq:LundbergovKoeficientExp}
                \end{equation}
                Tako dobimo zgornjo mejo za verjetnost propada
                \begin{equation*}
                    \psi(u) \leq e^{-\ell u} = e^{-\mu u\left(\frac{\rho}{1 + \rho}\right)}
                \end{equation*}
                in vidimo, da pove"canje stopnje prihodkov premij "cez neko mejo ne bistveno 
                vpliva na oceno, saj je
                \begin{equation*}
                    \lim_{\rho\to\infty}e^{-\mu u\left(\frac{\rho}{1 + \rho}\right)} = e^{-\mu u}.
                \end{equation*}
                V nadaljevanju bomo videli, da je Lundbergova neenakost v primeru eksponentno 
                porazdeljenih zahtevkov skoraj to"cna vrednost verjetnosti propada, zgre"sena le za konstanto.
                V splo"snem pa je zelo te"zko dolo"citi Lundbergov koeficient kot funkcijo parametrov
                porazdelitev $X_1$ in $T_1$ in zato uporabljamo numeri"cne metode za njegovo aproksimacijo 
                kot na primer Monte Carlo simulacije. 
                \label{zgd:LundebrgovaNeenakostEksponentno}
            \end{zgled}

            Sedaj izpeljemo enega temeljnih rezultatov v teoriji tveganja, ki ga je leta 1930 dokazal Harald 
            Cramér. Za lahkorepe porazdleitve bomo dolo"cili to"cen red konvergence verjetnosti propada.

            \begin{izrek}(Asimptotika verjetnosti propada, lahkorepe porazdelitve)
                Naj bo $(U_t)_{t\geq0}$ proces tveganja v Cramér--Lundbergovem modelu, ki zado"s"ca NPC in 
                naj zanj obstaja Lundbergov koeficient $\ell$. Naj imajo slu"cajne spremenljivke 
                $(X_i)_{i\in\N}$ gostoto.\ Potem obstaja konstanta $C>0$, za katero velja 
                \begin{equation*}
                    \lim_{u\to\infty}e^{\ell u}\psi(u) = C.
                \end{equation*}
                \label{izr:CramerjevaMeja}
            \end{izrek}

            \begin{proof} 
                Najprej preoblikujemo ena"cbo (\ref{eq:verjetnostPrezivetja3}), tako da 
                upo"stevamo $\theta = 1 - \psi$

                \begin{align}
                    1 - \psi(u) &= (1 - q) + q\int_{(0, u]}\bigl(1 - \psi(u - x)\bigr)d\overline{F}_{X_1}(x), \nonumber \\
                    \psi(u) &= q\bigl(1 - \overline{F}_{X_1}(u)\bigr) + \int_{(0, u]}\psi(u - x)d\left(q\overline{F}_{X_1}(x)\right). \label{eq:verjetnostPropadaQ}
                \end{align}
                Vidimo, da je ena"cba (\ref{eq:verjetnostPropadaQ}) "se vedno defektna prenovitvena ena"cba.\ 
                Lahko bi jo analizriali kot v opombi \ref{op:OperatorPrenovitvena}, a izka"ze se, da
                izra"zava re"sitve, ki jo dobimo iz Banachovega izreka o negibni to"cki, za "studij 
                asimptotike ni najbolj"sa, "ceprav je dokaj eksplicitna. 
                V primeru lahkorepih porazdelitev se izka"ze, da lahko ena"cbo (\ref{eq:verjetnostPropadaQ}) s trikom 
                prevedemo na pravo prenovitveno ena"cbo. Za $x > 0$ 
                definiramo funkcijo $F_\ell$ kot Esscherjevo transformacijo funkcije $q\overline{F}_{X_1}$.
                \begin{equation}
                    F_\ell(x) = \int_{(0, x]}e^{\ell y}d\bigl(q\overline{F}_{X_1}(y)\bigr) = \frac{q}{\E\left[X_1\right]}\int_0^xe^{\ell y}\bigl(1 - F_{X_1}(y)\bigr)dy, 
                    \label{eq:EsscherjevaTransformacija}
                \end{equation}
                Poka"zimo, da je $F_\ell$ porazdelitvena funkcija. O"citno je nara"s"cajo"ca in velja
                \begin{align*}
                    \lim_{x\to\infty}F_\ell(x) &= \frac{q}{\E\left[X_1\right]}\int_0^\infty e^{\ell y} \bigl(1 - F_{X_1}(y)\bigr)dy \qquad \bigl(\alpha = 1 - F_{X_1}(y), \ d\beta = e^{\ell y}dy\bigr)\\
                                               &= \frac{q}{\E\left[X_1\right]}\biggl(\biggl(\frac{\bigl(1 - F_{X_1}(y)\bigr)e^{\ell y}}{\ell}\biggr)\Big|_{0}^{\infty} + \frac{1}{\ell}\int_0^\infty e^{\ell y}f_{X_1}(y)dy\biggr) \\
                                               &= \frac{q}{\E\left[X_1\right]}\frac{1}{\ell}\biggl(\E\left[e^{\ell X_1}\right] - 1\biggr).
                \end{align*}
                Sedaj upo"stevamo, da je $q = \frac{1}{1 + \rho} = \frac{\E\left[X_1\right]}{c\E\left[T_1\right]}$, 
                definicijo Lundbergovega koeficienta ter dejstvo, da je $T_1\sim\text{Exp}(\lambda)$ medprihodni "cas v $\text{HPP}(\lambda)$, da dobimo
                \begin{align*}
                    \lim_{x\to\infty}F_\ell(x)  &= \frac{\E\left[e^{\ell X_1}\right] - 1}{c\,\ell\ \E\left[T_1\right]}\\
                                                &= \frac{\frac{\lambda + c\,\ell}{\lambda} - 1}{c\,\ell \frac{1}{\lambda}} = 1.
                \end{align*}
                "Ce torej ena"cbo (\ref{eq:verjetnostPropadaQ}) pomno"zimo z $e^{\ell u}$, dobimo
                \begin{align}
                    e^{\ell u}\psi(u)   &= qe^{\ell u}\bigl(1 - \overline{F}_{X_1}(u)\bigr) + \int_{(0, u]}e^{\ell (u - x)}\psi(u - x)e^{\ell x}d\bigl(q\overline{F}_{X_1}(x)\bigr) \nonumber \\
                                        &= qe^{\ell u}\bigl(1 - \overline{F}_{X_1}(u)\bigr) + \int_{(0, u]}e^{\ell (u - x)}\psi(u - x)dF_\ell(x). \label{eq:CramérjevaPrenovitvenaEnacba}
                \end{align}
                Vidimo, da sedaj ena"cba (\ref{eq:CramérjevaPrenovitvenaEnacba}) ustreza obliki prenovitvene ena"cbe za par \newline $\left(qe^{\ell u}\bigl(1 - \overline{F}_{X_1}(u)\bigr), F_\ell\right)$
                in ker je funkcija $u\mapsto qe^{\ell u}\bigl(1 - \overline{F}_{X_1}(u)\bigr)$ omejena na kon"cnih 
                intervalih in $F_\ell$ nearitmeti"cna, lahko uporabimo Smithov klju"cni prenovitveni izrek \refPriloga{izr:Smith}, da dobimo
                re"sitev
                \begin{equation}
                    e^{\ell u}\psi(u) =qe^{\ell u}\bigl(1 - \overline{F}_{X_1}(u)\bigr) +  q\int_{(0, u]}e^{\ell(u - x)}\bigl(1 - \overline{F}_{X_1}(u - x)\bigr)dM^{\ell}(x),
                    \label{eq:resitevCramerjevePrenovitveneEnacbe}
                \end{equation}
                kjer je $M^{\ell}$ prenovitvena mera prenovitvenega procesa z medprihodnimi "casi, 
                ki imajo porazdelitveno funkcijo $F_\ell$. V splo"snem $M^{\ell}$ te"zko dolo"cimo. 
                "Ce pa je funkcija $u\mapsto qe^{\ell u}(1 - \overline{F}_{X_1}(u))$ direktno Riemannovo integrabilna
                \refPriloga{def:direktnaRieamnovaIntegrabilnost}, nam 
                Smithov izrek da asimptoti"cno vedenje re"sitve (\ref{eq:resitevCramerjevePrenovitveneEnacbe}),
                ko gre $u\to\infty$.
                Ozna"cimo 
                \begin{equation*}
                    f(u) := qe^{\ell u}(1 - \overline{F}_{X_1}(u)).
                \end{equation*}
                Direktno Riemannovo integrabilnost $f$ bomo pokazali z uporabo trditve \refPriloga{trd:kriterijZaDirektnoRiemannovoIntegrabilnost}. 
                Pokazati moramo, da je $f$ posplo"seno Riemannovo integrabilna in da ima omejeno totalno variacijo \refPriloga{def:totVariacija}. 
                %$\lim_{u\to\infty}f(u) = 0$ ter da obstaja Riemannov integral 
                %$\int_0^\infty f(u)du$. Za ta namen ocenimo funkcijo $f$
                %\begin{equation*}
                %    f(u) = \frac{e^{\ell u}}{\E\left[X_1\right]}\int_{(u, \infty)}dF_{X_1}(x) \leq \frac{1}{\E\left[X_1\right]}\int_{(u, \infty)}e^{\ell x}dF_{X_1}(x).
                %\end{equation*}
                %Ker velja 
                %\begin{equation*}
                %    \int_{(0, \infty)}e^{\ell x}dF_{X_1}(x) = \E\left[e^{\ell X_1}\right] < \infty, 
                %\end{equation*}
                %mora biti $\lim_{u\to\infty}f(u) = 0$.
                \noindent
                Da obstaja posplo"seni Riemannov integral $\int_0^\infty f(u)du$, poka"zemo prek Lebesgueovega integrala. 
                Po Tonellijevem izreku \refPriloga{izr:TonellijevIzrek} velja
                \begin{align*}
                    \int_{(0, \infty)}f(x)dx &= \frac{1}{\E\left[X_1\right]}\int_{(0, \infty)}e^{\ell u}\int_{(u, \infty)}dF_{X_1}(x)\,du \\
                    &= \frac{1}{\E\left[X_1\right]}\int_{(0, \infty)}\int_0^xe^{\ell u}du\, dF_{X_1}(x) \\
                    &= \frac{1}{\ell\, \E\left[X_1\right]} \int_{(0, \infty)}\left(e^{\ell x} - 1\right)dF_{X_1}(x) \\
                    &= \frac{\E\left[e^{\ell X_1} - 1\right]}{\E\left[X_1\right]} < \infty.
                \end{align*}
                Ker je $f$ zvezna ($\overline{F}_{X_1}$ je zvezna) mora obstajati tudi posplo"seni Riemannov integral $\int_0^\infty f(u)du$.
                Da poka"zemo, da ima $f$ omejeno totalno variacijo, jo izrazimo kot razliko 
                dveh nenara"s"cajo"cih funkcij. Ponovno uporabimo Tonellijev izrek, da poka"zemo naslednjo enakost
                \begin{align*}
                    \int_u^\infty f(x) dx 
                    &= \int_{[u, \infty)}e^{\ell x}\int_{(x, \infty)}d\overline{F}_{X_1}(y)\,dx \\
                    &= \int_{[u, \infty)}\int_{(u, y)}e^{\ell x}dx\,d\overline{F}_{X_1}(y) \\
                    &= \frac{1}{\ell}\int_{[u, \infty)}\left(e^{\ell y} - e^{\ell u}\right)d\overline{F}_{X_1}(y) \\
                    &= \frac{1}{\ell}\left[\frac{1}{\E\left[X_1\right]}\int_u^\infty e^{\ell y}\left(1 - F_{X_1}(y)\right)dy - f(u)\right].
                \end{align*}

                \pagebreak
                \noindent
                Enakost velja, ker obstaja $\int_0^\infty f(x)dx$.
                Z malo preurejanja dobimo 
                \begin{equation*}
                    f(u) = \underbrace{\frac{1}{\ell\,\E\left[X_1\right]}\int_u^\infty e^{\ell y}\left(1 - F_{X_1}(y)\right)dy}_{g(u)} - \underbrace{\ell\int_u^\infty f(x) dx}_{h(u)}. 
                \end{equation*}
                Tako izrazimo $f$ kot razliko funkcij $g$ in $h$, ki sta o"citno nenara"s"cajo"ci in ker velja 
                $\lim_{u\to\infty}g(u) = \lim_{u\to\infty}h(u) = 0$, imata omejeno totalno variacijo.
                Funkcija $f$ torej zado"s"ca predpostavkam trditve \refPriloga{trd:kriterijZaDirektnoRiemannovoIntegrabilnost} in 
                je direktno Riemannovo integrabilna. 
                Tako po Smithovem klju"cnem prenovitvenem izreku velja
                \begin{equation}
                    C = \lim_{u\to\infty}e^{\ell u}\psi(u) =  \frac{q}{\alpha} \int_0^\infty e^{\ell x}(1 - \overline{F}_{X_1}(x))dx,
                    \label{eq:CramerBoundConstant}
                \end{equation}
                kjer je $\alpha = \int_{(0, \infty)}x dF_\ell(x)$. S tem je izrek dokazan.
            \end{proof}

            \begin{zgled}(Nadaljevanje zgleda \ref{zgd:LundebrgovaNeenakostEksponentno})
                Vemo, da re"sitve 
                prenovitvene ena"cbe (\ref{eq:resitevCramerjevePrenovitveneEnacbe}) iz izreka \ref{izr:CramerjevaMeja}
                v splo"snem ne moremo eksplicitno izra"cunati.
                V zgledu \ref{zgd:LundebrgovaNeenakostEksponentno} pa smo privzeli, da zahtevke modeliramo 
                z eksponentno porazdelitvijo, torej $X_i\sim\text{Exp}(\mu)$. 
                V tem primeru se izka"ze, da lahko verjetnost propada eksplicitno izra"cunamo.
                "Ce si pogledamo ena"cbo
                (\ref{eq:resitevCramerjevePrenovitveneEnacbe}), vidimo, da moramo 
                izra"cunati le porazdelitev integriranega repa $\overline{F}_{X_1}(u)$ in 
                prenovitveno mero Esscherjeve transformacije $F_\ell$. Za eksponentno porazdelitev
                velja
                \begin{align*}
                    \overline{F}_{X_1}(u)   &= \frac{1}{\E\left[X_1\right]}\int_0^u\bigl(1 - F_{X_1}(t)\bigr)dt \\
                                            &= \mu\int_0^ue^{-\mu t}dt \\
                                            &= F_{X_1}(u),
                \end{align*}
                saj je pozabljiva. Prenovitveno mero Esscherjeve transformacije pa dobimo tako, 
                da najprej izra"cunamo porazdelitveno funkcijo $F_\ell$ podano v ena"cbi (\ref{eq:EsscherjevaTransformacija}). 

                \begin{align*}
                    F_\ell(u)  &= \frac{q}{\E\left[X_1\right]}\int_0^ue^{\ell x}\bigl(1 - F_{X_1}(x)\bigr)dx \\
                            &= \frac{\mu}{1 + \rho}\int_0^ue^{-x(\mu - \ell)}dx. \\
                \end{align*}
                Upo"stevamo rezultat (\ref{eq:LundbergovKoeficientExp}), torej 
                $\ell = \mu\left(\tfrac{\rho}{1 - \rho}\right)$ in vidimo, da je $F_\ell$ porazdelitvena 
                funkcija eksponentne slu"cajne spremenljivke s parametrom $\frac{\mu}{1 + \rho}$ oziroma 
                $\mu q$. Torej je prenovitvena mera $M^\ell(t)$ preprosto pri"cakovano "stevilo prihodov do 
                "casa $t$ v $\text{HPP}(\mu q)$, torej $M^\ell(t) = \mu qt$.
                "Ce vstavimo rezultata v ena"cbo (\ref{eq:resitevCramerjevePrenovitveneEnacbe}), dobimo
                \begin{align*}
                    e^{\ell u}\psi(u)   &= qe^{\ell u}e^{-\mu u} + q\int_{(0, u]}e^{\ell(u - x)}e^{-\mu(u - x)}dM^\ell(x)\\
                                        &= qe^{-u(\mu - \ell)} + \mu q^2\int_0^u e^{-(\mu - \ell)(u - x)}dx \\
                                        &= qe^{-u(\mu - \ell)} + \mu q^2 \frac{1}{\mu - \ell}\biggl(1 - e^{-u(\mu - \ell)}\biggr)\\
                                        &= qe^{-u(\mu - \ell)} + \frac{\mu}{(1 + \rho)^2}\frac{1 + \rho}{\mu}\biggl(1 - e^{-u(\mu - \ell)}\biggr) \\
                                        &= qe^{-u(\mu - \ell)} + q\biggl(1 - e^{-u(\mu - \ell)}\biggr) \\
                                        &= q = \frac{1}{1 + \rho}.
                \end{align*}
                Kon"cno dobimo, da je verjetnost propada enaka
                \begin{equation}
                    \psi(u) =  \frac{1}{1+\rho}\,e^{-\ell u}.
                \label{eq:eksplicitnaVerjetnostPropadaExp}
                \end{equation}
            \end{zgled}

                Vidimo, da se $\psi(u)$ od ocene, ki jo dobimo z Lundbergovo neenakostjo v zgledu \ref{zgd:LundebrgovaNeenakostEksponentno}, 
                res razlikuje le za konstanto $\tfrac{1}{1 + \rho}$.
                To je seveda zelo poseben primer, ko lahko vse koli"cine izra"cunamo eksplicitno.
                Poka"zimo, kako bi do približkov funkcije $\psi(u)$ v praksi lahko pri"sli z Monte 
                Carlo simulacijami.
            
            \begin{zgled}
                Kot v zgledu \ref{zgd:LundebrgovaNeenakostEksponentno} predpostavimo, da so 
                zahtevki porazdeljeni eksponentno, torej $X_i\sim\text{Exp}(\mu)$. Recimo, da je intenzivnost 
                prihodov zahtevkov $\lambda = 1$, stopnja prihodkov premij $c = 1500$ in 
                pri"cakovana vrednost zahtevkov $1000$ \euro, torej 
                $\mu = \tfrac{1}{1000}$. Potem lahko verjetnost propada eksplicitno izra"cunamo po 
                formuli (\ref{eq:eksplicitnaVerjetnostPropadaExp}). Prvo izra"cuamo $\rho$ po
                (\ref{eq:LundbergovKoeficientExp}), in $\ell$ po (\ref{eq:LundbergovKoeficientExpPrva}), torej
                \begin{align*}
                    \rho &= \frac{c\mu}{\lambda} - 1\\
                         &= \frac{1500\cdot\tfrac{1}{1000}}{1} - 1 \\
                         &= \frac{1}{2}, \\
                    \ell &= \mu - \frac{\lambda}{c} \\
                         &= \frac{1}{1000} - \frac{1}{1500}\\
                         &=  \frac{1}{3000}.
                \end{align*}
                Vstavimo vrednosti v (\ref{eq:eksplicitnaVerjetnostPropadaExp}) in dobimo
                \begin{equation*}
                    \psi(u) = \tfrac{2}{3}e^{-\frac{u}{3000}}.
                \end{equation*}
                Sedaj definiramo zaporedje $(u_n)_{n = 1}^{50}$ s predpisom $u_n = 500n$ in za vsak $n$
                simuliramo $10, 50$ in $100$ neodvisnih realizacij procesa tveganja, bodisi do "casa $T = 1000$ bodisi dokler
                ne propade in za vsak $n$ izra"cunamo pribli"zek za verjetnost propada kot dele"z propadlih 
                realizacij z vsemi. Aproksimacijo $\psi(u)$ prika"zemo na sliki \ref{fig:slika4}.

                \newpage
                \begin{figure}[H]
                    \centering
                    \includegraphics[width=\textwidth]{
                        C:/Users/38651/OneDrive - Univerza v Ljubljani/Desktop/Diploma/Diplomski-seminar/GraphsAndPhotos/slika3.pdf
                        }
                    \caption{Aproksimacija verjetnosti propada $\psi(u)$ z Monte Carlo simulacijami.}
                    \label{fig:slika4}
                \end{figure}

                \noindent
                Kot vidimo, se pribli"zki z nara"s"cajo"cim "stevilom simulacij pribli"zujejo funkciji $\psi(u)$, a za 
                res dobro aproksimacijo bi morali to "stevilo krepko pove"cati, saj na primer
                za vrednost $u = 16000$ pride $\psi(16000) \approx 0{,}0032186334$, kar je pribli"zno
                $0.3\%$ in v praksi ni zanemarljivo. Vendar pa v na"si simulaciji nobena realizacija procesa 
                tveganja ni padla pod 0.


                %Definiramo "se zaporedje $(\hat{u}_n)_{n = 1}^{25}$ s predpisom $\hat{u}_n = 5000n$ in za vsak $n$
                %izra"cunamo produkt $\psi(\hat{u}_n)e^{\ell \hat{u}_n}$, kjer do ocene za $\psi(\hat{u}_n)$
                %pridemo na enak na"cin kot zgoraj le, da tokrat za posamezno oceno $\psi(\hat{u}_n)$ izvedemo 
                %1000 simulacij, saj je verjetnost propada zelo nizka. Po formuli 
                %(\ref{eq:CramerBoundConstant}) dobimo, da mora $\psi(\hat{u}_n)e^{\ell \hat{u}_n}$ 
                %konvergirati proti
 %
                %\begin{align*}
                %    C       &= \frac{1}{\mu} \int_{(0, \infty)}e^{\ell x}(1 - \overline{F}_{X_1}(x))dx \\
                %            &= 1000 \int_{(0, \infty)}e^{\frac{x}{3000}}\left(1 - (1  - e^{\frac{x}{1000}})\right)dx \\
                %            &= \frac{4}{9000}  \int_{(0, \infty)}e^{-\frac{2x}{3000}}dx \\
                %            &= \frac{2}{3},
                %\end{align*}
                %kar je razvidno, "ze "ce pogledamo (\ref{eq:eksplicitnaVerjetnostPropadaExp}).
                %Rezultate prika"zemo na sliki \ref{fig:slika5}.
%
                %\begin{figure}[H]
                %    \centering
                %    \includegraphics[width=\textwidth]{
                %        C:/Users/38651/OneDrive - Univerza v Ljubljani/Desktop/Diploma/Diplomski-seminar/GraphsAndPhotos/slika4.pdf
                %        }
                %    \caption{Aproksimacija verjetnosti propada $\psi(u)$ z Monte Carlo simulacijami}
                %    \label{fig:slika5}
                %\end{figure}
                %\noindent
                %Vidimo da i$\psi(\hat{u}_n)e^{\ell \hat{u}_n}$ res konvergira k $C = \frac{2}{3}$.
                \label{zgd:MonteCarlo}
            \end{zgled}


        
    \subsection{Te"zkorepe porazdelitve}
        Rezultati, ki smo jih izpeljali v prej"snjem razdelku, temeljijo na predpostavki zahtevkov
        z lahkorepimi porazdelitvami, kar interpretiramo, kot da je verjetnost zahtevkov, ki zelo 
        odstopajo od povpre"cja, zelo majhna. V praksi pa se pogosto zgodi, da ta predpostavka ni 
        izpolnjena in pojavi se vpra"sanje, ali lahko "se vedno kaj povemo o asimptotiki verjetnosti 
        propada. Izka"ze se, da v primeru, ko je porazdelitev 
        integriranega repa zahtevkov subeksponentna, ta to"cno dolo"ca asimptoti"cno vedenje verjetnosti 
        propada. 

        Subeksponentne porazdelitve so poseben razred te"zkorepih porazdelitev, ki ga lahko definiramo na 
        ve"c na"cinov. Za na"se namene bo najbolj uporabna naslednja definicija iz \cite{9}.
        \begin{definicija}
            Verjetnostna porazdelitev $F$ na $[0, \infty)$ je \textit{subeksponentna}, "ce za vsak $n\geq2$ in 
            neodvisne slu"cajne spremenljivke $X_1, \dots, X_n$ s to porazdelitvijo velja 
            \begin{equation*}
                \lim_{x\to\infty}\frac{\Prob\left(X_1 + \cdots + X_n > x\right)}{\Prob\left(X_1 > x\right)} = n
            \end{equation*}
            in $F(x) < 1$ za vsak $x > 0$.
            \label{def:subeksponentnaPorazdelitev}
        \end{definicija}

        \begin{opomba}
            Ekvivalentna in bolj intuitivna definicija subeksponentne porazdelitve je, da velja 
            \begin{equation*}
                \lim_{x\to\infty}\frac{\Prob\left(X_1 + \cdots + X_n > x\right)}{\Prob\left(\max\{X_1, \dots, X_n\} > x\right)} = 1 \quad \text{za vsak} \quad n\geq2, 
            \end{equation*}
            kar pomeni, da je repna porazdelitev vsote $n$ slu"cajnih spremenljivk asimptoti"cno
            primerljiva s porazdelitvijo najve"cje. Dokaz ekvivalence lahko bralec najde v \cite{9} na strani $437$.
        \end{opomba}

        \begin{lema}
            Naj bo $F_{X_1}$ subeksponentna porazdelitvena funkcija nenegativne slu"cajne spremenljivke 
            $X_1$. Potem za vsak $\varepsilon > 0$ obstaja pozitivna konstanta $K(\varepsilon) < \infty$, 
            da za vsak $n\geq2$ in $x \geq 0$ velja 
            \begin{equation*}
                \frac{\Prob(X_1 + \cdots + X_n > x)}{\Prob(X_1 > x)} \leq K(\varepsilon)(1+\varepsilon)^n. 
            \end{equation*} 
            \label{lema:ocenaSubeksponentnePorazdelitve}
        \end{lema}

        \begin{proof}
            Fiksiramo $n\geq2$ in $x\geq0$. Ulomek preoblikujemo
            \begin{align*}
                \frac{\Prob(X_1 + \cdots + X_n > x)}{\Prob(X_1 > x)} 
                &= \frac{(1 - F_{X_1}^{*n}(x)) + F_{X_1}(x) - F_{X_1}(x)}{1 - F_{X_1}(x)} \\
                &= 1 + \frac{F_{X_1}(x) - F_{X_1}^{*n}(x)}{1 - F_{X_1}(x)}. \\
            \end{align*}
            Naj bo $a_n = \sup_{x\geq0}\biggl\{ 1 + \frac{F_{X_1}(x) - F_{X_1}^{*n}(x)}{1 - F_{X_1}(x)}\biggr\}.$ Za 
            konstanto $M\in(0, \infty)$ lahko ocenimo
            \begin{align*}
                a_{n+1} &\leq 1 + \sup_{0\leq x\leq M}\int_{(0,x]}\frac{1 - F_{X_1}^{*n}(x - y)}{1 - F_{X_1}(x)}\, dF_{X_1}(y) \\
                            & \quad + \sup_{x\geq M}\int_{(0, x]}\frac{1 - F_{X_1}^{*n}(x - y)}{1 - F_{X_1}(x - y)}\frac{1 - F_{X_1}(x - y)}{1 - F_{X_1}}(x) \, dF_{X_1}(y) \\
                        &\leq 1 + T + a_n\sup_{x\geq M}\frac{F_{X_1}(x) - F_{X_1}^{*2}(x)}{1 - F_{X_1}(x)},
            \end{align*}
            kjer je $T = (1 - F_{X_1})^{-1}(M) < \infty$. Ker je $F_{X_1}$ subeksponentna, lahko 
            za vsak $\varepsilon > 0$ izberemo tak $M$, da velja
            \begin{align*}
                a_{n+1} \leq 1 + T + a_n(1 + \varepsilon)
            \end{align*}
            in tako dobimo "zeleno oceno
            \begin{equation*}
                a_n \leq \underbrace{(1 + T)\frac{1}{\varepsilon}}_{K(\varepsilon)}(1 + \varepsilon)^n.
            \end{equation*}
        \end{proof}
        
        \begin{izrek}(Asimptotika verjetnosti propada, subeksponentne porazdelitve)
            Naj bo $(U_t)_{t\geq0}$ proces tveganja v Cramér--Lundbergovem modelu, ki zado"s"ca NPC in 
            naj bodo zahtevki $(X_i)_{i\in\N}$ neodvisni in enako porazdeljeni z gostoto, 
            pri"cakovano vrednostjo\newline $\E\left[X\right] < \infty$ in naj bo $\overline{F}_{X_1}$ subeksponentna.
            Potem za verjetnost propada $\psi(u)$ velja
            \begin{equation}
                \lim_{u\to\infty}\frac{\psi(u)}{1 - \overline{F}_{X_1}(u)} = \frac{1}{\rho}.
                \label{eq:tezkorepnePorazdelitveAsimptotika}
            \end{equation}
            \label{izr:tezkorepnePorazdelitveAsimptotika}
        \end{izrek}

        \begin{proof}
            V opombi \ref{op:OperatorPrenovitvena} smo izrazili verjetnost pre"zivetja $\theta$ kot negibno to"cko 
            operatorja $A$ in v opombi \ref{op:thetaKotGeometrijska} pokazali, da je $\theta$ porazdelitvena
            funkcija sestavljene geometrijske porazdelitve. Torej lahko verjetnost propada zapi"semo kot 
            \begin{align*}
                \psi(u) = 1 - \theta(u) = \Prob(C > u) = (1 - q)\sum_{k=1}^\infty q^k\,\Prob(\overline{W}_k > u).
            \end{align*}

%            Najprej poka"zimo, da lahko verjetnost 
%            pre"zivetja iz leme \ref{lema:verjetnostPrezivetja} 
%            predstavimo kot porazdelitveno funkcijo sestavljene
%            geometrijske porazdelitve \refPriloga{def:CompoundGeometricDistribution}. "Ce definiramo 
%            $G \sim \text{Geom}(\frac{\rho}{1 + \rho})$ in zaporedje neodvisnih enako porazdeljenih 
%            slu"cajnih spremenljivk $(\overline{X}_i)_{i\in\N}$ s porazdelitveno funkcijo $\overline{F}_{X_1}$, 
%            bomo pokazali, da porazdelitvena funkcija $F_C$ slu"cajne spremenljivke
%            $C = \sum_{i=1}^{G}\overline{X}_i$ zado"s"ca ena"cbi 
%            \begin{equation}
%                F_C(u) = \frac{\rho}{1 + \rho} + \frac{1}{1 + \rho}\int_{(0, u]}F_C(u - x)d\overline{F}_{X_1}(x),
%                \label{eq:verjetnostPrezivetja4}
%            \end{equation}
%            ki je natanko ena"cba (\ref{eq:verjetnostPrezivetja3}) iz leme \ref{lema:verjetnostPrezivetja}.
%            Porazdelitvena funkcija $F_C$ ima obliko
%            \begin{align}
%                \Prob\left(C \leq u\right)  &= \Prob\left(G = 0\right) + \sum_{n = 1}^\infty\Prob\left(G = n\right)\Prob\left(\overline{X}_1 + \cdots + \overline{X}_{n} \leq u\right) \nonumber\\
%                                            &= \frac{\rho}{1 + \rho} + \frac{\rho}{1 + \rho}\sum_{n = 1}^\infty\frac{1}{(1 + \rho)^n}\Prob\left(\overline{X}_1 + \cdots + \overline{X}_{n} \leq u\right). \label{eq:distributionOfC}
%            \end{align}
%            Za preglednost ponovno uvedemo oznako $q= \frac{1}{1 + \rho}$ in 
%             $p = \frac{\rho}{1 + \rho}$ in ena"cbo (\ref{eq:distributionOfC}) preoblikujemo 
%            \begin{align*}
%                \Prob\left(C \leq u\right)  
%                    &= p + qp\overline{F}_{X_1}(u) + p\sum_{n = 2}^\infty q^{n}\Prob\left(\overline{X}_1 + \cdots + \overline{X}_{n} \leq u\right) \\
%                    &= p + qp\overline{F}_{X_1}(u) + qp\sum_{n = 2}^\infty q^{n-1}\int_{(0, u]}\Prob\left(x + \overline{X}_2 + \cdots + \overline{X}_{n} \leq u\right)d\overline{F}_{X_1}(x) \\
%                    &= p + q\int_{(0, u]}p\left[1 + \sum_{n = 2}^\infty q^{n-1}\Prob\left(\overline{X}_2 + \cdots + \overline{X}_n \leq u - x\right)\right]d\overline{F}_{X_1}(x) \\
%                    &= p + q\int_{(0, u]}\Prob\left(C \leq u - x\right)d\overline{F}_{X_1}(x).
%            \end{align*}
%            Vidimo, da $F_C$ zado"s"ca ena"cbi (\ref{eq:verjetnostPrezivetja3}), torej je res $F_C = \theta$. 
            \noindent
            Limito $\lim_{u\to\infty}\frac{\psi(u)}{1 - \overline{F}_{X_1}(u)}$ lahko tako zapi"semo kot
            \begin{align*}
                \lim_{u\to\infty}\frac{\psi(u)}{1 - \overline{F}_{X_1}(u)}   &= \lim_{u\to\infty}(1 - q)\sum_{k=1}^{\infty}q^k\frac{\Prob\left(\overline{W}_k > u\right)}{1 - \overline{F}_{X_1}(u)}.
            \end{align*}
            Ker je $\overline{F}_{X_1}$ subeksponentna, "ce vzamemo $\varepsilon < \frac{1}{q} - 1$, 
            lahko po lemi \ref{lema:ocenaSubeksponentnePorazdelitve} zaporedje funkcij 
            \begin{equation*}
                f_k(n) = \frac{\Prob\left(\overline{W}_k > u_k\right)}{1 - \overline{F}_{X_1}(u_k)}, \quad n\geq 2,\ k\in\N, 
            \end{equation*}
            za vsak $n\geq 2$ omejimo s funkcijo $K(\varepsilon)(1 + \varepsilon)^n$, ki je integrabilna glede na mero, ki "steje. Limito $u\to\infty$ po 
            realnih "stevilih nadomestimo z limito po zaporedjih pozitivnih "stevil $u_k\xrightarrow{k\to\infty}\infty$.
            Tako lahko po Lebesgueovem izreku o dominirani konvergenci \refPriloga{izr:dominiranaKonvergenca}
            zamenjamo vrstni red limite in vsote.
            Ker je $\overline{F}_{X_1}$ subeksponentna, za vsak $n\in\N$ velja
            \begin{equation*}
                \lim_{u\to\infty}\frac{\Prob\left(\overline{W}_k > u\right)}{1 - \overline{F}_{X_1}(u)} = n.
            \end{equation*}
            Kon"cno je
            \begin{align*}
                \lim_{u\to\infty}\frac{\psi(u)}{1 - \overline{F}_{X_1}(u)} &= (1 - q)\sum_{k=1}^\infty q^kk \\
                                 &= \frac{q}{(1 - q)} =  \frac{1}{\rho}.
            \end{align*}
        \end{proof}

        Izreka \ref{izr:CramerjevaMeja} in \ref{izr:tezkorepnePorazdelitveAsimptotika} poka"zeta klju"cno 
        razliko med lahkorepimi in te"zkorepimi porazdelitvami. Limita (\ref{eq:tezkorepnePorazdelitveAsimptotika})
        nam pove, da je konvergenca verjetnosti propada $\psi$ enakega reda kot $(1 - \overline{F}_{X_1})$, ki pa 
        ni zanemarljiva tudi za velike vrednosti $u$. To pomeni, da so zavarovalni"ski produkti, za katere 
        verjamemo, da so zahtevki subeksponentno porazdeljneni, nevarni, saj imajo najve"cji zahtevki velik 
        vpliv na celoten proces tveganja in se lahko propad zgodi zaradi enega samega velikega
        zahtevka kot v zgledu \ref{zgd:weibullProcesTveganja}.

        \begin{zgled}
        Naj bo $(U_t)_{t\geq0}$ proces tveganja v Cramér--Lundbergovem modelu, ki zado"s"ca NPC.\ Naj 
        nadalje velja, da so zahtevki neodvisni Weibullovo porazdeljeni s parametroma
        $a= \frac{1}{4}$ in $b= 16$, torej $X_i\sim\text{Weibull}(\frac{1}{4}, 16)$. 
        Dokaz, da je $\overline{F}_{X_1}$ subeksponentna porazdelitvena funkcija, lahko bralec 
        najde v \cite{9} na strani 444.
        Recimo, da je intenzivnost 
        prihodov zahtevkov $\lambda = 1$ in stopnja prihodkov premij $c = 500$.
        Podobno kot v zgledu \ref{zgd:MonteCarlo} z Monte Carlo simulacijami poka"zimo, da 
        verjetnost propada res pada proti $0$
        z enakim redom konvergence kot rep $\overline{F}_{X_1}$, ko gre $u\to\infty$. 
        Porazdelitev integriranega repa $\overline{F}_{X_1}$ ima obliko
        \begin{equation*}
            \overline{F}_{X_1}(u) = \frac{1}{\E\left[X_1\right]}\int_{(0, u]}e^{-\left(\tfrac{x}{16}\right)^{1/4}}dx.
        \end{equation*}
        Iz zgleda \ref{zgd:weibullProcesTveganja} vemo, da je $\E\left[X_1\right] = 384$. Z uvedbo nove spremenljivke
        $z = x^{1/4} (dz = \frac{1}{4x^{3/4}}dx)$ z nekaj ra"cunanja dobimo 
        \begin{equation*}
            \overline{F}_{X_1}(u) = 1 - \frac{\left(u^{3/4} + 6 \sqrt{u} + 24 \sqrt[4]{u} + 48\right)e^{-{\sqrt[4]{u}}/{2}}}{48}.
        \end{equation*}
        Izra"cunamo "se 
        \begin{equation*}
        \rho = \frac{c \E\left[T_1\right]}{\E\left[X_1\right]} - 1 = \frac{500}{384} - 1 \approx 0.3020833.
        \end{equation*}
        Po izreku \ref{izr:tezkorepnePorazdelitveAsimptotika} razmerje $\tfrac{\psi(u)}{1 - \overline{F}_{X_1}(u)}$ konvergira proti $\tfrac{1}{\rho} \approx 3.3103451$.
        Zaporedje $(u_n)_{n = 1}^{50}$, definirano kot $u_n = 2000n$ za vsak $n$, podobno kot v zgledu \ref{zgd:MonteCarlo} simuliramo $10, 100$ in $250$ realizacij
        procesa tveganja in za vsak $n$ izra"cunamo pribli"zek za razmerje $\tfrac{\psi(u_n)}{1 - \overline{F}_{X_1}(u_n)}$.
        Rezultate prika"zemo na sliki \ref{fig:slika5}.
       
        \begin{figure}[H]       
            \centering
            \includegraphics[width=\textwidth]{
                C:/Users/38651/OneDrive - Univerza v Ljubljani/Desktop/Diploma/Diplomski-seminar/GraphsAndPhotos/slika5.pdf
                }
            \caption{Aproksimacija verjetnosti propada $\psi(u)$ z Monte Carlo simulacijami (modra) in 
            to"cna vrednost funkcije (rde"ca).}
            \label{fig:slika5}
        \end{figure}

        \noindent
        Vidimo, da razmerje vizualno res konvergira proti $\tfrac{1}{\rho}$, ampak seveda bi 
        za bolj"so natan"cnost morali pove"cati za"cetni kapital $u$ in "stevilo simulacij. 
        \label{zg:MonteCarloTezkiRepi}
        \end{zgled}

        \newpage

\section{Priloga}
    Priloga je namenjena predvsem za dodatne definicije in trditve, s katerimi naj bi bil bralec seznanjen in 
    so bile izpu"scene v glavnem za namene preglednosti besedila. V primeru, da bralec potrebuje osve"ziti dolo"cene pojme, ki 
    se pojavljajo v besedilu, jih ve"cino lahko najde v tem razdelku.
    \begin{definicija}
        Naj bo $X$ diskretna slu"cajna spremenljivka z vrednostmi v $\N_0$. Potem za $u\in\mathbb{C}$ 
        z $|u| < 1$ definiramo \textit{rodovno funkcijo} slu"cajne spremenljivke $X$ kot
        \begin{equation*}
            G_X(u) = \E\left[u^X\right] = \sum_{k = 0}^\infty u^k\Prob(X = k).
        \end{equation*}
        Naj bo $Y$ poljubna slu"cajna spremenljivka. Za $u\in\mathbb{C}$ definiramo
         \textit{momentno-rodovno funkcijo} slu"cajne spremenljivke $Y$ kot 
        \begin{equation*}
            M_Y(u) = \E\left[e^{uY}\right].
        \end{equation*} 
        Za $u\in\R$ definiramo "se \textit{karakteristi"cno funkcijo} slu"cajne spremenljivke $Y$ kot
        \begin{equation*}
            \varphi_Y(u) = M_Y(iu) =  \E\left[e^{iuY}\right].
        \end{equation*}
        \label{def:rodovneFunkcije}
    \end{definicija}

    \begin{opomba}
        Karakteristi"cno funkcijo bi lahko sicer definirali tudi za kompleksne argumente, vendar pa 
        "ze vrednosti karakteristi"cne funkcije za vse realne argumente enoli"cno dolo"cajo porazdelitev 
        slu"cajne spremenljivke. 
    \end{opomba}

    \begin{definicija}
        Naj bo $\boldsymbol{X}$ $n$-razse"zni slu"cajni vektor. Potem ima za $u\in\R^n$ \textit{karakteristi"cna 
        funkcija} slu"cajnega vektorja $\boldsymbol{X}$ obliko
        \begin{equation*}
            \varphi_{\boldsymbol{X}}(u) = \E\left[e^{i\boldsymbol{u}^T\boldsymbol{X}}\right].
        \end{equation*}
        \label{def:karakteristicnaSlucVektor}
    \end{definicija}

    \begin{trditev}
        Naj bo $\boldsymbol{X}$ $n$-razse"zni slu"cajni vektor. Komponente $\boldsymbol{X}$ so neodvisne
        slu"cajne spremenljivke natanko tedaj, ko ima karakteristi"cna funkcija slu"cajnega vektorja
        $\boldsymbol{X}$ obliko
        \begin{equation*}
            \varphi_{\boldsymbol{X}}(u) = \prod_{i=1}^n\varphi_{X_i}(u_i), \quad u\in\R^n.
        \end{equation*}
        \label{trd:karakteristicnaNeodivsnost}
    \end{trditev}

    \begin{proof}
        Dokaz trditve lahko bralec najde v \cite{7} na strani 236. 
    \end{proof}

    \begin{definicija}
        Naj bosta $F$ in $G$ porazdelitveni funkciji dveh neodvisnih nenegativnih slu"cajnih 
        spremenljivk $X$ in $Y$. \textit{Konvolucijo} funkcij $F$ in $G$ definiramo kot
        Lebesgue--Stieltjesov integral
        \begin{equation*}
            (F*G)(t) = \int_{[0, t]}F(t - x)\, dG(x) = \int_{[0, t]}G(t - x)\, dF(x).
        \end{equation*} 
        Konvolucija $(F*G)$ se ujema s porazdelitveno funkcijo vsote $X + Y$, kar lahko 
        enostavno poka"zemo z uporabo transformacijske formule.
        Za neodvisne in enako porazdeljene nenegativne slu"cajne spremenljivke $X_1, X_2, \dots, X_n$ 
        s porazdelitveno funkcijo $F_{X_1}$ rekurzivno definiramo \textit{k-to konvolucijo} kot
        \begin{equation*}
            F_{X_1}^{*k}(t) = (F_{X_1}*\cdots*F_{X_1})(t) = \int_{[0, t]}F_{X_1}^{*(k-1)}(t - x)\, dF_{X_1}(x),
        \end{equation*}
        ki se prav tako ujema s porazdelitveno funkcijo vsote $X_1 + X_2 + \cdots + X_n$. 
        \label{def:konvolucija}
    \end{definicija}

    \begin{definicija}
        Naj bo $(\Omega, \mathcal{F}, \mathbb{P})$ verjetnostni prostor in naj bo $T\neq\emptyset$
        neprazna indeksna množica ter $(E, \Sigma)$ merljiv prostor. \textit{Slučajni proces}, 
        parametriziran s $T$, je družina slučajnih elementov $X_t : \Omega \to E$,
         ki so $(\mathcal{F}, \Sigma)$-merljivi za vsak $t \in T$.
        \label{def:slucProc}
    \end{definicija}

    \begin{opomba}
        V delu se omejimo na primer, ko $T$ predstavlja "cas, torej $T = [0, \infty)$ in da slu"cajne
        spremenljivke 
        zavzemajo vrednosti v realnih "stevilih, torej $(E, \Sigma) = (\R, \B_{\R})$, kjer $\B_\R$ 
        predstavlja Borelovo $\sigma$-algebro na $\R$.
        \label{op:Konvencije}
    \end{opomba}


    \begin{definicija}
        Za fiksen $\omega \in \Omega$ je preslikava 
        $[0, \infty) \rightarrow \mathbb{R}; \ t \mapsto X_t(\omega)$ 
        \textit{trajektorija} oziroma \textit{realizacija} slučajnega procesa $(X_t)_{t\geq0}$.
        Tako lahko slu"cajni proces gledamo kot predpis, ki vsakemu elementu vzor"cnega prostora 
        $\Omega$ priredi slu"cajno funkcijo
        $(X_t(\omega))_{t\geq0}: [0, \infty) \rightarrow \mathbb{R}$.
        \label{def:realizac}
    \end{definicija}

    \begin{definicija}
        Naj bo $(X_t)_{t\geq0}$ slu"cajni proces. Potem za $s < t$ definiramo
        \textit{prirastek procesa} $X_t - X_s$ na intervalu $[s, t]$. Proces $(X_t)_{t\geq0}$ ima 
        \textit{neodvisne prirastke}, če so za vsak nabor realnih "stevil
        $0 \leq t_1 < t_2 < \ldots < t_n < \infty$ prirastki
        $$
            X_{t_2} - X_{t_1}, \ X_{t_3} - X_{t_2}, \ \ldots, \ X_{t_n} - X_{t_{n-1}}
        $$
        med seboj neodvisni.
        \label{def:prirastek}
    \end{definicija}

    \begin{definicija}
        Naj bo $(X_t)_{t\geq0}$ slu"cajni proces. Potem pravimo, da ima proces
        \textit{stacionarne prirastke}, "ce za vsak $s < t$ in vsak $h > 0$ velja, 
        da ima $X_{t+h} - X_{s+h}$ enako porazdelitev kot $X_t - X_s$.
        \label{def:stacPrir}
    \end{definicija}

    \begin{definicija}
        Naj bodo $(X_t^{(i)})_{t\geq0},$ $i=1, \dots, n,$ slu"cajni procesi definirani na 
        $(\Omega, \F, \Prob)$. Pravimo, da so $(X_t^{(i)})_{t\geq0}$ \textit{neodvisni}, 
        "ce so za poljubne kon"cne nabore realnih "stevil $0 \leq t_1^{(i)} < t_2^{(i)} < \ldots < t_{n_i}^{(i)} < \infty$ 
        slu"cajni vektorji
        $(X_{t_1^{(i)}}^{(i)}, X_{t_2^{(i)}}^{(i)}, \ldots, X_{t_{n_i}^{(i)}}^{(i)})$ med seboj neodvisni. 
        \label{def:neodvisnostProcesov}
    \end{definicija}

    %\begin{definicija}
    %    Naj bo $(\Omega, \mathcal{F}, \Prob)$ verjetnostni prostor. Nara"scajocemu zaporedju 
    %    $\sigma$-algeber $(\mathcal{F}_t)_{t\geq0}$ pravimo \textit{filtracija}, "ce za vsak $ 0\leq s\leq t$ velja
    % $\mathcal{F}_s \subseteq \mathcal{F}_{t} \subseteq \mathcal{F}.$ Za slu"cajni proces $(X_t)_{t\geq0}$ pravimo, 
    % da je \textit{prilagojen} filtraciji $\mathcal{F}$, "ce je $X_t$ $\mathcal{F}_t$-merljiva za vsak $t\geq0$.
    % \label{def:filtracija}
    %\end{definicija}

    %\begin{trditev}
    %    Naj bosta $X$ in $Y$ slu"cajni spremenljivki definirani na $(\Omega, \mathcal{F}, \Prob)$ in 
    %    $g$ in $h$ borelovi funkciji.
    %    Potem sta $X$ in $Y$ neodvisni natanko tedaj, ko sta $g(X)$ in $h(Y)$ neodvisni. Trditev naravno 
    %    posplo"simo na slu"cajne vektorje. 
    %\end{trditev}
%
    %\begin{proof}
    %    Naj bosta $B$ in $C$ poljubni borelovi mno"zici. "Ce sta $X$ in $Y$ neodvisni, velja 
    %    $\Prob(h(X)\in B, g(Y)\in C) = \Prob(X\in h^{-1}(B))\Prob(Y\in g^{-1}(C)) = \Prob(h(X)\in B) \Prob(g(Y)\in C)$. 
    %    "Ce sta $g(X)$ in $h(Y)$ neodvisni pa velja 
    %    $ \Prob(X\in B, Y\in C) = \Prob(h(X)\in h(B))\Prob( g(Y)\in g(C)) = \Prob(X\in B)\Prob( Y\in C)$. 
    %    
    %\end{proof}

    \begin{trditev}
        Naj bodo $X, Y$ in $Z$ slu"cajne spremenljivke ter $g$ in $h$ poljubni Borelovi funkciji. 
        "Ce velja $X\mid Z\sim Y$, velja tudi $X\mid Z\sim X\mid g(Z)\sim Y$.
        \label{trd:pogojneLastnosti}
    \end{trditev}

    \begin{proof}
        Da je $X\mid Z\sim Y$, pomeni, da ima $X$ pogojno na $\sigma(Z)$ vedno isto porazdelitev, potem pa mora
        biti to tudi brezpogojna porazdelitev, prav tako pa tudi pogojna porazdelitev glede
        na manj"so $\sigma$-algebro $\sigma(g(Z))$.
    \end{proof}

    \begin{definicija}
        Naj bo $X_1, X_2, \dots$ "stevno zaporedje slu"cajnih spremenljivk. Pravimo, da 
        so slu"cajne spremenljivke $X_1, X_2, \dots$ \textit{izmenljive}, "ce za vsako kon"cno 
        permutacijo $\pi$ velja, da je skupna porazdelitev slu"cajnih spremenljivk\newline $X_1, X_2, \dots$ enaka
        skupni porazdelitvi slu"cajnih spremenljivk $X_{\pi(1)}, X_{\pi(2)}, \dots$.
        \label{def:Izmenljivost}
    \end{definicija}

    \begin{trditev}
        Naj bo $X_1, X_2, \dots$ "stevno zaporedje neodvisnih enako porazdeljenih slu"cajnih spremenljivk. 
        Potem so slu"cajne spremenljivke $X_1, X_2, \dots$ izmenljive.
    \end{trditev}

    \begin{proof}
        Izmenljivost neposredno sledi iz definicije neodvisnosti slu"cajnih spremenljivk, saj se 
        skupna porazdelitev faktorizira na produkt posameznih porazdelitev.
    \end{proof}

    \begin{trditev}
        Naj bodo $X_1, \dots, X_n$ izmenljive slu"cajne spremenljivke in $g: \R^n \rightarrow \R^k$
        poljubna Borelova simetri"cna funkcija. Potem so vsi slu"cajni vektorji 
        $$
            (X_i, g(X_1, \dots, X_n)),  \quad i = 1, \dots, n
        $$
        enako porazdeljeni. Sledi, da so tudi vse pogojne porazdelitve slu"cajnih spremenljivk
        $$
            X_i\mid g(X_1, \dots, X_n), \quad i = 1, \dots, n
        $$
        enake.
        \label{trd:izmenljivostSimetricnaFunkcija}
    \end{trditev}

    \begin{proof}
        Za $i=1, \dots, n$ definiramo funkcijo
        \begin{align*}
            G_i(x_1, \dots, x_n) := (x_i, g(x_1, \dots, x_n)).
        \end{align*}
        Naj bo $\pi$ permutacija, ki zamenja $i$ in $j$, zaradi izmenljivosti in simetrije funkcije 
        $g$ velja: 
        \begin{align*}
            (X_i, g(X_1, \dots, X_n)) &= G_i(X_1, \dots, X_n) \\
                    &\sim G_i(X_{\pi(1)}, \dots, X_{\pi(n)}) \\
                    &= (X_{\pi(i)}, g(X_{\pi(1)}, \dots, X_{\pi(n)})) \\
                    &= (X_j, g(X_1, \dots, X_n)).
        \end{align*}
    \end{proof}

    \begin{definicija}
        Slu"cajna spremenljivka $X$ ima \textit{Weibullovo porazdelitev} s parametroma $a, b > 0$, 
        "ce ima njena porazdelitvena funkcija obliko 
        \begin{equation*}
            F_X(x) = 1 - e^{-\left(\tfrac{x}{b}\right)^a} \quad \text{za} \ x\geq 0
        \end{equation*}
        oziroma gostota obliko
        \begin{equation*}
            f_X(x) = \left(\frac{a}{b}\right)\left(\frac{x}{b}\right)^{a-1}e^{-\left(\tfrac{x}{b}\right)^a} \quad \text{za} \ x\geq 0.
        \end{equation*}
        \label{def:WeibullovaPorazdelitev}
    \end{definicija}

    \begin{trditev}
        Naj bo $X$ nenegativna slu"cajna spremenljivka na verjetnostnem prostoru $(\Omega, \mathcal{F}, \Prob)$, 
        ki ima prvi moment. Potem velja 
        \begin{equation*}
            \E\left[X\right] = \int_0^\infty\bigl(1 - F_X(x)\bigr)\,dx.
        \end{equation*}
        \label{trd:PricakovanaVrednostZPrezivetveno}
    \end{trditev}

    \begin{proof}
        Vsako "stevilo $X\geq 0$ lahko zapi"semo kot 
        \begin{equation*}
            X = \int_0^\infty\mathbbm{1}_{\{x < X\}}dx = \int_0^\infty\mathbbm{1}_{\{X < x\}}dx.
        \end{equation*}
        "Ce sedaj uporabimo Fubinijev izrek, dobimo
        \begin{align*}
            \E\left[X\right] &= \E\left[\int_0^\infty\mathbbm{1}_{\{X < x\}}dx\right] \\
                             &= \int_0^\infty\E\left[\mathbbm{1}_{\{X < x\}}\right]dx \\
                             &= \int_0^\infty\bigl(1 - \Prob\left(X > x\right)\bigr)dx \\
        \end{align*}
    \end{proof}

    \begin{trditev}(Neenakost Markova)
        \label{trd:neenakostMarkova}
        Naj bo $X$ nenegativna slu"cajna spremenljivka.
        Potem za  $x>0$ velja
        \begin{equation*}
            \Prob\left(X \geq x\right) \leq \frac{\E\left[X\right]}{x}.
        \end{equation*}
    \end{trditev}

    \begin{proof}
        Naj bo $x > 0$. Velja
        \begin{equation*}
            x\mathbbm{1}_{\{X \geq x\}} \leq X \iff x\Prob\left(X \geq x\right) \leq \E\left[X\right].
        \end{equation*}
    \end{proof}

    %\begin{definicija}
    %    Naj bo $X_1, X_2, \dots$ zaporedje slu"cajnih spremenljivk s porazdelitvenimi funkcijami
    %    $F_{X_1}, F_{X_2}, \dots$ in naj bo $X$ slu"cajna spremenljivka s porazdelitveno funkcijo $F_X$.
    %    Pravimo da zaporedje $(X_n)_{n\in\N}$ \textit{konvergira v porazdelitvi} k slu"cajni spremenljivki $X$,
    %    "ce za vsak $x\in\R$, v katerem je $F_X$ zvezna, velja
    %    \begin{equation*}
    %        \lim_{n\to\infty}F_{X_n}(x) = F_X(x).
    %    \end{equation*}
    %    \label{def:KonvergencaVPorazdelitvi}
    %\end{definicija}

    \begin{izrek}(Krepki zakon velikih "stevil)
        Naj bo $(X_n)_{n\in\N}$ zaporedje neodvisnih enako porazdeljenih
        slu"cajnih spremenljivk na verjetnostnem prostoru $(\Omega, \mathcal{F}, \Prob)$
         s pri"cakovano vrednostjo $\E\left[X_i\right] = \mu <\infty$. Potem velja
        \begin{equation*}
            \frac{X_1 + X_2 + \cdots + X_n}{n}\xrightarrow[n\to\infty]{s.g.} \mu.
        \end{equation*}
        \label{izr:KrepkiZakonVelikihStevil}
    \end{izrek}

    \begin{proof}
        Dokaz izreka lahko bralec najde v \cite{7} na strani 192.
    \end{proof}

    \begin{izrek}(Izrek o enoli"cnosti)
        Naj bosta $X$ in $Y$ slu"cajni spremenljivki, ne nujno definirani na istem verjetnostnem prostoru.
        "Ce za vsak $u\in\R$ velja $\varphi_X(u) = \varphi_Y(u)$, imata $X$ in $Y$ enako porazdelitev.
        \label{izr:enolicnost}
    \end{izrek}

    \begin{proof}
        Dokaz izreka lahko bralec najde v \cite{13} na strani 346.
    \end{proof}

    \begin{izrek}(Lévijev izrek o kontinuiteti)
        Naj bo $(X_n)_{n\in\N}$ zaporedje slu"cajnih spremenljivk (ne nujno na istem verjetnostnem prostoru)
        in $X$ "se ena slu"cajna spremenljivka. Potem za vsak $u\in\R$ velja
        \begin{equation*}
            \varphi_{X_n}(u) \xrightarrow{n\to\infty} \varphi_X(u) 
        \end{equation*}
        natanko tedaj, ko velja
        \begin{equation*}
            X_n \xrightarrow[n\to\infty]{d} X.
        \end{equation*}
        \label{izr:LevijevIzrek}
    \end{izrek}

    \begin{proof}
        Dokaz izreka lahko bralec najde v \cite{7} na strani 260.
    \end{proof}

    \begin{izrek}(Lebesgueov izrek o monotoni konvergenci)
        Naj bo $X_1, X_2, \dots $ zaporedje nenegativnih slu"cajnih spremenljivk na 
        verjetnostnem prostoru $(\Omega, \mathcal{F}, \Prob)$ in naj bo $X:= \lim_{n\to\infty}X_n$ 
        njihova limita. Naj za vsak $\omega \in \Omega$
        velja $X_1(\omega) \leq X_2(\omega) \leq \dots$ Potem velja 
        \begin{equation*}
            \lim_{n\to\infty}\E\left[X_n\right] = \E\left[\lim_{n\to\infty}X_n\right] = \E\left[X\right].
        \end{equation*}
        \label{izr:monotonaKonvergenca}
    \end{izrek}

    \begin{proof}
        Dokaz izreka lahko bralec najde v \cite{7} na strani 105.
    \end{proof}  

    \begin{izrek}(Lebesgueov izrek o dominirani konvergenci)
        Naj bo $X_1, X_2, \dots $ zaporedje slu"cajnih spremenljivk na verjetnostnem prostoru
        $(\Omega, \mathcal{F}, \Prob)$ in naj bo $X:= \lim_{n\to\infty}X_n$ njihova limita.
        Naj bo $Y$ slu"cajna spremenljivka definirana na istem verjetnostnem prostoru z $\E\left[Y\right]<\infty$ in
        naj za vsak $n\in\N$ in vsak $\omega\in\Omega$ velja $|X_n(\omega)| \leq Y(\omega)$. Potem je $X$ integrabilna
        in velja 
        \begin{equation*}
            \lim_{n\to\infty}\E\left[X_n\right] = \E\left[\lim_{n\to\infty}X_n\right] = \E\left[X\right].
        \end{equation*}
        \label{izr:dominiranaKonvergenca}
    \end{izrek}

    \begin{proof}
        Dokaz izreka lahko bralec najde v \cite{7} na strani 107.
    \end{proof}

    \begin{izrek}(Tonellijev izrek)
        Naj bosta $X$ in $Y$ slu"cajni spremenljivki definirani vsaka na svojem verjentnostnem prostoru
        in naj imata vsaka svojo gostoto $f_X$ in $f_Y$ glede na Lebesgueovo mero.
        Potem velja
        \begin{equation*}
            \int_{\R^2}f_{X, Y}(x, y)\mathcal{L}^2(dx, dy) 
            = \int_{\R}\left(\int_{\R}f_{X, Y}(x, y)dx\right)dy = \int_{\R}\left(\int_{\R}f_{X, Y}(x, y)dy\right)dx.
        \end{equation*}
        \label{izr:TonellijevIzrek}
    \end{izrek}

    \begin{proof}
        Dokaz izreka lahko bralec najde v \cite{12} na strani 201.
    \end{proof} 

    \begin{trditev}(Lastnost vrstilnih statistik)
        Naj bo $(N_t)_{t\geq0}$ homogeni Poissonov proces z intenzivnostjo $\lambda > 0$. 
        Za $k\in\N$ je pogojno na dogodek $\{N_t = k\}$ vektor "casov prihodov porazdeljen kot 
        \begin{equation*}
            (V_1, \dots, V_k) \mid \{N_t = k\} \sim (U_{(1)}, \dots, U_{(k)}),
        \end{equation*}
        kjer je $(U_{(1)}, \dots, U_{(k)})$ vektor vrstilnih statistik vektorja $(U_1, \dots, U_k)$ 
        neodvisnih \newline enako porazdeljenih slu"cajnih spremenljivk $U_i\sim U\left([0, t]\right)$.
        \label{trd:VrstilneStatistikeHPP}
    \end{trditev}

    \begin{proof}
        Dokaz trditve lahko bralec najde v \cite{4} na strani 24.
    \end{proof}

    \begin{definicija}
        Naj bo $X$ nenegativna slu"cajna spremenljivka in $F_X$ njena porazdelitvena funkcija. 
        Potem za $u\in\R$ \textit{Laplace-Stieltjesovo transformacijo} funkcije $F_X$ oziroma porazdelitve 
        slu"cajne spremenljivke $X$ definiramo kot
        \begin{equation*}
            \hat{F}_X(u) = \int_{[0, \infty)}e^{-ux}dF_X(x).
        \end{equation*}
        \label{def:LaplaceStiltjesovaTransformacija}
    \end{definicija}

    \begin{definicija}
        Naj bo $F$ porazdelitvena funkcija neke nenegativne slu"cajne spremenljivke, ki ni skoraj gotovo enaka ni"c in ima prvi
        moment. \textit{Porazdelitev integriranega repa} te slu"cajne spremenljivke je porazdelitev
        s porazdelitveno funkcijo
        \begin{align*}
            \overline{F}(x) = \frac{1}{\mathbb{E}[X]} \int_0^x (1 - F(t)) \, dt.
        \end{align*}
        \label{def:porazdelitevintegriranegaRepa}
    \end{definicija}

    \begin{opomba}
        Iz trditve \refPriloga{trd:PricakovanaVrednostZPrezivetveno} sledi, da je $\overline{F}$ res 
        porazdelitvena funkcija. Nadalje za poljubno Borelovo merljivo funkcijo $h:\R\to\R$ velja 
        \begin{align*}
            \E\left[h(\overline{X})\right] &= \int_{[0, \infty)}h(x)d\overline{F}(x)\\
                &= \frac{1}{\E\left[X\right]}\int_0^\infty h(x)\left(1 - F(x)\right)dx,\\
        \end{align*}
        br"z ko kateri od teh izrazov obstaja.
    \end{opomba}

    \begin{definicija}
        \textit{Prenovitveni proces} na verjetnostnem protoru $(\Omega, \mathcal{F}, \Prob)$ je slu"cajni 
        proces,
        dolo"cen z zaporedjem neodvisnih enako porazdeljenih medprihodnih "casov $(T_n)_{n\in\N}$, 
        ki zavzamejo vrednosti v $\R^+\cup\{\infty\}$, in sicer je podan z zvezo 
        \begin{equation*}
            N_t = \sum_{n=1}^{\infty}\mathbbm{1}_{\{S_n\leq t\}},
        \end{equation*}
        kjer je $S_n = T_1 + T_2 + \cdots + T_n$ "cas $n$-tega prihoda. Pripadajo"co 
        \textit{prenovitveno mero} prenovitvenega procesa definiramo kot $M(t) = \E\left[N_t\right]$ za 
        $t > 0$.
        \label{def:PrenovitveniProces}
    \end{definicija}

    \begin{definicija}
        \textit{Prenovitvena ena"cba} je ena"cba oblike 
        \begin{equation*}
            f(t) = g(t) + \int_{[0, t]}f(t - s)dF(s), \quad t\geq 0,
        \end{equation*}
        kjer sta neznana funkcija $f$ in znana funkcija $g$ definirani na $\R^+$, $F$ pa je 
        porazdelitvna funkcija neke pozitivne slu"cajne spremenljivke $X$.
        \label{def:prenovitvenaEnacba}
    \end{definicija}

    \begin{definicija}
        Za nenegativno merljivo funkcijo \( f : [0, \infty) \to [0, \infty) \) pravimo, da je \textit{direktno 
        Riemannovo integrabilna} (d.R.i.), če za vsak $\delta > 0$ velja
        \begin{equation*}
            \sum_{k \geq 0} \left( \sup_{t \in [k\delta, (k+1)\delta)} f(t) \right) < \infty \quad \text{in}
        \end{equation*}
        \begin{equation}
             \lim_{\delta \downarrow 0} \delta \sum_{k \geq 0} \left( \sup_{t \in [k\delta, (k+1)\delta)} f(t) \right) = \lim_{\delta \downarrow 0} \delta \sum_{k \geq 0} \left( \inf_{t \in [k\delta, (k+1)\delta)} f(t) \right)
             \label{eq:limitaDirektnegaRiemannovegaIntegrala}
        \end{equation}
        Če \(f\) zadošča navedenima zahtevama, definiramo njen \textit{direktni Riemannov integral} 
        \[
        \text{d.R.i.} \int_{0}^{\infty} f(t) \, dt
        \]
        kot limito (\ref{eq:limitaDirektnegaRiemannovegaIntegrala}).
        Funkcija \(f\) poljubnega predznaka je d.R.i., če sta le-taki \(f^+ = \max\{f, 0\}\) in \(f^- = \max\{-f, 0\}\), pri čemer je
        \[
        \text{d.R.i.} \int_{0}^{\infty} f(t) \, dt = \text{d.R.i.} \int_{0}^{\infty} f^+(t) \, dt - \text{d.R.i.} \int_{0}^{\infty} f^-(t) \, dt.
        \]
        \label{def:direktnaRieamnovaIntegrabilnost}
    \end{definicija}

    %\begin{trditev}(Kriterij za direktno Riemannovo integrabilnost)
    %    Naj bo $f \geq 0$ nenara"s"cajo"ca funkcija. Potem je $f$ direktno Riemannovo integrabilna natanko tedaj, ko je
    %    posplo"seno Riemannovo integrabilna. Tedaj je njen direktni Riemannov integral enak posplo"senemu.
    %    \label{trd:kriterijZaDirektnoRiemannovoIntegrabilnost}
    %\end{trditev}
%
    %\begin{proof}
    %    Dokaz trditve lahko bralec najde v \cite{8} na strani 235. 
    %\end{proof} 

    \begin{definicija}
        \textit{Totalna variacija} funkcije $f$ na intervalu $I$ je definirana kot 
        $$
            V(f;\> I):= \sup\limits_{\substack{x_0, x_1, \dots, x_n \in I \\ x_0 \leq x_1 \leq \cdots \leq x_n}}\sum_{k=1}^{n}|f(x_k) - f(x_{k-1})|;
        $$
        lahko je kon"cna ali neskon"cna. Za $a\leq b$ definiramo "se $V(f;\> a, b) := V(f;\> [a, b])$ in 
        $V(f;\> a, \infty) := V(f;\> [0, \infty))$.
        
        \label{def:totVariacija}
    \end{definicija}

    \begin{trditev}
        Vsaka funkcija $f:[0, \infty) \to [0, \infty)$ za katero obstaja posplo"seni Riemannov integral
        $\int_0^\infty f(t)dt$ in ima omejeno totalno variacijo na $[0, \infty)$, je direktno Riemannovo integrabilna.
        \label{trd:kriterijZaDirektnoRiemannovoIntegrabilnost}
    \end{trditev}

    \begin{proof}
        Za $\delta > 0$ ocenimo: 
        \begin{align*}
            \sum_{k=0}^\infty\left(\sup_{t\in[k\delta, (k+1)\delta)}f(t)\right) 
            &\leq \sum_{k=0}^\infty \left[\frac{1}{\delta}\int_{k\delta}^{(k+1)\delta}f(t)dt + V(f;\>k\delta, (k+1)\delta)\right] \\
            &\leq \frac{1}{\delta}\int_0^\infty f(t)dt + V(f;\>0, \infty) < \infty
        \end{align*}
        in 
        \begin{align*}
            &\delta\sum_{k=0}^\infty\sup_{t\in[k\delta, (k+1)\delta)}f(t) - \delta\sum_{k=0}^\infty\inf_{t\in[k\delta, (k+1)\delta)}f(t) \\
            &= \delta\sum_{k=0}^\infty\left(\sup_{t\in[k\delta, (k+1)\delta)}f(t) - \inf_{t\in[k\delta, (k+1)\delta)}f(t)\right) \\
            &\leq \delta\sum_{k=0}^\infty V(f;\>k\delta, (k+1)\delta) \\
            &= \delta V(f;\>0, \infty),
        \end{align*}
        slednje pa gre proti ni"c, ko gre $\delta$ proti ni"c.
    \end{proof}

    \begin{izrek}(Smithov klju"cni prenovitveni izrek)
        "Ce je funkcija $g$ iz prenovitvene ena"cbe $(g, F)$ (definicija \ref{def:prenovitvenaEnacba})
        omejena na kon"cnih intervalih ter $X$ ima prvi moment in ni aritmeti"cna 
        $(\nexists a\in\R: \ \Prob\left(X \in \mathbb{Z} a\right) = 1)$, je
        \begin{equation*}
            f(t) = g(t) +  \int_{[0, t]}g(t - s)dM(s), \quad t\geq 0,
        \end{equation*}
        enoli"cna re"sitev te ena"cbe. Funkcija $M$ je prenovitvena mera prenovitvenega procesa z medprihodno 
        porazdelitvijo $F$.
        "Ce je dodatno funkcija $g$ direktno Riemannovo integrabilna, pa velja "se
        \begin{equation*} 
            \lim_{t\to\infty}f(t) = \frac{1}{\E\left[X\right]}\int_{[0, \infty)}g(t)dt.
        \end{equation*}
        \label{izr:Smith}
    \end{izrek}

    \begin{proof}
        Dokaz izreka lahko bralec najde v \cite{8} na strani 237. 
    \end{proof}

    \begin{izrek}(Banachov izrek o negibni to"cki)
        Naj bo $(M, d)$ poln metri"cni prostor in naj bo 
        $$
            f:(M, d) \to (M, d)
        $$
        skr"citev, torej naj obstaja tak $0 \leq k < 1$, da za vsak $x, y \in M$ velja 
        $d(f(x), f(y)) \leq k d(x, y)$. Potem obstaja natanko en $x^*\in M$, za katerega velja $f(x^*) = x^*$. 
        "Se ve"c, za vsak $x\in M$ velja $x^* = \lim_{n\to\infty}\underbrace{(f\circ \cdots \circ f)}_{n}(x)$.
         \label{izr:Banach}
    \end{izrek}

    \begin{proof}
        Dokaz izreka lahko bralec najde v \cite{11} na strani 284.
    \end{proof}

%-----------------------------------------KONEC VSEBINE--------------------------------------------%
\newpage
\section*{Slovar strokovnih izrazov}

\noindent
\textbf{trajektorija} \ sample path \\
\textbf{sestavljeni procesi} \  compound processes \\
\textbf{sestavljeni Poissonov proces} \ compound Poisson process \\
\textbf{markiranje procesa} \ space-time decomposition of process \\
\textbf{neskon"cna deljivost} \ infinite divisibility \\
\textbf{proces tveganja} \ risk process \\
\textbf{verjetnost propada} \ probability of ruin \\
\textbf{ogrodje procesa tveganja} \ skeleton process \\
\textbf{lahkorepa porazdelitev} \ light-tailed distribution \\
\textbf{te"zkorepa porazdelitev} \ heavy-tailed distribution \\
\textbf{subeksponentna porazdelitev} \ subexponential distribution \\
\textbf{prenovitveni proces} \ renewal process \\
\textbf{defektna prenovitvena ena"cba} \ defective renewal equation \\
\textbf{prenovitvena mera} \ renewal function \\

% Literatura
\begin{thebibliography}{99}
\bibitem{1}S.E. Shreve, Stochastic Calculus for Finance II: Continuous-Time Models, Springer, (2004).
\bibitem{2}S.M. Ross, Stochatic Processes: Second Edition, Wiley, (1996).
\bibitem{3}P. Embrechts, C. Klüppelberg, T. Mikosch, Modelling Extremal Events: For Insurance and Finance, Springer, (1997).
\bibitem{4}T. Mikosch, Non-Life Insurance Mathematics: An Introduction with the Poisson Process, Second Edition, Springer, (2009).
\bibitem{5}M. Mandjes, O. Boxma, The Cramér--Lundberg model and its variants, Springer, (2023).
\bibitem{6}F. Spitzer, Principles of Random Walk, Second Edition, Springer, (1976).
\bibitem{7}B. Fristedt, L. Gray, A Modern Approach to Probability Theory, Springer, (1996).
\bibitem{8}S.I. Resnick, Adventures in Stochastic Processes, Birkhäuser, (1992).
\bibitem{9}R.J. Adler, R.E. Feldman, A practical guide to heavy tails: statistical techniques and applications, Birkhäuser, (1998).
\bibitem{10}K. Sato, Lévy Processes and Infinitely Divisible Distributions, Cambridge University Press, (1999).
\bibitem{11}J. Globevnik, M. Brojan, Analiza 1, DMFA--založništvo, (2016).
\bibitem{12}T. Tao, An introduction to measure theroy, American Mathematical Society, (2011).
\bibitem{13}P. Billingsley, Probability and Measure, Third edition, Wiley, (1995).
\end{thebibliography}

\end{document}
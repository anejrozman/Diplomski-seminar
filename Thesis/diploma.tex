\documentclass[12pt,a4paper]{amsart}
% ukazi za delo s slovenscino -- izberi kodiranje, ki ti ustreza
\usepackage[slovene]{babel}
\usepackage[T1]{fontenc}
\usepackage[utf8]{inputenc}
\usepackage{amsmath,amssymb,amsfonts,amsthm}
\usepackage{url}
%\usepackage[normalem]{ulem}
\usepackage[dvipsnames,usenames]{color}
\usepackage{graphicx}
\usepackage{tikz}
\usepackage{dsfont}
\usepackage{caption}
\usepackage{subcaption}
\usepackage{bm}
\usepackage{float}
\allowdisplaybreaks

% ne spreminjaj podatkov, ki vplivajo na obliko strani
\textwidth 15cm
\textheight 24cm
\oddsidemargin.5cm
\evensidemargin.5cm
\topmargin-5mm
\addtolength{\footskip}{10pt}
\pagestyle{plain}
\overfullrule=15pt % oznaci predlogo vrstico


% ukazi za matematicna okolja
\theoremstyle{definition} % tekst napisan pokoncno
\newtheorem{definicija}{Definicija}[section]
\newtheorem{primer}[definicija]{Primer}
\newtheorem{opomba}[definicija]{Opomba}

\renewcommand\endprimer{\hfill$\diamondsuit$}


\theoremstyle{plain} % tekst napisan posevno
\newtheorem{lema}[definicija]{Lema}
\newtheorem{izrek}[definicija]{Izrek}
\newtheorem{trditev}[definicija]{Trditev}
\newtheorem{posledica}[definicija]{Posledica}



% ukaz za slovarsko geslo
\newlength{\odstavek}
\setlength{\odstavek}{\parindent}
\newcommand{\geslo}[2]{\noindent\textbf{#1}\hspace*{3mm}\hangindent=\parindent\hangafter=1 #2}


% naslednje ukaze ustrezno popravi
\newcommand{\program}{Finančna matematika} % ime studijskega programa: Matematika/Finan"cna matematika
\newcommand{\imeavtorja}{Anej Rozman} % ime avtorja
\newcommand{\imementorja}{~doc.~dr. Martin Raič} % akademski naziv in ime mentorja
\newcommand{\naslovdela}{Potencialni naslov:Sestavljen Poissonov proces in njegova uporaba v financah} % naslov dela
\newcommand{\letnica}{2024} % letnica diplome

% Moji ukazi
\newcommand{\R}{\mathbb{R}}
\newcommand{\N}{\mathbb{N}}
\newcommand{\E}{\mathbb{E}}
\newcommand{\F}{\mathcal{F}}
\newcommand{\Prob}{\mathbb{P}}
\newcommand{\1}{\mathds{1}}
\newcommand{\Pois}[1]{\text{Pois}(#1)}
\newcommand{\Var}[1]{\text{Var}\left[#1\right]}





\begin{document}

% od tod do povzetka ne spreminjaj nicesar
\thispagestyle{empty}
\noindent{\large
UNIVERZA V LJUBLJANI\\[1mm]
FAKULTETA ZA MATEMATIKO IN FIZIKO\\[5mm]
\program\ -- 1.~stopnja}
\vfill

\begin{center}{\large
\imeavtorja\\[2mm]
{\bf \naslovdela}\\[10mm]
Delo diplomskega seminarja\\[1cm]
Mentor: \imementorja}
\end{center}
\vfill

\noindent{\large
Ljubljana, \letnica}
\pagebreak

\thispagestyle{empty}
\tableofcontents
\pagebreak

\thispagestyle{empty}
\begin{center}
{\bf \naslovdela}\\[3mm]
{\sc Povzetek}
\end{center}
% tekst povzetka v slovenscini

\vfill
\begin{center}
{\bf Compound Poisson process and its application in finance}\\[3mm] % angleski naslov
{\sc Abstract}
\end{center}
% tekst povzetka v anglescini
Prevod zgornjega povzetka v angle"s"cino.

\vfill\noindent
{\bf Math. Subj. Class. (2010):} 91G10 60G00 60G01  \\[1mm]
{\bf Klju"cne besede:} slu"cajni procesi, sestavljen Poissonov proces  \\[1mm]
{\bf Keywords:} Stochastic processes, Lévy processes
\pagebreak



% tu se zacne besedilo seminarja
\section{Uvod}

    Poissonov proces "steje "stevilo prihodov v danem "casovnem intervalu, kjer narava prihodov 
    sledi dolo"cenim omejitvam. Sestavljen Poissonov proces, je podoben 
    Poissonovemu, razen da je vsak prihod ute"zen z neko slu"cajno spremenljivko. Na primer, stranke, 
    ki gredo v trgovino, sledijo Poissonovemu procesu, znesek denarja, ki ga porabijo, pa lahko 
    sledi sestavljenemu Poissonovemu procesu. Slika \ref{fig:slika1} prikazuje primer trajktorije. 
    Na osi $x$ je "cas, na osi $y$ pa kumulativna vsota vseh prihodov do tega "casovnega trenutka.

    \begin{figure}[H]
        \centering
        \includegraphics[width=\textwidth]{C:/Users/38651/OneDrive - Univerza v Ljubljani/Desktop/Diploma/Diplomski-seminar/GrapsAndPhotos/slika1.pdf}
        \caption{Primer trajektroije sestavljenega Poissonovega procesa}
        \label{fig:slika1}
    \end{figure}
    
    \noindent
    Hitro vidimo, da je to zelo zanimiva ideja slu"cajnega
    procesa, ki ima veliko potencialnih uporab. Mogoce:V delu se bomo osredoto"cili na njegovo uporabo v 
    financah. Za za"cetek definirajmo osnovne pojme ter Sestavljen Poissonov proces.

    \begin{definicija}
        Naj bo $(\Omega, \mathcal{F}, \mathbb{P})$ verjetnostni prostor in naj bo $T\neq\emptyset$
        neprazna indeksna množica ter $(S, \Sigma)$ merljiv prostor. \textit{Slučajni proces}, 
        parametriziran s $T$, je družina slučajnih spremenljivk $X_t : \Omega \to S$,
         ki so $(\mathcal{F}, \Sigma)$-merljive za vsak $t \in T$.
        \label{def:slucProc}
    \end{definicija}

    \begin{opomba}
        Dr"zali se bomo konvencije, da $T$ predstavlja "cas, torej $T = [0, \infty)$.
        V tem primeru govorimo o zveznem slu"cnem procesu.
        \label{op:TCas}
    \end{opomba}

    \begin{definicija}
        Za fiksen $\omega \in \Omega$ je preslikava 
        $[0, \infty) \rightarrow \mathbb{R}; \ t \mapsto X_t(\omega)$ 
        \textit{trajektorija} oziroma \textit{realizacija} slučajnega procesa $(X_t)_{t\geq0}$.
        \label{def:realizac}
    \end{definicija}

    \begin{opomba}
        Na slu"cajni proces lahko gledamo tudi kot na predpis, ki nam iz vor"cnega prostora 
        $\Omega$ priredi slu"cajno funkcijo
        $(X_t(\omega))_{t\geq0}: [0, \infty) \rightarrow \mathbb{R}$.
        \label{op:slucFunkc}
    \end{opomba}

    \begin{definicija}
        Naj bo $(X_t)_{t\geq0}$ slu"cajni proces. Potem za $s < t$ definiramo
        \textit{prirastek procesa} $X_t - X_s$ na intervalu $[s, t]$. Proces $(X_t)_{t\geq0}$ ima 
        \textit{neodvisne prirastke}, če so za vsak nabor realnih "stevil
        $0 \leq t_1 < t_2 < \ldots < t_n < \infty$ prirastki
        $$
            X_{t_2} - X_{t_1}, \ X_{t_3} - X_{t_2}, \ \ldots, \ X_{t_n} - X_{t_{n-1}}
        $$
        med seboj neodvisni.
        \label{def:prirastek}
    \end{definicija}

    \begin{definicija}
        Naj bo $(X_t)_{t\geq0}$ slu"cajni proces. Potem pravimo, da ima proces
        \textit{stacionarne prirastke}, "ce za vsak $s < t$ in vsak $h > 0$ velja, 
        da ima $X_{t+h} - X_{s+h}$ enako porazdelitev kot $X_t - X_s$.
        \label{def:stacPrir}
    \end{definicija}

    \begin{definicija}
        Naj bo $\lambda > 0$. Slučajnemu procesu $(N_t)_{t\geq 0}$ definiranem na verjetnostnem 
        prostoru $(\Omega, \mathcal{F}, \mathbb{P})$ z vrednostmi v $\N_0$ pravimo 
        \textit{Poissonov proces} z intenzivnostjo $\lambda$, če zadošča naslednjim pogojem:
        \begin{enumerate}
            \item $N_0 = 0$ \ $\Prob$-skoraj gotovo.
            \item $(N_t)_{t\geq 0}$ ima neodvisne in stacionarne prirastke,
            \item Za $0 \leq s < t$ velja $ N_t - N_s \sim\Pois{\lambda(t - s)}$,
        \end{enumerate}
        \label{def:HPP}
    \end{definicija}

    \begin{opomba}
        Vidimo, da v definiciji ne zahtevamo, da so skoki procesa le +1... 
        \label{op:skoki}
    \end{opomba}

    \begin{definicija}
        Naj bo $(N_t)_{t\geq0}$ Poissonov proces z intenzivnostjo $\lambda$. 
        Naj bo $(X_i)_{i\geq1}$ zaporedje neodvisnih (med sabo in $N_t$) in enako porazdeljenih slučajnih spremenljivk 
        z vrednostmi v $\mathbb{R}$. Potem je \textit{sestavljen Poissonov proces} 
        $(S_t)_{t\geq0}$ definiran kot
        $$
            S_t = \sum_{i=1}^{N_t} X_i.
        $$
        \label{def:CPP}
    \end{definicija}

    \begin{opomba}
        Vidimo, da je Poissonov proces le poseben primer sestavljenega Poissonovega procesa, ko za
        $X_i$ vzamemo konstantno funkcijo $X_i = 1$ za vsak $i$. Bolj v splo"snem, "ce za $X_i$ 
        postavimo $X_i = \alpha$, potem velja $S_t = \alpha N_t$.
        \label{op:CPPHPPPovezava}
    \end{opomba}

    V nadaljevanju bomo Poissonovemu procesu rekli HPP (angl. Homogeneous Poisson Process) in 
    sestavljenemu Poissonovemu procesu CPP (angl. Compound Poisson Process).

\section{Lastnosti sestavljenega Poissonovega procesa}

    V tem poglavju si bomo ogledali osnovne lastnosti sestavljenega Poissonovega procesa. Pogledali
    si bomo... 
    \begin{trditev}
        CPP ima neodvisne in stacionarne prirastke.
        \label{trd:neodvPrirCPP}
    \end{trditev}

    \begin{proof}
        Za nabor realnih "stevil $0 \leq t_1 < t_2 < \ldots < t_n < \infty$ lahko slu"cajne
        spremeljivke $S_{t_i} - S_{t_{i-1}}$ zapi"semo kot
        \begin{align*}
            S_{t_i} - S_{t_{i-1}} &= \sum_{j=N_{t_{i-1}}+1}^{N_{t_i}} X_j. 
        \end{align*}
        Neodvisnost prirastkov sledi po neodvisnosti $X_i$ od $X_j$ za $i\neq j$ in $N_t$. 
        Naj bo $h > 0$ in $s < t$. Potem velja
        \begin{align*}
            S_{t+h} - S_{s+h} &= \sum_{j=N_{s+h}+1}^{N_{t+h}} X_j \\
        \end{align*}
        Vsota ima $N_{t+h} - N_{s+h}$ členov. Ker za HPP velja 
        $N_{t+h} - N_{s+h} \sim N_t - N_s$, je 
        \begin{align*}
            \sum_{j=N_{s+h}+1}^{N_{t+h}} X_j = \sum_{j=N_{s}+1}^{N_{t}} X_j = S_t - S_s.
        \end{align*}
    \end{proof}

    \begin{definicija}
        neki
    \end{definicija}

    Izra"cunajmo pri"cakovano vrednost in varianco CPP. Naj bo $(N_t)_{t\geq 0}$ HPP z 
    intenzivnostjo $\lambda$ in naj bo $\theta = \E\left[X_i\right]$ pri"cakovana vrednost 
    slu"cajnih spremenljivk $X_i$ za vsak $i$. Po formuli za popolno pri"cakovano vrednost velja 
    $\E\left[S_t\right] = \E\left[\E\left[S_t\mid N_t\right]\right]$. Torej

    \begin{align*}
        \E\left[S_t\right] &= \sum_{k=0}^{\infty} \E\left[S_t | N_t = k\right] \Prob\left(N_t = k\right)\\
                           &= \sum_{k=0}^{\infty} \E\left[\sum_{i=1}^{k} X_i\right] \Prob\left(N_t = k\right)\\
                           &= \sum_{k=0}^{\infty}k\E\left[X_i\right]\frac{(\lambda t)^k}{k!}e^{-\lambda t}\\
                           &= \theta\lambda te^{-\lambda t}\sum_{k=1}^{\infty}\frac{(\lambda t)^{k-1}}{(k-1)!}\\
                           &= \theta\lambda t.
    \end{align*}

    \noindent
    Za izra"cun variance potrebujemo dodatno predpostavko, da imajo slu"cajne spremenljivke $X_i$ 
    drugi moment. V tem primeru ozna"cimo $\Var{X_i} = neki$. Potem velja

    \begin{align*}
        pisi.
    \end{align*}

    \noindent
    Inzra"cunajmo "se momentno rodovno funckijo CPP. Ozna"cimo z $M_X(u)$ momentno rodovno funkcijo 
    s.s $X_i$ za vsak $i$ in z $M_{S_t}$ momentno rodovno funkcijo CPP.
    \begin{align*}
        M_{S_t}(u) &= \E\left[\exp\left[uS_t\right]\right] =
                     \E\left[\exp\left[u\sum_{i = 1}^{N_t}X_i\right]\right]\\
                   &= \Prob\left(N_t = 0\right) + \sum_{k=1}^{\infty}
                       \E\left[\exp\left[u\sum_{i = 1}^{N_t}X_i\mid N_t=k\right]\right]\Prob\left(N_t = k\right)\\ 
                   &= \Prob\left(N_t = 0\right) + \sum_{k=1}^{\infty}
                       \E\left[\exp\left[u\sum_{i = 1}^kX_i\right]\right]\Prob\left(N_t = k\right)\\
                   &= e^{-\lambda t} + \sum_{k=1}^{\infty}
                       \underbrace{\E\left[e^{uX}\right]^n}_{M_X(u)^n}\frac{(\lambda t)^k}{k!}e^{-\lambda t}\\ 
                   &= e^{-\lambda t} + e^{-\lambda t}\sum_{k=1}^\infty\frac{\left(M_X(u)\lambda t\right)^k}{k!}\\
                   &= e^{\lambda t\left(M_X(u) - 1\right)}    
    \end{align*}

    Iz opombe \ref{op:CPPHPPPovezava}, sledi, da "ce za $X_i$ vzamemo konstantno funkcijo 
    $X_i = 1$, dobimo HPP. Tako vidimo, da je momentno rodovna funkcija HPP enaka 
    $M_{S_t}(u) = e^{\lambda t\left(e^u - 1\right)}$. Poleg tega takoj dobimo, da sta rodovna in
    karakteristi"cna funkcija CPP enaki

    \begin{align*}
        \varphi_{S_t}(u) &= e^{\lambda t\left(e^{iu} - 1\right)} \ \text{in} \\
        G_{S_t}(u) &= e^{\lambda t\left(e^{iu} - 1\right)}.
    \end{align*} 













\section*{Slovar strokovnih izrazov}

%\geslo{}{}
%
%\geslo{}{}
%


% seznam uporabljene literature
\begin{thebibliography}{99}

\bibitem{1}S.E. Shreve, Stochastic Calculus for Finance II: Continuous-Time Models, Springer, (2004).
\bibitem{2}S.M. Ross, Stochatic Processes: Second Edition, Wiley, (1996).
\end{thebibliography}

\end{document}
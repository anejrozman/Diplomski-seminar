\documentclass[]{beamer} %zaradi [handout] \pause ne dela
\usepackage[T1]{fontenc}
\usepackage[utf8]{inputenc}
\usepackage[slovene]{babel}
\usepackage{pgfpages}
\usepackage{amsmath}
\usepackage{amssymb}
\usepackage{colortbl}
\usepackage{tikz}
\usepackage{array}
\usepackage{amsmath,amsthm, amsfonts,amssymb}
\usepackage{mathtools}
\usepackage{dsfont}

\setbeameroption{hide notes}
%\setbeameroption{show notes on second screen=right}

\mode<presentation>
\usetheme{Berlin}
\useinnertheme[shadows]{rounded}
\useoutertheme{infolines}
\usecolortheme{seahorse}
\usepackage{palatino}
\usefonttheme{serif}

%okolja za izreke, definicije, ...
\theoremstyle{plain}
\newtheorem{izrek}{Izrek}
\newtheorem{definicija}{Definicija}
\newtheorem{trditev}{Trditev}
\newtheorem{posledica}{Posledica}
\newtheorem{opomba}{Opomba}
\newtheorem{zgled}{Zgled}
\newtheorem{lema}{Lema}

%\beamertemplatenavigationsymbolsempty
\setbeamertemplate{headline}{}
%\setbeamertemplate{footline}{}

\title[CPP in njegova uporaba v financah]{Sestavljeni Poissonov proces in njegova upraba v financah}
\subtitle{}
\author[Anej Rozman]{Anej Rozman}
\institute[]{Mentor: doc.~dr. Martin Raič}
\date[]{}

\newcommand{\R}{\mathbb{R}}
\newcommand{\N}{\mathbb{N}}
\newcommand{\E}{\mathbb{E}}
\newcommand{\F}{\mathcal{F}}
\newcommand{\B}{\mathcal{B}}
\newcommand{\Prob}{\mathbb{P}}
\newcommand{\1}{\mathds{1}}
\newcommand{\Pois}[1]{\text{Pois}(#1)}
\newcommand{\Var}[1]{\text{Var}\left[#1\right]}

\begin{document}


\frame{\titlepage}

\begin{frame}
  \frametitle{Poissonov proces}
  \begin{definicija}
    Naj bo $\lambda > 0$. Slučajnemu procesu $(N_t)_{t\geq 0}$, definiranem na verjetnostnem 
    prostoru $(\Omega, \mathcal{F}, \mathbb{P})$ in z vrednostmi v $\mathbb{N}_0$, pravimo 
    \textit{Poissonov proces} z intenzivnostjo $\lambda$, če zadošča naslednjim pogojem:
    \begin{enumerate}
        \item $N_0 = 0$ \ $\Prob$-skoraj gotovo.
        \item $(N_t)_{t\geq 0}$ ima neodvisne in stacionarne prirastke,
        \item Za $0 \leq s < t$ velja $ N_t - N_s \sim\Pois{\lambda(t - s)}$,
    \end{enumerate}
  \end{definicija}
\end{frame}

\end{document}
\documentclass[]{beamer} %zaradi [handout] \pause ne dela
\usepackage[T1]{fontenc}
\usepackage[utf8]{inputenc}
\usepackage[slovene]{babel}
\usepackage{pgfpages}
\usepackage{amsmath}
\usepackage{amssymb}
\usepackage{colortbl}
\usepackage{tikz}
\usepackage{array}
\usepackage{amsmath,amsthm, amsfonts,amssymb}
\usepackage{mathtools}
\usepackage{dsfont}

\setbeameroption{hide notes}
%\setbeameroption{show notes on second screen=right}

\mode<presentation>
\usetheme{Berlin}
\useinnertheme[shadows]{rounded}
\useoutertheme{infolines}
\usecolortheme{seahorse}
\usepackage{palatino}
\usefonttheme{serif}

%okolja za izreke, definicije, ...
\theoremstyle{plain}
\newtheorem{izrek}{Izrek}
\newtheorem{definicija}{Definicija}
\newtheorem{trditev}{Trditev}
\newtheorem{posledica}{Posledica}
\newtheorem{opomba}{Opomba}
\newtheorem{zgled}{Zgled}
\newtheorem{lema}{Lema}

%\beamertemplatenavigationsymbolsempty
\setbeamertemplate{headline}{}
%\setbeamertemplate{footline}{}

\title[CPP in njegova uporaba v financah]{Sestavljeni Poissonov proces in njegova upraba v financah}
\subtitle{}
\author[Anej Rozman]{Anej Rozman}
\institute[]{Mentor: doc.~dr. Martin Raič}
\date[]{}

\newcommand{\R}{\mathbb{R}}
\newcommand{\N}{\mathbb{N}}
\newcommand{\E}{\mathbb{E}}
\newcommand{\F}{\mathcal{F}}
\newcommand{\B}{\mathcal{B}}
\newcommand{\Prob}{\mathbb{P}}
\newcommand{\1}{\mathds{1}}
\newcommand{\Pois}[1]{\text{Pois}(#1)}
\newcommand{\Var}[1]{\text{Var}\left[#1\right]}

\begin{document}


\frame{\titlepage}

\begin{frame}
  \frametitle{Poissonov proces}
  \begin{definicija}
    Naj bo $\lambda > 0$. Slučajnemu procesu $(N_t)_{t\geq 0}$, definiranem na verjetnostnem 
    prostoru $(\Omega, \mathcal{F}, \mathbb{P})$ in z vrednostmi v $\mathbb{N}_0$, pravimo 
    \textit{Poissonov proces} z intenzivnostjo $\lambda$, če zadošča naslednjim pogojem:
    \begin{enumerate}
        \item $N_0 = 0$ \ $\Prob$-skoraj gotovo.
        \item $(N_t)_{t\geq 0}$ ima neodvisne in stacionarne prirastke,
        \item Za $0 \leq s < t$ velja $ N_t - N_s \sim\Pois{\lambda(t - s)}$,
    \end{enumerate}
  \end{definicija}
\end{frame}

\begin{frame}
  \frametitle{Prirastki}
  \begin{definicija}
    Naj bo $(X_t)_{t\geq0}$ slu"cajni proces. Potem za $s < t$ definiramo
    \textit{prirastek procesa} $X_t - X_s$ na intervalu $[s, t]$. Proces $(X_t)_{t\geq0}$ ima 
    \textit{neodvisne prirastke}, če so za vsak nabor realnih "stevil
    $0 \leq t_1 < t_2 < \ldots < t_n < \infty$ prirastki
    $$
        X_{t_2} - X_{t_1}, \ X_{t_3} - X_{t_2}, \ \ldots, \ X_{t_n} - X_{t_{n-1}}
    $$
    med seboj neodvisni.
    \label{def:prirastek}
    \end{definicija}
    \pause
    \begin{definicija}
    Pravimo, da ima proces $(X_t)_{t\geq0}$
    \textit{stacionarne prirastke}, "ce za vsak $s < t$ in vsak $h > 0$ velja, 
    da ima $X_{t+h} - X_{s+h}$ enako porazdelitev kot $X_t - X_s$.
    \label{def:stacPrir}
    \end{definicija}

  \end{frame}

\begin{frame}
  \frametitle{Sestavljeni Poissonov proces}
  \begin{definicija}
    Naj bo $(N_t)_{t\geq0}$ Poissonov proces z intenzivnostjo $\lambda$. 
    Naj bo $(X_i)_{i\geq1}$ zaporedje neodvisnih (med sabo in $(N_t)_{t\geq0}$) in enako 
    porazdeljenih slučajnih spremenljivk z vrednostmi v $\mathbb{R}$. Potem je 
    \textit{sestavljeni Poissonov proces} $(S_t)_{t\geq0}$ definiran kot
    $$
        S_t = \sum_{i=1}^{N_t} X_i.
    $$
    \label{def:CPP}
\end{definicija}

\end{frame}

\begin{frame}
    \frametitle{Neodvisnost in stacionarnost prirastkov}
    \begin{trditev}
        $CPP$ ima neodvisne in stacionarne prirastke.
        \label{trd:neodvPrirCPP}
    \end{trditev}
    \pause
    \begin{proof}
        Za nabor realnih "stevil $0 \leq t_1 < t_2 < \ldots < t_n < \infty$ lahko slu"cajne
        spremeljivke $S_{t_i} - S_{t_{i-1}}$ zapi"semo kot
        \begin{align*}
            S_{t_i} - S_{t_{i-1}} &= \sum_{j=N_{t_{i-1}}+1}^{N_{t_i}} X_j. 
        \end{align*}
        Neodvisnost prirastkov sledi po neodvisnosti $X_i$ od $X_j$ za $i\neq j$ in $N_t$. 
    \end{proof}
\end{frame}

\begin{frame}
    \frametitle{Neodvisnost in stacionarnost prirastkov}
    \begin{proof}
        Naj bo $h > 0$ in $s < t$. Potem velja
        \begin{align*}
            S_{t+h} - S_{s+h} &= \sum_{j=N_{s+h}+1}^{N_{t+h}} X_j \\
        \end{align*}
        Vsota ima $N_{t+h} - N_{s+h}$ členov. Ker za $HPP(\lambda)$ velja 
        $N_{t+h} - N_{s+h} \sim N_t - N_s$, je 
        \begin{align*}
            \sum_{j=N_{s+h}+1}^{N_{t+h}} X_j = \sum_{j=N_{s}+1}^{N_{t}} X_j = S_t - S_s.
        \end{align*}
    \end{proof}
\end{frame}



\begin{frame}
    \frametitle{Pri"cakovana vrednost in varianca}
    \begin{trditev}
        Naj bo $(S_t)_{t\geq 0}$ $CPP$ in naj bosta $\mu = \E\left[X_i\right] < \infty$ 
        pri"cakovana vrednost in $\sigma^2= \Var{X_i} <\infty$ varianca
        slu"cajnih spremenljivk $X_i$ za vsak $i$. Potem sta za $t\geq0$ pri"cakovana vrednost in 
        varianca $S_t$ enaki 
        \begin{equation*}
            \E\left[S_t\right] = \mu\lambda t \qquad \text{in} \qquad \Var{S_t} = \lambda t\left(\sigma^2 + \mu^2\right).
        \end{equation*}
        \label{trd:PricVarCPP}
    \end{trditev}
\end{frame}


\begin{frame}
  \frametitle{Karakteristi"cna funkcija}
  \begin{trditev}
    Naj bo $(S_t)_{t\geq0}$ $CPP$. Naj bodo slu"cajne spremenljivke $X_i$, ki jih se"stevamo v 
    $CPP$ enako porazdeljene kot $X$. Potem ima za $t\geq0$ karakteristi"cna funkcija $\varphi_{S_t}$ 
    obliko
    \begin{equation*}
        \varphi_{S_t}(u) = e^{\lambda t\left(\varphi_X(u) - 1\right)}, 
    \end{equation*}
    kjer $\varphi_X$ ozna"cuje karakteristi"cno funkcijo $X$.
    \label{trd:MomentGener}
    \end{trditev}
\end{frame}

\begin{frame}
    \frametitle{Nekaj o porazdelitvi}
    \begin{trditev}
        Naj bo $N\sim \Pois{\lambda}$  za $\lambda >0$ in $X_1, X_2, \dots X_n$ neodvisne s.s. (neodvisne 
        med sabo in od $N$) enako porazdeljene kot
        $$ X\sim
        \begin{pmatrix}
            a_1 & a_2 & a_3  \dots & \\
            \tfrac{\lambda_1}{\lambda} & \tfrac{\lambda_2}{\lambda} & \tfrac{\lambda_3}{\lambda} \dots & 
        \end{pmatrix},
        $$
        za neko zaporedje realnih "stevil $(a_n)_{n\in\N}$ in 
        zaporedje poztivnih realnih "stevil $(\lambda_n)_{n\in\N}$ za katere velja 
        ${\sum_{i=1}^\infty\lambda_i = \lambda}$.
        Potem velja 
        \begin{equation*}
            \sum_{j=1}^\infty a_jY_j \sim \sum_{j=1}^NX_j,
        \end{equation*}
        kjer so $Y_1, \dots Y_n$ neodvisne s.s.\ porazdeljene kot 
        $\Pois{\lambda_1}, \Pois{\lambda_2} \dots $
        \label{trd:NXjeEnakoaY}
    \end{trditev}
\end{frame}

\begin{frame}
    \frametitle{Cramér--Lundbergov model: Teorija propada}
    \begin{definicija}
        Naj bo $(S_t)_{t\geq0}$ $CPP$.
        \textit{proces tveganja} v Cramér--Lundbergovem modelu definiramo kot
        \begin{align*}
            U_t = u + p(t) - S_t,
        \end{align*}
        kjer je $u \geq 0$ za"cetni kapital in $p(t)$ funkcija prihodkov. 
        \label{def:procesTveganja}
    \end{definicija}
\end{frame}

\end{document}